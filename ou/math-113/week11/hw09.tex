
\documentclass[fleqn,addpoints]{exam}

\usepackage{amsmath}
\usepackage{graphics}
\usepackage{cancel}
\usepackage{polynom}
\usepackage{mdwlist}

% \printanswers

\ifprintanswers
\usepackage{2in1, lscape}
\fi

\title{Math 113 Homework Nine}
\author{}
\date{\today}


\begin{document}

\maketitle

\section{From the Book}

\begin{itemize*}
  \item Read pages 186-201.
  \item pp. 194-196: 10-14, 30, 35, 45-49, 55, 58, 60
  \item pp. 201-203: 15-19, 30-34, 45, 46
\end{itemize*}

\section{Extra Credit}

\begin{questions}

\question
A 100 pound bag of potatoes is 99\% water.  After sitting out in the sun for a while, some of the water
evaporates, so the potatoes are only 98\% water.  How much does the bag now weigh?

\begin{solution}
Since the bag starts out 99\% water, it initially contains 99 pounds of water and 1 pound of other stuff.  

After some of the water evaporates, the other stuff remains, so there is still one pound of other stuff.  We also know
that the bag is now 2\% other stuff, since it is now 98\% water.  If we let $x$ be the weight of the bag after the
evaporation, we know that:

\( .02x = 1 \)

Solving for $x$:
\begin{align*}
  .02x = 1 \\
  \frac{2x}{100} = 1 \\  
  \frac{x}{50} = 1 \\  
  x = 50 \\  
\end{align*}

So the bag now weighs 50 pounds.

\end{solution}

\question 

An unusual parlor trick is performed as follows.  Ask spectator A to jot down any three-digit number, and then to repeat
the digits in the same order to make a six-digit number. (for example, 394,394).  With your back turned so that you
cannot see the number, ask A to pass the sheet of paper to spectator B, who is requested to divide the number by 7.

``Don't worry about the remainder,'' you tell him, ``because there won't be any.''  B is surprised to discover that you
are right.  Without telling you the result, he passes it on to spectator C, who is told to divide it by 11.  Once again,
you state that there will be no remainder, and this also proves correct.

With your back still turned, and no knowledge whatever of the figures obtained by these computations, you direct a
fourth spectator, D, to divide the last result by 13.  Again the division comes out even.  This final result is written
on a slip of paper which is folded and handed to you.  Without opening it, you pass it on to spectator A.

``Open this,'' you tell him, ``and you will find your original three-digit number.''

Prove that the trick cannot fail to work regardless of the digits chosen by the first spectator.

\begin{solution}
Here is the procedure for constructing the original 6-digit number, written in a little more detail:
\begin{enumerate*}
  \item shift the original 3-digit number to the left three digits
  \item add the original 3-digit number to the result
\end{enumerate*}

Shifting to the left 3 digits is the same as multiplying by 1000.  So if the original number is $n$, what you end up
with is: 

\( 1000n + n  = 1001n \).

The trick is to do a bunch of operations on $1001n$, leaving $n$ in the end.  We can see why it works, if we factor
$1001n$:  

$1001n = 7 \cdot 11 \cdot 13 \cdot n$

Spectators B, C, and D each cancel one of the factors, and in the end you are
left with A's original 3-digit number.

\end{solution}

\end{questions}

\ifprintanswers


\begin{description}

\subsection{Pages 194-196}

\item[10] 
\begin{align*}
  \frac{3n}{n^2-36} - \frac{2}{5n + 30} &= \frac{3n}{(n+6)(n-6)} - \frac{2}{5(n+6)} \\
  &= \frac{3n}{(n+6)(n-6)} \left( \frac{5}{5} \right) - \frac{2}{5(n+6)} \left( \frac{n-6}{n-6} \right) \\
  &= \frac{5 \cdot 3n - 2(n-6)}{5(n+6)(n-6)} \\
  &= \frac{15n - 2n + 12}{5(n+6)(n-6)} \\
  &= \frac{13n + 12}{5(n+6)(n-6)} \\
\end{align*}

\item[11]
\begin{align*}
  \frac{5}{x} - \frac{5x-30}{x^2+6x} + \frac{x}{x+6} &= \frac{5}{x} - \frac{5x-30}{x(x+6)} + \frac{x}{x+6} \\
  &= \frac{5}{x} \left( \frac{x+6}{x+6} \right) - \frac{5x-30}{x(x+6)} + \frac{x}{x+6} \left( \frac{x}{x} \right) \\
  &= \frac{5x+30 - (5x - 30) + x^2}{x(x+6)} \\
  &= \frac{5x + 30 - 5x + 30 + x^2}{x(x+6)} \\
  &= \frac{x^2 + 60}{x(x+6)} \\
\end{align*}


\item[12]
\begin{eqnarray*}
  && \frac{3}{x+1} + \frac{x+5}{x^2-1} - \frac{3}{x-1} \\
  &=& \frac{3}{x+1} \left( \frac{x-1}{x-1} \right) + \frac{x+5}{(x+1)(x-1)} - \frac{3}{x-1} \left( \frac{x+1}{x+1} \right) \\
  &=& \frac{3(x-1) + x + 5 - 3(x + 1)}{(x+1)(x-1)} \\
  &=& \frac{3x - 3 + x + 5 - 3x - 3}{(x+1)(x-1)} \\
  &=& \frac{\cancel{(x-1)}}{(x+1)\cancel{(x-1)}} \\
  &=& \frac{1}{x+1} \\
\end{eqnarray*}

\item[13]
\begin{eqnarray*}
  && \frac{3}{x^2+9x+14} + \frac{5}{2x^2+15x+7} \\
  &=& \frac{3}{(x+2)(x+7)} + \frac{5}{(2x+1)(x+7)} \\
  &=& \frac{3}{(x+2)(x+7)} \left( \frac{2x+1}{2x+1} \right) + \frac{5}{(2x+1)(x+7)} \left( \frac{x+2}{x+2} \right) \\
  &=& \frac{3(2x+1) + 5(x+2)}{(x+2)(x+7)(2x+1)} \\
  &=& \frac{6x+3 + 5x+10}{(x+2)(x+7)(2x+1)} \\
  &=& \frac{11x+13}{(x+2)(x+7)(2x+1)} \\
\end{eqnarray*}

\item[14]
\begin{eqnarray*}
  && \frac{6}{x^2+11x+24} + \frac{4}{3x^2+13x+12} \\
  &=& \frac{6}{(x+8)(x+3)} + \frac{4}{(3x+4)(x+3)} \\
  &=& \frac{6}{(x+8)(x+3)} \left(\frac{3x+4}{3x+4}\right) + \frac{4}{(3x+4)(x+3)} \left(\frac{x+8}{x+8}\right) \\
  &=& \frac{6(3x+4) + 4(x+8)}{(x+8)(x+3)(3x+4)} \\
  &=& \frac{18x+24+4x+32}{(x+8)(x+3)(3x+4)} \\
  &=& \frac{22x+56}{(x+8)(x+3)(3x+4)} \\
\end{eqnarray*}

\item[30]
\begin{eqnarray*}
  &&\frac{2x+5}{x^2+3x-18} - \frac{3x-1}{x^2+4x-12} + \frac{5}{x-2} \\
  &=& \frac{2x+5}{(x+6)(x-3)} - \frac{3x-1}{(x+6)(x-2)} + \frac{5}{x-2} \\
  &=& \frac{2x+5}{(x+6)(x-3)} \left( \frac{x-2}{x-2} \right) - \frac{3x-1}{(x+6)(x-2)} \left( \frac{x-3}{x-3} \right) \\
  && + \frac{5}{x-2} \left( \frac{(x+6)(x-3)}{(x+6)(x-3)} \right) \\
  &=& \frac{(2x+5)(x-2) - (3x-1)(x-3) + 5(x+6)(x-3)}{(x+6)(x-3)(x-2)} \\
  &=& \frac{2x^2+x-10 - (3x^2-10x+3) + 5(x^2+3x-18)}{(x+6)(x-3)(x-2)} \\
  &=& \frac{2x^2+x-10 - 3x^2 +10x -3 + 5x^2+15x-90}{(x+6)(x-3)(x-2)} \\
  &=& \frac{-x^2+11x-13 + 5x^2+15x-90}{(x+6)(x-3)(x-2)} \\
  &=& \frac{4x^2+26x-103}{(x+6)(x-3)(x-2)} \\
\end{eqnarray*}

\item[35]
\begin{eqnarray*}
  &&  \frac{t+3}{3t-1} + \frac{8t^2+8t+2}{3t^2-7t+2} - \frac{2t+3}{t-2} \\
  &=& \frac{t+3}{3t-1} + \frac{8t^2+8t+2}{(3t-1)(t-2)} - \frac{2t+3}{t-2} \\
  &=& \frac{t+3}{3t-1} \left( \frac{t-2}{t-2} \right)  + \frac{8t^2+8t+2}{(3t-1)(t-2)}
             - \frac{2t+3}{t-2} \left( \frac{3t-1}{3t-1} \right) \\
  &=& \frac{(t+3)(t-2) + 8t^2+8t+2 - (2t+3)(3t-1)}{(3t-1)(t-2)} \\
  &=& \frac{t^2+t-6 + 8t^2+8t+2 - (6t^2+7t-3)}{(3t-1)(t-2)} \\
  &=& \frac{9t^2+9t-4 - 6t^2-7t+3)}{(3t-1)(t-2)} \\
  &=& \frac{3t^2+2t-1}{(3t-1)(t-2)} \\
  &=& \frac{\cancel{(3t-1)}(t+1)}{\cancel{(3t-1)}(t-2)} \\
  &=& \frac{t+1}{t-2} \\
\end{eqnarray*}

\item[45]
\begin{align*}
  \cfrac{ \cfrac{6}{a} - \cfrac{5}{b^2} } { \cfrac{12}{a^2} + \cfrac{2}{b} } &= 
       \left( \frac{a^2b^2}{a^2b^2} \right) \cfrac{ \cfrac{6}{a} - \cfrac{5}{b^2} } { \cfrac{12}{a^2} + \cfrac{2}{b} } \\
  &= \frac{6ab^2 - 5a^2}{12b^2 + 2a^2b}
\end{align*}

\item[46]
\begin{align*}
  \cfrac{ \cfrac{4}{ab} - \cfrac{3}{b^2} } { \cfrac{1}{a} + \cfrac{3}{b} } &=
      \cfrac{ \cfrac{4b-3a}{ab^2} } { \cfrac{b+3a}{ab} } \\
  &= \frac{4b-3a}{ab^2} \cdot \frac{ab}{b+3a} \\
  &= \frac{4b-3a}{b(b+3a)} \\
\end{align*}

\item[47]
\begin{align*}
  \cfrac{ \cfrac{2}{x} - 3}{\cfrac{3}{y} + 4} &= \cfrac{ \cfrac{2}{x} - \cfrac{3}{1}}{\cfrac{3}{y} + \cfrac{4}{1}} \\
  &= \left( \frac{xy}{xy} \right) \cfrac{ \cfrac{2}{x} - \cfrac{3}{1}}{\cfrac{3}{y} + \cfrac{4}{1}} \\
  &= \frac{2y-3xy}{3x+4xy} \\
\end{align*}

\item[48]
\begin{align*}
  \cfrac{1 + \cfrac{3}{x}} {1 - \cfrac{6}{x}} &= \cfrac{\cfrac{1}{1} + \cfrac{3}{x}} {\cfrac{1}{1} - \cfrac{6}{x}} \\
  &= \left( \frac{x}{x} \right)\cfrac{\cfrac{1}{1} + \cfrac{3}{x}} {\cfrac{1}{1} - \cfrac{6}{x}} \\
  &= \frac{x+3}{x-6} \\
\end{align*}

\item[49]
\begin{align*}
  \cfrac{3 + \cfrac{2}{n+4}}{5 - \cfrac{1}{n+4}} &= \cfrac{\cfrac{3}{1} + \cfrac{2}{n+4}}{\cfrac{5}{1} - \cfrac{1}{n+4}} \\
  &= \left( \frac{n+4}{n+4} \right)\cfrac{\cfrac{3}{1} + \cfrac{2}{n+4}}{\cfrac{5}{1} - \cfrac{1}{n+4}} \\
  &= \frac{3(n+4) + 2}{5(n+4) - 1} \\
  &= \frac{3n+14}{5n+19} \\
\end{align*}

\item[55]
\begin{align*}
  \cfrac{\cfrac{2}{x-3} - \cfrac{3}{x+3} } { \cfrac{5}{x^2-9} - \cfrac{2}{x-3} } 
      &= \cfrac{\cfrac{2}{x-3} - \cfrac{3}{x+3} } { \cfrac{5}{(x+3)(x-3)} - \cfrac{2}{x-3} } \\
  &= \cfrac{\cfrac{2}{x-3} - \cfrac{3}{x+3} } { \cfrac{5}{(x+3)(x-3)} - \cfrac{2}{x-3} } 
          \left( \frac{(x+3)(x-3)}{(x+3)(x-3)} \right) \\
  &= \frac{2(x+3) - 3(x-3)} { 5 - 2(x+3)} \\
  &= \frac{2x+6-3x+9} {5-2x-6} \\
  &= \frac{-x+15} {-2x-1} \\
  &= \left( \frac{-1}{-1} \right)\frac{-x+15} {-2x-1} \\
  &= \frac{x-15} {2x+1} \\
\end{align*}

\item[58]
\begin{align*}
  \cfrac{a}{\cfrac{1}{a}+4} + 1 &= \cfrac{a}{\cfrac{1+4a}{a}} + 1 \\
  &= a \left( \frac{a}{1+4a} \right) + 1 \\
  &= \frac{a^2}{1+4a} + 1 \\
  &= \frac{a^2}{1+4a} + \frac{1+4a}{1+4a} \\
  &= \frac{a^2+4a+1}{4a+1} \\
\end{align*}

\item[60]
\begin{align*}
  1 + \cfrac{x}{1 + \cfrac{1}{x}} &= 1 + \cfrac{x}{\cfrac{x+1}{x}} \\
  &= 1 + x \left( \frac{x}{x+1} \right) \\
  &= 1 + \frac{x^2}{x+1} \\
  &= \frac{x+1}{x+1} + \frac{x^2}{x+1} \\
  &= \frac{x^2+x+1}{x+1} \\
\end{align*}

\subsection{Pages 201-203}

I had a hard time getting a few of them to display the long division, so some of the answers are just the solution without the work.

\item[15]
\[ \polylongdiv{2x^2-x-4}{x-1} \]
\[ 2x+1 - \frac{3}{x-1}\]

\item[16]
\[ \polylongdiv{3x^2-2x-7}{x+2} \]
\[ 3x-8 + \frac{9}{x+2}\]

\item[17]
\[ \polylongdiv{15x^2+22x-5}{3x+5} \]

\item[18]
\[ \polylongdiv{12x^2-32x-35}{2x-7} \]

\item[19]
\[ \polylongdiv{3x^3+7x^2-13x-21}{x+3} \]

\item[30]
\[ \polylongdiv{x^3-8}{x-4} \]
\[ x^2+4x+16 + \frac{56}{x-4} \]

\item[31]
\[ \polylongdiv{2x^3-x-6}{x+2} \]
\[ 2x^2-4x+7 - \frac{20}{x+2} \]

\item[32]
\[ \polylongdiv{5x^3+2x-3}{x-2} \]
\[ 5x^2+10x+22 + \frac{41}{x-2} \]

\item[33]
\[ \frac{4a^2-8ab+4b^2}{a-b} = 4a - 4b \]

\item[34]
\[ \polylongdiv{3x^2-2xy-8y^2}{x-2y} \]

\item[45]
\[ \frac{2n^4+3n^3-2n^2+3n-4}{n^2+1} = 2n^2+3n-4\]


\item[46]
\[ \frac{3n^4+n^3-7n^2-2n+2}{n^2-2} = 3n^2+n-1 \]

\item[50]
\[ \polylongdiv{x^4-1}{x-1} \]

\end{description}

\fi


\ifprintanswers
\else
\vspace{5 cm}

{\em Do not worry about your difficulties in Mathematics. I can assure you mine are still greater.}

\vspace{0.1 in}
\hspace{0.5 in} --Albert Einstein

%% % {\em Imagination is more important than knowledge.}

\fi

\end{document}
