\documentclass[fleqn]{article}
\usepackage{amsmath}

\title{Math 113--Week Three Notes}
\author{}
\date{February 3, 2010}

\oddsidemargin 0in
\topmargin -0.5in
\textwidth 6.5in
\textheight 9in

\setlength{\mathindent}{1in}

\begin{document}

\maketitle

\section{Non-Word Problems}

Most of the problems which aren't word problems just require practice.  The steps required to any problem are similar to
the steps required to solve other problems.  Once you get the hang of doing it, you just need to practice a bit
so you can solve the problems quickly and easily.

Although most of the steps are fairly mechanical, there are many opportunities to make a careless mistake.  I found when
I was making up the answer key that my first attempt often contained some sort of mistake with the sign, failing to copy
the equation correctly from one step to the next, etc.  

Unless you are much more careful than I am, you are likely to also make similar mistakes.  Fortunately, it is easy to
find out if something went wrong.  All you need to do is plug your answer back into the equation and see if everything
works out correctly.  Usually, I try to do something slightly different when I am checking my work.  This way, if I messed
something up finding the solution, I'm less likely to make the same mistake checking the solution.

If you plug your number back into the equation and it works out correctly, you can be fairly confident you have the
right answer.  When you get the answer to the original equation, you are actually only half done with solving the
problem.  Equally important is checking your work.  You are only fully done when you have gotten the answer and verified
that it is correct.

\section{Word Problems}

Several people have mentioned that they find word problems to be troublesome.  Here are a few suggestions which may help.

Reading a word problem is a little different from reading a novel.  Usually every phrase in the problem means
something and needs to be accounted for in the equation.  But don't be overwhelmed by trying to make sense of the entire
sentence at once.  Just take the sentence a phrase at a time and turn each phrase into a part of the equation.

For example:

``Find a number such that one-half of the number is 3 less than two-thirds of the number.''

Taking the sentence one phrase at a time:

\begin{tabular}{|c|c|}
\hline
  English       & Math \\
\hline
  find a number & $x$ \\
  such that     & here come the conditions that the {\em x} must satisfy \\
  one-half of the number & $1/2 \cdot x$ \\
  is & =  \\
  3 less than & $-3$ \\
  two-thirds of the number & $2/3 \cdot x$ \\
\hline
\end{tabular}

\vspace{.5 cm}

Then you just need to take the entries from the ``Math'' column and write them in a row:

\( \displaystyle \frac{x}{2} = -3 + \frac{2}{3} x \)

Solving for $x$ gives: $x = 18$.

You should always go back and check your answer with the original problem.  One-half of 18 is 9.  Two-thirds of 18 is
12.  9 is three less than 12, so the answer is correct.






\end{document}


