
\documentclass[fleqn,addpoints]{exam}

\usepackage{amsmath}
\usepackage{graphics}
\usepackage{cancel}
\usepackage{polynom}
\usepackage{mdwlist}
\usepackage{2in1, lscape}

\title{Math 113 Chapter Five Study Guide}
\author{}
\date{\today}

\begin{document}

\maketitle

\section{Section 5.1--Integer Exponents}

\subsection{Notes}
\begin{itemize*}
   \item when multiplying, add exponents 
   \item when dividing, subtract exponents 
   \item when raising to a power, multiply exponents 
   \item when moving a factor from the denominator to the numerator or from the numerator to the denominator, change the
     sign of the exponent 
\end{itemize*}

\subsection{Examples}
\begin{itemize}
  \item \( (2^{-3})^2 \)
%  \item \( \left( \frac{3^2}{5^{-1}} \right)^{-1} \)
  \item \( \left( \dfrac{2a^2b^-2}{5b^{1}a^3} \right)^{-2} \)
  \item \( (2xy^{-2})(3x^{-2}y^3) \)
\end{itemize}

\section{Section 5.2 and 5.3--Roots and Radicals}

\subsection{Notes}
Simplest radical form means:
\begin{itemize*}
    \item no fraction with a radical sign
    \item no radical in the denominator
    \item all the radicals in the numerator must be simplified as much as possible
\end{itemize*}

\subsection{Examples}
\begin{itemize}
%  \item \( \sqrt{27} \)
  \item \( \dfrac{1}{3}\sqrt{90} \)
  \item \( \sqrt{\dfrac{22}{9}} \)
  \item \( \dfrac{6\sqrt{5}}{5\sqrt{12}}\)
  \item \( 3\sqrt{24} + 2\sqrt{54} - 12\sqrt{6} \)
%  \item \( \sqrt{32x^5} \)
  \item \( 4\sqrt{ab} - \sqrt{16ab} + 10\sqrt{25ab} \)
\end{itemize}

\section{Section 5.4--Products and Quotients Involving Radicals}

\subsection{Notes}

\begin{itemize*}
  \item radicals with the same index may be combined with multiplication or division
  \item if you have something like $\sqrt{x} + \sqrt{y}$ in the denominator of a fraction, you can make the radicals go
    away by multiplying the top and bottom by $\sqrt{x} - \sqrt{y}$
\end{itemize*}

\subsection{Examples}
\begin{itemize}
  \item \( \sqrt{2} (\sqrt{3} + \sqrt{5}) \)
  \item \( (\sqrt{3} + \sqrt{6})(\sqrt{3} - 2\sqrt{7}) \)
  \item \( \displaystyle \frac{\sqrt{7}}{2\sqrt{3} - 5} \)
\end{itemize}


\section{Section 5.5--Equations Involving Radicals}

\subsection{Notes}

To solve equations involving radicals:
\begin{itemize*}
  \item use addition or subtraction to make the left side and the right side each contain at most one radical, if possible
  \item square both sides
  \item repeat if necessary if there are any radicals left
  \item solve using the techniques we learned earlier  
  \item check the solutions and discard any that don't actually work in the original equation
\end{itemize*}

\subsection{Examples}
\begin{itemize}
  \item \( \sqrt{5x} = 10\)
  \item \( \sqrt{7x-6} - \sqrt{5x+2} = 0 \)
  \item \( \sqrt{t+3} - \sqrt{t-2} = \sqrt{7-t} \)
\end{itemize}

\section{Section 5.6--Merging Exponents and Roots}

\subsection{Notes}

\begin{itemize*}
  \item a fractional exponent is an alternate notation for a radical
  \item the rules from Section 5.1 for integer exponents also apply to fractional exponents
\end{itemize*}

\subsection{Examples}
\begin{itemize}
 \item \( 25^{\frac{1}{2}}\)
  \item \( \left( \dfrac{8}{125} \right)^{\frac{2}{3}} \)
  \item \( (3x^{\frac{1}{4}}y^{\frac{1}{5}})^3 \)
\end{itemize}

\section{Section 5.7--Scientific Notation}

\subsection{Notes}
\begin{itemize*}
%  \item scientific notation is a convenient way to write very large or very small numbers
  \item in scientific notation, numbers are written as $(n)(10)^a$ where $n$ is number between 1 and 10 and $a$ is an integer.  To convert
    back to standard notation:
    \begin{itemize*}
        \item if $a$ is positive move the decimal point to the right $a$ places
        \item if $a$ is negative move the decimal point to the left $a$ places
    \end{itemize*}
  \item calculations with large or small numbers are often easier if you convert the numbers to scientific notation first
\end{itemize*}

\subsection{Examples}
\begin{itemize}
  \item \( 40,000,000\) and \( 0.0025 \) 
  \item \( (0.0003)(.00005) \) 
\end{itemize}

\end{document}
