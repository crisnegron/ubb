
\documentclass[fleqn,addpoints]{exam}

\usepackage{amsmath}
\usepackage{graphics}
\usepackage{cancel}
\usepackage{polynom}
\usepackage{mdwlist}

\printanswers

\ifprintanswers
\usepackage{2in1, lscape}
\fi

\title{Math 113 Homework 10}
\author{}
\date{\today}


\begin{document}

\maketitle

%% \section{Chapter Four Test}

%% The Chapter Four test will be next week.  See the {\em Study Guide} for a summary of material from this chapter.

\section{From the Book}

\begin{itemize*}
  \item Read pages 203-220.
  \item pp. 209-211: 20-24, 40-44, 45, 46, 51, 56, 60
  \item pp. 218-220: 15-19, 31, 38, 48, 52, 54
\end{itemize*}

\ifprintanswers

\begin{description}

\subsection{Pages 209-211}

\item[20] 
\begin{align*}
  \frac{3}{2x-1} &= \frac{5}{3x+2} \\
  3(3x+2) &= 5(2x-1) \\
  9x+6 &= 10x-5 \\
  -x &= -11 \\
  x &= 11 \\
\end{align*}

\item[21]
\begin{align*}
  \frac{-2}{x-5} &= \frac{1}{x+9} \\
  -2(x+9) &= x-5 \\
  -2x-18 &= x-5 \\
  -3x = 13 \\
  x = - \frac{13}{3} \\
\end{align*}

\item[22]
\begin{align*}
  \frac{5}{2a-1} &= \frac{-6}{3a+2} \\
  5(3a+2) &= -6(2a-1) \\
  15a+10 &= -12a+6 \\
  27a = -4 \\
  a = -\frac{4}{27} \\
\end{align*}

\item[23]
\begin{align*}
  \frac{x}{x+1} - 2 &= \frac{3}{x-3} \\
   \frac{x(x-3)}{x+1} - 2(x+1)(x-3) &= 3(x+1) \\
   x(x-3) - 2(x+1)(x-3) &= 3(x+1) \\
   x^2-3x - 2(x^2-2x-3) &= 3x+3 \\
   x^2-3x - 2x^2 + 4x + 6 &= 3x+3 \\
   -x^2 + x + 3 &= 3x \\
   -x^2 - 2x + 3 &= 0 \\
   x^2 + 2x - 3 &= 0 \\
   (x+3)(x-1) &= 0 \\
\end{align*}

\( x = \{1, -3\} \)

\item[24]
\begin{align*}
  \frac{x}{x-2} + 1 &= \frac{8}{x-1} \\
  \frac{x}{x-2} + \frac{x-2}{x-2} &= \frac{8}{x-1} \\
  \frac{x + x - 2}{x-2}  &= \frac{8}{x-1} \\
  \frac{2x - 2}{x-2}  &= \frac{8}{x-1} \\
  (2x-2)(x-1) &= 8(x-2) \\
  2x^2-4x+2 &= 8x-16 \\
  2x^2-12x+18 &= 0 \\
  x^2-6x+9 &= 0 \\
  (x-3)(x-3) &= 0 \\
\end{align*}

\( x = 3 \)

\item[40]
\begin{align*}
  \frac{n}{n+1} + \frac{1}{2} &= \frac{-2}{n+2} \\
  \frac{2n + n + 1}{2(n+1)}  &= \frac{-2}{n+2} \\
  \frac{3n + 1}{2(n+1)}  &= \frac{-2}{n+2} \\
  (3n+1)(n+2) &= -4(n+1) \\
  3n^2+7n+2 &= -4n-4 \\
  3n^2+11n+6 &= 0 \\
  (3n+2)(n+3) &= 0 \\
\end{align*}

\[ n = \{ - \frac{2}{3}, -3 \} \]

\item[41]
\begin{align*}
  \frac{-3}{4x+5} &= \frac{2}{5x-7} \\
  -3(5x-7) &= 2(4x+5) \\
  -15x+21 &= 8x+10 \\
  -23x &= -11 \\
  x &= \frac{11}{23}
\end{align*}

\item[42]
\begin{align*}
  \frac{7}{x+4} &= \frac{3}{x-8} \\
  7(x-8) &= 3(x+4) \\
  7x-56 &= 3x+12 \\
  4x &= 68 \\
  x &= 17 \\
\end{align*}

\item[43]
\begin{align*}
  \frac{2x}{x-2} + \frac{15}{x^2-7x+10} &= \frac{3}{x-5} \\
  \frac{2x}{x-2} + \frac{15}{(x-5)(x-2)} &= \frac{3}{x-5} \\
  2x(x-5) + 15 &= 3(x-2) \\
  2x^2-10x + 15 &= 3x-6 \\
  2x^2-13x + 21 &= 0 \\
  (2x - 7)(x - 3) &= 0 \\
\end{align*}

\[ x = \{ 3, \frac{7}{2} \} \]

\item[44]
\begin{align*}
  \frac{x}{x-4} - \frac{2}{x+3} &= \frac{20}{x^2-x-12} \\
  \frac{x}{x-4} - \frac{2}{x+3} &= \frac{20}{(x-4)(x+3)} \\
  x(x+3) - 2(x-4) &= 20 \\
  x^2+3x - 2x + 8 &= 20 \\
  x^2 +x + 8 &= 20 \\
  x^2 +x - 12 &= 0 \\
  (x+4)(x-3) &= 0 \\
\end{align*}

\( x = \{-4, 3\} \)

\item[45]
\begin{align*}
  \frac{x}{1750 - x} &= \frac{3}{4} \\
  4x &= 3(1750 - x) \\
  4x &= 5250 - 3x \\
  7x &= 5250 \\
  x  &= 750 \\
\end{align*}

One person gets \$750 and the other gets \$1,000.

\item[46]

Since 1 inch = 5 feet, to convert from inches on the map to feet, we just need to multiply by 5.

The width is $\displaystyle w = \frac{7}{2} \cdot 5 = \frac{35}{2} = 17.5$ feet.

The length is $\displaystyle w = \frac{23}{4} \cdot 5 = \frac{115}{4} = 28.75$ feet

\item[51]
\begin{align*}
  \frac{900}{50,000} &= \frac{x}{60,000} \\
  x &= 60,000 \cdot \frac{900}{50,000} \\
  x &= 6 \cdot \frac{900}{5} \\
  x &= 1,080 \\
\end{align*}

\item[56]
\begin{align*}
  \frac{2 + x}{5 + x} &= \frac{7}{8} \\
  8(x+2) &= 7(x+5) \\
  8x+16 &= 7x+35 \\
  x &= 19 \\
\end{align*}

\item[60]

The perimeter is $114 = 2L + 2W$.  Solving for $L$, $L = 57 - W$.
\begin{align*}
  \frac{7}{12} &= \frac{W}{57-W} \\
  7(57-W) &= 12W \\
  399 - 7W &= 12W \\
  19W &= 399 \\
  W &= 21 \\
\end{align*}

The length is: $L = 57 - 21 = 36$

We can check everything: $2 \cdot 36 + 2 \cdot 21 = 114$ and $\displaystyle \frac{21}{36} = \frac{7}{12}$.

\subsection{Pages 218-219}

\item[15]
\begin{align*}
  \frac{2x}{x+3} - \frac{3}{x-6} &= \frac{29}{x^2-3x-18} \\
  \frac{2x}{x+3} - \frac{3}{x-6} &= \frac{29}{(x+3)(x-6)} \\
  2x(x-6) - 3(x+3) &= 29 \\
  2x^2 - 12x - 3x - 9 &= 29 \\
  2x^2 -9x - 38 &= 0 \\
  (2x-19)(x+2) &= 0 \\
\end{align*}

\( \displaystyle x = \{ -2, \frac{19}{2} \} \)

\item[16]
\begin{align*}
  \frac{x}{x-4} - \frac{2}{x+8} &= \frac{63}{x^2+4x-32} \\
  \frac{x}{x-4} - \frac{2}{x+8} &= \frac{63}{(x-4)(x+8)}
  x(x+8) - 2(x-4) &= 63 \\
  x^2 + 8x - 2x + 8 &= 63 \\
  x^2 + 6x -55 &= 0 \\
  (x+11)(x-5) &= 0 \\
\end{align*}

\( x = \{ -11, 5 \} \)

\item[17]
\begin{align*}
  \frac{a}{a-5} + \frac{2}{a-6} &= \frac{2}{a^2-11a+30} \\
  \frac{a}{a-5} + \frac{2}{a-6} &= \frac{2}{(a-5)(a-6)} \\
  a(a-6) + 2(a-5) &= 2 \\
  a^2-6a+2a-10 &= 2 \\
  a^2-4a-12 &= 0 \\
  (a-6)(a+2) &= 0 \\
\end{align*}

The possible solutions are 6 and -2, but 6 would make a few of the denominators 0, so -2 is the only valid solution.

\item[18]
\begin{align*}
  \frac{a}{a+2} + \frac{3}{a+4} &= \frac{14}{a^2+6a+8} \\
  \frac{a}{a+2} + \frac{3}{a+4} &= \frac{14}{(a+2)(a+4)} \\
  a(a+4) + 3(a+2) &= 14 \\
  a^2 + 4a + 3a + 6 &= 14 \\
  a^2 + 7a - 8 &= 0 \\
  (a+8)(a-1) &= 0 \\
\end{align*}

\( \displaystyle a = \{ -8, 1 \} \)

\item[19]
\begin{align*}
  \frac{-1}{2x-5} + \frac{2x-4}{4x^2-25} &= \frac{5}{6x+15} \\
  \frac{-1}{2x-5} + \frac{2x-4}{(2x+5)(2x-5)} &= \frac{5}{3(2x+5)} \\
  -3(2x+5) + 3(2x-4) &= 5(2x-5) \\
  -6x-15 + 6x-12 &= 10x-25 \\
  -27 &= 10x-25 \\
  10x &= -2 \\
  x &= -\frac{1}{5} \\
\end{align*}

\item[31]
\begin{align*}
  y &= \frac{5x}{6} + \frac{2}{9} \\
  18y &= 18 \left( \frac{5x}{6} + \frac{2}{9} \right) \\
  18y &= 15x + 4 \\
  15x &= 18y - 4 \\
  x &= \frac{18y - 4}{15} \\
\end{align*}

\item[38]
\begin{align*}
  \frac{1}{R} &= \frac{1}{S} + \frac{1}{T} \\
  (RST) \left( \frac{1}{R} \right) &= RST \left( \frac{1}{S} + \frac{1}{T} \right) \\
  ST &= RT + RS \\
  ST &= R(T + S) \\
  R &= \frac{ST}{S+T} \\
\end{align*}

\item[48]

\begin{itemize}
  \item $t_{barry} = 3$ hours
  \item $t_{sanchez} = 5$ hours
  \item $d_{barry} = d_{sanchez} = 1$ job
  \item $r_{barry} = 1/3$ jobs/hour
  \item $r_{sanchez} = 1/5$ jobs/hour
\end{itemize}

The rate when Barry and Sanchez work together is: $r_{together} = 1/3 + 1/5$.  Since $t=d/r$ and $d$ is ``1 job'', the
equation we need to solve is:
\begin{align*}
  t & = \frac{1}{\cfrac{1}{3} + \cfrac{1}{5}} \\
  t & = \left( \frac{15}{15} \right) \left( \frac{1}{\cfrac{1}{3} + \cfrac{1}{5}} \right) \\
  t & = \frac{15}{3 + 5} \\
  t & = \frac{15}{8} \\
\end{align*}

So working together, they take $\displaystyle \frac{15}{8}$ hours (or 1:52:30 in hours/minutes/seconds).

\item[52]

\begin{itemize}
  \item $d_{sue} = 60$ miles
  \item $d_{doreen} = 50$ miles
  \item $t_{sue} = t_{doreen} - 2$ hours
  \item $r_{sue} = r_{doreen} + 10 = \frac{50}{t_{doreen}} + 10$ mph
\end{itemize}


The distance for Sue is: 
\begin{align*}
  d_{sue} &= r_{sue}t_{sue} \\
  & = \left( \frac{50}{t_{doreen}} + 10 \right) (t_{doreen} - 2)  \\
  & = 60 \\
\end{align*}

Solving for $t_{doreen}$:
\begin{align*}
  \left( \frac{50}{t} + 10 \right) (t- 2) &= 60 \\
  \left( \frac{50}{t} + 10 \right) (t- 2) &= 60 \\
  50t - 100 + 10t^2 - 20t &= 60t \\
  10t^2 - 30t - 100 &= 0 \\
  t^2 - 3t - 10 &= 0 \\
  (t-5)(t+2) &= 0 \\
\end{align*}

The solutions are 5 and -2, but -2 doesn't make sense as a time, so 5 is the only solution.  

\begin{itemize}
  \item It took Doreen 5 hours and her rate was $r_{doreen} = 50/5 = 10$ mph.  
  \item It took Sue 3 hours and her rate was $r_{sue} = 60/3 = 20$ mph.  
\end{itemize}

\item[54]

The working together, the pipes can fill the pool in 1 hour and 12 minutes which is 6/5 hour.  The first pipe has a rate
of 1 pool/2 hours.

We need to fill one pool, so the ``distance'' is ``one pool'' 
\begin{align*}
  r_{both} \cdot t_{both} &= 1 \\
  \left( \frac{1}{2} + r \right) \cdot \frac{6}{5} &= 1 \\
  \frac{6}{10} + \frac{6r}{5} &= 1 \\
  6 + 6r &= 10 \\
  6r &= 4 \\
  r &= \frac{2}{3} \\
\end{align*}

So the rate for the other pipe is $2/3$ pools/hour.  And the time for this pipe to fill one pool would be 
$t=d/r = 1/(2/3) = 3/2$ hour.  So the other pipe can fill the pool in 1 hour and 30 minutes.

\end{description}

\fi

%% \section{Review}

%% To review for the test, you should take a look at the {\em Study Guide} and the Chapter 4 {\em Review Problems} and 
%% {\em Sample Test}.

\vspace{5 in}



{\em Imagination is more important than knowledge.}

\vspace{0.1 cm}
\hspace{0.5 cm} --Albert Einstein

\end{document}
