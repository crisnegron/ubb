\documentclass{exam}

\usepackage{units} 
\usepackage{graphicx}
\usepackage[fleqn]{amsmath}
\usepackage{cancel}
\usepackage{float}
\usepackage{mdwlist}
\usepackage{booktabs}
\usepackage{cancel}
\usepackage{polynom}
\usepackage{caption}
\usepackage{fullpage}
\usepackage{comment}
\usepackage{enumerate}
\usepackage{xfrac}

\newcommand{\dg}{\ensuremath{^\circ}} 
\everymath{\displaystyle}

\printanswers
\excludecomment{comment}

\ifprintanswers 
  \usepackage{2in1, lscape} 
\fi

\author{}
\date{\today}
\title{Math 142 \\ Homework Eight}

\begin{document}

  \maketitle

  \section{Homework}
  Section 6.2: 1-12, 15-18, 23-26, 29-32, 39-42, 45-48, 52, 54, 56-57, 59, 61-65

  \section{Extra Credit}
  Section 6.2: 43-44

  \ifprintanswers
    \begin{description}
      \item[43]
        Call the diagonal $a$:
        \begin{align*}
          \tan \theta & = \frac{a}{10} \\
          a           & = 10 \tan \theta \\
        \end{align*}

        The right angle in the bottom left corner is the sum of two smaller angles.  Call the top angle $\alpha$ and the
        bottom angle $\beta$.  $\alpha$ is the third angle in the right triangle that includes $\theta$, so:
        \[
          \alpha = 90 \dg - \theta
        \]

        Since $\alpha$ and $\beta$ form a right triangle:
        \begin{align*}
          \alpha + \beta          & = 90 \dg \\
          90 \dg - \theta + \beta & = 90 \dg \\
          \beta                   & = \theta \\
        \end{align*}

        The expression for $x$ is:
        \begin{align*}
          \sin \theta & = \frac{x}{a} \\
                      & = \frac{x}{10 \tan \theta} \\
          x           & = \boxed{ 10 \sin \theta \tan \theta } \\
        \end{align*}

      \item[44]
        \begin{align*}
          \sin \theta & = \frac{a}{1} \\
          a           & = \sin \theta \\
          \\
          \tan \theta & = \frac{b}{1} \\
          b           & = \tan \theta \\
          \\
          \cos \theta & = \frac{1}{c} \\
          c           & = \sec \theta \\
          \\
          \cos \theta & = \frac{d}{1} \\
          d           & = \cos \theta \\
          \\
        \end{align*}

    \end{description}

  \fi

  \section{Review}
  \begin{enumerate}
    \item If $\tan t = \frac{3}{4}$ and $t$ is in $Q1$, find the values of all six trigonometric functions.

      \begin{solution}
        \begin{align*}
          \frac{y}{x} & = \frac{3}{4} \\
          y           & = \frac{3x}{4} \\
          \\
          x^2 + \left( \frac{3x}{4} \right)^2 &= 1 \\
          \frac{25}{16} x^2 & = 1 \\
          x                 & = \frac{4}{5} \\
          y                 & = \frac{3}{5} \\
        \end{align*}

        \begin{tabular}[H]{ccc}
          \toprule
          $\sin t$      & $\cos t$      & $\tan t$      \\
          $\frac{3}{5}$ & $\frac{4}{5}$ & $\frac{3}{4}$ \\

          \midrule

          $\csc t$      & $\sec t$      & $\cot t$ \\
          $\frac{5}{3}$ & $\frac{5}{4}$ & $\frac{4}{3}$ \\

          \bottomrule
        \end{tabular}

      \end{solution}

    \item A bicycle wheel is rotating at the rate of 1.5 revolutions per second.  If the wheel has a radius of 12
      inches, find an equation for the distance from the ground of a point on the wheel.

      \begin{solution}
        \begin{align*}
          \omega & = 2 \pi f \\
                 & = 2 \pi \times 1.5 \\
                 & = 3 \pi \\
        \end{align*}

        The equation is:
        \[
          h(t) = 12 + 12 \cos 3 \pi t 
        \]
      \end{solution}

  \end{enumerate}

  \ifprintanswers

    \section{Section 6.2}
    \begin{description}

      \item[1] 
        \begin{tabular}[H]{ccc}
          \toprule

          $\sin$         & $\cos$         & $\tan$         \\
          $\sfrac{4}{5}$ & $\sfrac{3}{5}$ & $\sfrac{4}{3}$ \\

          \midrule
          $\csc$         & $\sec$         & $\cot$ \\
          $\sfrac{5}{4}$ & $\sfrac{5}{3}$ & $\sfrac{3}{4}$ \\

          \bottomrule
        \end{tabular}

      \item[2] 
        \begin{tabular}[H]{ccc}
          \toprule

          $\sin$          & $\cos$           & $\tan$          \\
          $\sfrac{7}{25}$ & $\sfrac{24}{25}$ & $\sfrac{7}{24}$ \\

          \midrule

          $\csc$          & $\sec$           & $\cot$ \\
          $\sfrac{25}{7}$ & $\sfrac{25}{24}$ & $\sfrac{24}{7}$ \\

          \bottomrule
        \end{tabular}

      \item[3] 
        Find the missing side:
        \begin{align*}
          x^2 + 40^2 & = 41^2 \\
          x          & = 9 \\
        \end{align*}

        \begin{tabular}[H]{ccc}
          \toprule

          $\sin$           & $\cos$          & $\tan$          \\
          $\sfrac{40}{41}$ & $\sfrac{9}{41}$ & $\sfrac{40}{9}$ \\

          \midrule

          $\csc$           & $\sec$          & $\cot$ \\
          $\sfrac{41}{40}$ & $\sfrac{41}{9}$ & $\sfrac{9}{40}$ \\

          \bottomrule
        \end{tabular}

      \item[4] 
        Find the missing side:
        \begin{align*}
          15^2 + 8^2 & = r^2 \\
          r          & = 17 \\
        \end{align*}

        \begin{tabular}[H]{ccc}
          \toprule

          $\sin$           & $\cos$          & $\tan$          \\
          $\sfrac{15}{17}$ & $\sfrac{8}{17}$ & $\sfrac{15}{8}$ \\

          \midrule

          $\csc$           & $\sec$          & $\cot$ \\
          $\sfrac{17}{15}$ & $\sfrac{17}{8}$ & $\sfrac{8}{15}$ \\

          \bottomrule
        \end{tabular}

      \item[5] 
        Find the missing side:
        \begin{align*}
          3^2 + 2^2 & = r^2 \\
          r         & = \sqrt{13} \\
        \end{align*}

        \begin{tabular}[H]{ccc}
          \toprule
          $\sin$                 & $\cos$                 & $\tan$         \\
          $\sfrac{2}{\sqrt{13}}$ & $\sfrac{3}{\sqrt{13}}$ & $\sfrac{2}{3}$ \\

          \midrule

          $\csc$                 & $\sec$                 & $\cot$ \\
          $\sfrac{\sqrt{13}}{2}$ & $\sfrac{\sqrt{13}}{3}$ & $\sfrac{3}{2}$ \\

          \bottomrule
        \end{tabular}

      \item[6] 
        Find the missing side:
        \begin{align*}
          x^2 + 7^2 & = 8^2 \\
          x         & = \sqrt{15} \\
        \end{align*}

        \begin{tabular}[H]{ccc}
          \toprule

          $\sin$         & $\cos$                 & $\tan$                 \\
          $\sfrac{7}{8}$ & $\sfrac{\sqrt{15}}{8}$ & $\sfrac{7}{\sqrt{15}}$ \\

          \midrule

          $\csc$         & $\sec$                 & $\cot$ \\
          $\sfrac{8}{7}$ & $\sfrac{8}{\sqrt{15}}$ & $\sfrac{\sqrt{15}}{7}$ \\

          \bottomrule
        \end{tabular}

      \item[7]
        Find the missing side:
        \begin{align*}
          3^2 + 5^2 & = r^2 \\
          x         & = \sqrt{34} \\
        \end{align*}

        \begin{parts}
          \part $\sin \alpha = \cos \beta = \boxed{ \frac{3}{\sqrt{34}} }$
          \part $\tan \alpha = \cot \beta = \boxed{ \frac{3}{5} }$ \\
          \part $\sec \alpha = \csc \beta = \boxed{ \frac{\sqrt{34}}{5} }$ 
        \end{parts}

      \item[8]
        Find the missing side:
        \begin{align*}
          x^2 + 4^2 & = 7^2 \\
          x         & = \sqrt{33} \\
        \end{align*}

        \begin{parts}
           \item $\sin \alpha = \cos \beta = \boxed{ \frac{4}{7} }$
           \item $\tan \alpha = \cot \beta = \boxed{ \frac{4}{\sqrt{33}} }$
           \item $\sec \alpha = \csc \beta = \boxed{ \frac{7}{4} }$
        \end{parts}

      \item[9]
        \begin{align*}
          \sin 30 \dg & = \frac{x}{25} \\
          \frac{1}{2}     & = \frac{x}{25} \\
          x               & = \boxed{ \frac{25}{2} } \\
        \end{align*}

      \item[10]
        \begin{align*}
          \sin 45 \dg    & = \frac{12}{x} \\
          \frac{\sqrt{2}}{2} & = \frac{12}{x} \\
          x                  & = \boxed{ 12 \sqrt{2} } \\
        \end{align*}

      \item[11]
        \begin{align*}
          \sin 60 \dg    & = \frac{x}{13} \\
          \frac{\sqrt{3}}{2} & = \frac{x}{13} \\
          x                  & = \boxed{ \frac{13 \sqrt{3}}{2} } \\
        \end{align*}

      \item[12]
        \begin{align*}
          \tan 30 \dg    & = \frac{4}{x} \\
          \frac{\sqrt{3}}{3} & = \frac{4}{x}  \\
          x                  & = \boxed{ 4 \sqrt{3} } \\
        \end{align*}

      \item[15]
        \begin{align*}
          \sin \theta & = \frac{y}{28} \\
          y           & = \boxed{ 28 \sin \theta } \\
          \\
          \cos \theta & = \frac{x}{28} \\
          x           & = \boxed{ 28 \cos \theta } \\
        \end{align*}

      \item[16]
        \begin{align*}
          \tan \theta & = \frac{x}{4} \\
          x           & = \boxed{ 4 \tan \theta } \\
          \\
          \cos \theta & = \frac{4}{y} \\
          y           & = \boxed{ \frac{4}{\cos \theta} } \\
        \end{align*}

      \item[17] 
        Find the missing side:
        \begin{align*}
          x^2 + 3^2 & = 5^2 \\
          x         & = 4 \\
        \end{align*}

        \begin{tabular}[H]{ccc}
          \toprule

          $\sin$         & $\cos$         & $\tan$         \\
          $\sfrac{3}{5}$ & $\sfrac{4}{5}$ & $\sfrac{3}{4}$ \\

          \midrule

          $\csc$         & $\sec$         & $\cot$ \\
          $\sfrac{5}{3}$ & $\sfrac{5}{4}$ & $\sfrac{4}{3}$ \\

          \bottomrule
        \end{tabular}

      \item[18] 
        Find the missing side:
        \begin{align*}
          9^2 + y^2 & = 40^2 \\
          x         & = 7 \sqrt{31} \\
        \end{align*}

        \begin{tabular}[H]{ccc}
          \toprule
          $\sin$                    & $\cos$          & $\tan$                   \\
          $\sfrac{7 \sqrt{31}}{40}$ & $\sfrac{9}{40}$ & $\sfrac{7 \sqrt{31}}{9}$ \\

          \midrule

          $\csc$                    & $\sec$          & $\cot$ \\
          $\sfrac{40}{7 \sqrt{31}}$ & $\sfrac{40}{9}$ & $\sfrac{9}{7 \sqrt{31}}$ \\

          \bottomrule
        \end{tabular}

      \item[23]
        \begin{align*}
          \sin \frac{\pi}{6} + \cos \frac{\pi}{6} & = \frac{1}{2} + \frac{\sqrt{3}}{2} \\
                                                  & = \boxed{ \frac{1 + \sqrt{3}}{2} } \\
        \end{align*}

      \item[24]
        \begin{align*}
          \sin 30 \dg \cdot \csc 30 \dg & = \frac{1}{2} \cdot 2 \\
                                        & = \boxed{ 1 } \\
        \end{align*}

      \item[25]
        \begin{align*}
          \sin 30 \dg \cos 60 \dg + \sin 60 \dg \cos 30 \dg & =
            \frac{1}{2} \times \frac{1}{2} + \frac{\sqrt{3}}{2} \times \frac{\sqrt{3}}{2} \\
          & = \frac{1}{4} + \frac{3}{4} \\
          &= \boxed{ 1 } \\
        \end{align*}

      \item[26]
        \begin{align*}
          \sin^2 60 \dg + \cos^2 60 \dg & = \left( \frac{\sqrt{3}}{2} \right)^2 + \left( \frac{1}{2} \right)^2 \\
                                        & = \frac{3}{4} + \frac{1}{4} \\
                                        & = \boxed{ 1 } \\
        \end{align*}

      % \item[27]
      %   \begin{align*}
      %     ( \cos 30 \dg )^2 - ( \sin 30 \dg )^2 & = ( \frac{\sqrt{3}}{2} )^2 - ( \frac{1}{2}  )^2 \\
      %                                                   & = \frac{3}{4} + \frac{1}{4} \\
      %                                                   & = \boxed{ \frac{1}{2} } \\
      %   \end{align*}
  
      % \item[28]
      %   \begin{align*}
      %     \left( \sin \frac{\pi}{3} \cos \frac{\pi}{4} - \sin \frac{\pi}{4} \cos \frac{\pi}{3} \right)^2 & = 
      %     \left( \frac{\sqrt{3}}{2} \frac{\sqrt{2}}{2} - \frac{\sqrt{2}}{2} \frac{1}{2} \right)^2 \\
      %                % & = ( \frac{\sqrt{6}}{4} - \frac{\sqrt{2}}{4} \right)^2 \\
      %                % & = ( \frac{\sqrt{6} - \sqrt{2}}{4} \right)^2 \\
      %                % & = ( \frac{6 - 2 \sqrt{12} + 4}{16} \right)^2 \\
      %   \end{align*}
  
      \item[29]
        \begin{align*}
          \sin 45 \dg & = \frac{16}{r} \\
          r           & = \boxed{ 16 \sqrt{2} } \\
          \\
          \cos 45 \dg & = \frac{x}{16 \sqrt{2}} \\
          x           & = \boxed{ 16 } \\
        \end{align*}

      \item[30]
        \begin{align*}
          \sin 75 \dg & = \frac{100}{r} \\
          r           & \approx \boxed{ 103.53 } \\
          \\
          \tan 75 \dg & = \frac{100}{x} \\
          x           & \approx \boxed{ 26.795 } \\
        \end{align*}

      \item[31]
        \begin{align*}
          \cos 52 \dg & = \frac{35}{r} \\
          r           & \approx \boxed{ 56.849 } \\
          \\
          \tan 52 \dg & = \frac{y}{35} \\
          r           & \approx \boxed{ 44.798 } \\
        \end{align*}

      \item[32]
        \begin{align*}
          \sin 68 \dg & = \frac{y}{1000} \\
          y           & \approx \boxed{ 927.18 } \\
          \\
          \cos 68 \dg & = \frac{x}{1000} \\
          x           & \approx \boxed{ 374.61 } \\
        \end{align*}

      \item[39]
        Divide $x$ into a left and right half and then add them up to get the total.

        \begin{align*}
          \tan 60 \dg & = \frac{100}{x_l} \\
          \sqrt{3}    & = \frac{100}{x_l} \\
          x_l         & = \frac{100 \sqrt{3}}{3} \\
          \\
          \tan 30 \dg        & = \frac{100}{x_r} \\
          \frac{1}{\sqrt{3}} & = \frac{100}{x_r} \\
          x_r                & = 100 \sqrt{3} \\
          \\
          x & = \frac{100 \sqrt{3}}{3} + 100 \sqrt{3} \\
            & = \frac{400 \sqrt{3}}{3} \\
            & \approx \boxed{ 230.9 }
        \end{align*}

      \item[40]
        \begin{align*}
          \tan 30 \dg & = \frac{100}{x_1} \\
          x_1         & = 85 \sqrt{3} \\
          \\
          \tan 60 \dg & = \frac{85}{x_2} \\
          x_2         & = \frac{85 \sqrt{3}}{3} \\
          \\
          x  & = x_1 - x_2 \\
             & = 85 \sqrt{3} - \frac{85 \sqrt{3}}{3} \\
             & = \frac{170 \sqrt{3}}{3} \\
             & \approx \boxed{ 98.1 } \\
        \end{align*}

      \item[41]
        \begin{align*}
          \sin 60 \dg & = \frac{50}{r} \\
          r           & = \frac{100}{\sqrt{3}} \\
          \\
          \sin 65 \dg & = \frac{ \sfrac{100}{\sqrt{3}} }{x} \\
          x          & \approx \boxed{ 63.7 } \\
        \end{align*}

      \item[42]
        \begin{align*}
          \sin 30 \dg & = \frac{5}{y} \\
          y           & = 10 \\
          \\
          \tan 30 \dg & = \frac{x}{10} \\
          x          & \approx \boxed{ 5.8 } \\
        \end{align*}

      \item[45]
        Using feet for the height:
        \begin{align*}
          \tan 11 \dg & = \frac{x}{5280} \\
          x           & \approx \boxed{ \unit[1026]{ft} } \\
        \end{align*}

      \item[46]
        \begin{enumerate}[(a)]
          \item 
            \begin{align*}
              \sin 22 \dg & = \frac{35,000}{d} \\
              d           & \approx \unit[93,431]{ft} \\
                          & \approx \boxed{ \unit[17.7]{mi} } \\
            \end{align*}

          \item 
            \begin{align*}
              \tan 22 \dg & = \frac{35,000}{d} \\
              d           & \approx \unit[86,628]{ft} \\
                          & \approx \boxed{ \unit[16.4]{mi} } \\
            \end{align*}

        \end{enumerate}

      \pagebreak

      \item[47]
        \begin{enumerate}[(a)]
          \item 
            \begin{align*}
              \tan 0.5 \dg & = \frac{d}{240,000} \\
              d           & \approx \boxed{ \unit[2094]{mi} } \\
            \end{align*}

          \item 
            no---it misses by about 1000 miles
        \end{enumerate}

      \item[48]
        \begin{align*}
          \tan 23 \dg & = \frac{200}{d} \\
          d           & \approx \boxed{ \unit[471]{ft} } \\
        \end{align*}

      \item[52]
        \begin{align*}
          \sin 65 \dg & = \frac{h}{600} \\
          d           & \approx \boxed{ \unit[544]{ft} } \\
        \end{align*}

      \item[54]
        \begin{align*}
          \tan 14 \dg & = \frac{y_1}{x} \\
          y_1         & \approx \unit[0.2493 x]{ft} \\
          \\
          \tan 18 \dg & = \frac{y_2}{x} \\
          y_2         & \approx \unit[0.3249 x]{ft} \\
          \\
          y_1 + y_2           & = 60 \\
          0.2493 x + 0.3249 x & = 60 \\
          x                   & \approx \boxed{ \unit[104]{ft} } \\
        \end{align*}

      \item[56]
        \begin{align*}
          \cos 35 \dg & = \frac{d_1}{5150} \\
          d_1         & \approx \unit[4219]{ft} \\
          \\
          \cos 52 \dg & = \frac{d_2}{5150} \\
          d_2         & \approx \unit[3171]{ft} \\
          \\
          d & \approx 4219 + 3171 \\
            & = \boxed{ \unit[7389]{ft} } \\
        \end{align*}

      \item[57]
        The distances are the same but you need to subtract them now:
        \begin{align*}
          d & \approx 4219 - 3171 \\
            & = \boxed{ \unit[1048]{ft} } \\
        \end{align*}

      \item[59]
        \begin{align*}
            \tan 32 \dg & = \frac{h}{d} \\
            d           & \approx 1.6003 h \\
            \\
            \tan 35 \dg & = \frac{h}{d - 1000} \\
            \tan 35 \dg & = \frac{h}{1.6h - 1000} \\
            h           & \approx \boxed{ \unit[5809]{ft} } \\
        \end{align*}

      \item[61]
        \begin{align*}
            \sin 0.15 \dg & = \frac{240,000}{d} \\
            d           & \approx \boxed{ \unit[ 9.17 \times 10^7 ]{mi} }  \\
        \end{align*}

      \item[62]
        \begin{enumerate}[(a)]
          \item The circumference of the earth is $2 \pi \times 3960 = \unit[24,881]{mi}$.

            The angle is:
            \[
              \frac{6155}{24,881} \cdot 360 \dg = \boxed{ 89.05 \dg }
            \]

          \item Find the distance from the center of the earth to the moon:
            \begin{align*}
              \cos 89.05 \dg & = \frac{3960}{d + 3960} \\
              d_{center}     & \approx \unit[ 234,884 ]{mi} \\
            \end{align*}

            Subtract the radius of the earth:
            \[
              d_{surface} \approx 234,884 - 3960 = \boxed{ \unit[230,934]{mi} } 
            \]

        \end{enumerate}

        \item[63]
          \begin{align*}
            \sin 60.276 \dg & = \frac{r}{r + 600} \\
            r               & \approx \boxed{ \unit[3960]{mi} } \\
          \end{align*}

        \item[64]
          \begin{align*}
            \sin 0.000211 \dg & = \frac{9.3 \times 10^7}{d} \\
            d               & \approx \boxed{ \unit[2.52 \times 10^{13}]{mi} } \\
          \end{align*}

        \item[65]
          \begin{align*}
            \sin 46.3 \dg & = \frac{d}{1} \\
            d             & \approx \boxed{ \unit[0.723]{Au} } \\
          \end{align*}

    \end{description}

  \else
    \vspace{7 cm}
    \begin{quote}
      \begin{em}
        Anarchism is founded on the observation that since few men are wise enough to rule themselves, even fewer are
        wise enough to rule others.  
      \end{em}
    \end{quote}
    \hspace{1 cm} --Edward Abbey
  \fi

\end{document}

