\documentclass[fleqn, onecolumn]{article}

\usepackage{fullpage}
\usepackage{graphicx}
\usepackage{float}
\usepackage{amsmath}
\usepackage{amssymb}
\usepackage{polynom}
\usepackage{caption}
\usepackage{mdwlist}
\usepackage{parskip}

\newcommand{\degree}{\ensuremath{^\circ}} 

\everymath{\displaystyle}
\setlength{\mathindent}{1 cm}

\author{Ed Tellman}
\title{Math 141---Precalculus}
\date{January 9, 2013}

\begin{document}

\maketitle

\section{Overview}
This course will cover chapters 2-4 of {\em Precalculus}, by Stewart, Redlin, and Watson.  You should be familiar with
the material in chapter 1 from your algebra class.

\section{Homework and Exams}

You should expect to spend three or four hours each week doing homework.  Feel free to work together with other students
on the homework.

Each chapter will be followed by an in-class test.  

\section{Credit}
As far as I know, we're still working out the arrangements with Edmonds Community College so that people who are
interested can get official college credit for this class.  If all goes well, you'll be able to take a final exam from
either Edmonds or Ohio University to to get credit.  

Last semester we were a bit short of funds, so people who wanted to get credit had to pay for the tuition themselves.  I
think the we are probably in the same situation this semester.

\section{Topics}

\subsection{Chapter 2---Functions}
A function is a rule for turning one number into a different one.  Examples of functions that you are already 
familiar with are ``square root,'' ``absolute value,'' and ``square.''.  We'll talk about functions and how 
to graph them, transform them, and combine them in various ways.

\subsection{Chapter 3---Polynomial and Rational Functions}
Polynomial functions are functions which only involve powers of the variable.  We'll talk about how to graph polynomial
functions and how to find where their graphs cross the x-axis .  We'll also talk about what happens when you divide one
polynomial by another one and how to graph the resulting function.

\subsection{Chapter 4---Exponential and Logarithmic Functions}

Exponential functions are functions where the variable appears in the exponent instead of the base.  This type of
function comes up whenever the rate of change of something depends on the amount of it you have already.  For example,
the more money you have already, the more money you will make in interest, so interest is computed using an exponential
function.

Logarithms are the inverse of exponentials.  They were invented about 500 years ago by a very interesting
farmer/mathematician named John Napier.  For the next 450 years they were indispensable for any complicated
calculations.  They are used today to describe things which have an extremely wide difference between large and small,
like the volume and pitch of sounds.

\end{document}

