\documentclass{exam}

\usepackage{fullpage}
\usepackage{enumerate}
\usepackage{siunitx} 
\usepackage{graphicx}
\usepackage[fleqn]{amsmath}
\usepackage{cancel}
\usepackage{polynom}
\usepackage{float}
\usepackage{mdwlist}
\usepackage{booktabs}
\usepackage{cancel}
\usepackage{polynom}
\usepackage{caption}

\newcommand{\degree}{\ensuremath{^\circ}} 
\everymath{\displaystyle}

\title{Math 113 Homework Three}
\author{}
\date{\today}

\printanswers

\ifprintanswers
\usepackage{2in1, lscape}
\fi

\begin{document}

\maketitle

\section{From the Book}

\ifprintanswers

\subsection{Pages 50-51}

\begin{description}

\item[57]
The smaller number is {\em x} and the larger number is {\em 6x -3}.
\begin{eqnarray*}
  6x -3 -x & = & 67 \\
  5x - 3 & = &  67 \\
  5x & = & 70 \\
  x & = & 14 \\
\end{eqnarray*}

So the smaller number is 14 and the larger number is \( 6 \cdot 14 - 3 = 81 \).

check: \( 81 - 14 = 67 \)

\item[58]
The smaller number is {\em x} and the larger number is {\em 5x + 1}.
\begin{eqnarray*}
  x + 5x + 1 & = & 103 \\
  6x + 1 & = & 103 \\
  6x & = & 102 \\
  x & = & 17
\end{eqnarray*}

So the smaller number is 17 and the larger number is \( 5 \cdot 17 + 1 = 86 \).

check: \( 17 + 86 = 103 \)

\item[59]
\begin{eqnarray*}
  40x + 6 \cdot 2x & = & 572 \\
  40x + 12x & = & 572 \\
  52x & = & 572 \\
  x & = & 11 \\
\end{eqnarray*}
check: \(40 \cdot 11 + 6 \cdot 22 = 572 \)

\item[63]
\begin{eqnarray*}
  3x - 150 & = & 750 \\
  3x & = & 900 \\
  x & = & 300 \\
\end{eqnarray*}
check: \(3 \cdot 300 - 150 = 750\)

\subsection{Pages 58-59}

\item[30]
\begin{eqnarray*}
  \frac{2x + 7}{8} + x - 2 & = & \frac{x-1}{2} \\
  8 \left( \frac{2x + 7}{8} + x - 2 \right) & = & 8 \left( \frac{x-1}{2} \right) \\
  2x + 7 + 8x - 16 & = & 4x - 4 \\
  6x - 9 & = & -4 \\
  x & = & 5/6 \\
\end{eqnarray*}
check: \( \frac{2(5/6) + 7}{8} + 5/6 - 2 = \frac{5/6 - 1}{2} \)

\item[31]
\begin{eqnarray*}
  \frac{x + 3}{2} + \frac{x + 4}{5} = \frac{3}{10} \\
  10 \left( \frac{x + 3}{2} + \frac{x + 4}{5} \right) = 10 \left( \frac{3}{10} \right) \\
  5x + 15 + 2x + 8 = 3 \\
  7x + 23 = 3 \\
  x = -\frac{20}{7} \\
\end{eqnarray*}

check: \( \frac{-20/7 + 3}{2} + \frac{-20/7 + 4}{5} = \frac{3}{10} \\\)

\item[32]
\begin{eqnarray*}
  \frac{x - 2}{5} - \frac{x - 3}{4} & = & -\frac{1}{20} \\
  20(\frac{x - 2}{5} - \frac{x - 3}{4}) & = & 20(-\frac{1}{20}) \\
  4x - 8 - (5x - 15) & = & -1 \\
  -x + 7 & = & -1 \\
  x & = & 8 \\
\end{eqnarray*}
check: \( \frac{8 - 2}{5} + \frac{8 - 3}{4} =  -\frac{1}{20} \)


\item[33]
\begin{eqnarray*}
  n + \frac{2n - 3}{9} - 2 = \frac{2n + 1}{3} \\
  9n + 2n - 3 - 18 = 6n + 3 \\
  n = 24/5 \\
\end{eqnarray*}

check: \( \frac{24}{5} + \frac{2 \left( \frac{24}{5} \right) - 3}{9} - 2 = \frac{2 \left(\frac{24}{5} \right) + 1}{3} \)

\item[34]
\begin{eqnarray*}
  n - \frac{3n + 1}{6} - 1 & = & \frac{2n + 4}{12} \\
  n - \frac{3n + 1}{6} - 1 & = & \frac{2(n + 2)}{2 \cdot 6} \\
  n - \frac{3n + 1}{6} - 1 & = & \frac{n + 2}{6} \\
  6(n - \frac{3n + 1}{6} - 1) & = & 6 \left(\frac{n + 2}{6} \right) \\
  6n - 3n - 1 - 6 & = & n + 2 \\
  3n - 7 & = & n + 2 \\
  n & = & \frac{9}{2} \\
\end{eqnarray*}
check: \( \frac{9}{2} + \frac{3 \left( \cfrac{9}{2} \right) - 1}{6} - 1 = \frac{2 \left( \cfrac{9}{2} \right) + 4}{12}\)

\item[35]
\begin{align*}
  \frac{3}{4}(t - 2) - \frac{2}{5}(2t + 3) &= \frac{1}{5} \\
  20 \left( \frac{3}{4}(t - 2) - \frac{2}{5}(2t + 3) \right) &= 20 \left( \frac{1}{5} \right) \\
  15(t - 2) - 8(2t + 3) &= 4 \\
  15t - 30 - 16t + 24 &= 4 \\
  -t - 6 &= 4 \\
  t &= -10 \\
\end{align*}

check: \( \frac{3}{4}(-10 - 2) - \frac{2}{5}(2(-10) + 3) = \frac{1}{5} \)

\item[37]
\begin{eqnarray*}
  \frac{1}{2}(2x - 1) - \frac{1}{3}(5x + 2) & = & 3 \\
  6 \left( \frac{1}{2}(2x - 1) - \frac{1}{3}(5x + 2) \right) & = & 6 \cdot 3 \\
  3(2x - 1) - (5x + 2) & = & 18 \\
  -4x - 7 & = & 18 \\
   x & = & -25/4 \\
\end{eqnarray*}
check: \( \frac{1}{2}(2(-\frac{25}{4}) - 1) - \frac{1}{3} (5 \left(-\frac{25}{4} \right) + 2) = 3 \)

\item[40]
\begin{eqnarray*}
  2x + 5 + \frac{1}{2}(6x - 1) & = & -\frac{1}{2} \\
  2x + 5 + 3x - \frac{1}{2} & = & -\frac{1}{2} \\
  5x + 5 & = & 0 \\
  x & = & -1 \\
\end{eqnarray*}
check: \( 2(-1) + 5 + \frac{1}{2}(6(-1) - 1) = -\frac{1}{2} \)

\item[42]
\begin{eqnarray*}
  x/2 + 3x/4 & = & 4x/3 + 2 \\
  12(x/2 + 3x/4) & = & 12(4x/3 + 2) \\
  6x + 9x & = & 16x + 24 \\
  15x & = & 16x + 24 \\
  x & = & -24 \\
\end{eqnarray*}
check: \( (-24)/2 + 3(-24)/4 = 4(-24)/3 + 2 \)

\item[45]
\begin{eqnarray*}
  x + 1/3(x + 1) + 3/8(x + 2) & = & 25 \\
  24(x + 1/3(x + 1) + 3/8(x + 2)) & = & 24 \cdot 25 \\
  24x + 8(x + 1) + 12(x + 2) & = & 600 \\
  41x + 26 & = & 600 \\
  x & = & 14 \\
\end{eqnarray*}
check: \( 14 + 1/3(14 + 1) + 3/8(14 + 2) = 25 \)

\item[49] 
\begin{tabular}{|c|c|}
\hline
  $x$ & Angie's age \\
  $x + 8$ & Angie's age in eight years \\
  $3/5(64 - x + 8)$ & Angie's age in eight years \\
  $64 - x$ & Mother's age in eight years \\
\hline
\end{tabular}

\begin{eqnarray*}
  3/5(64 - x + 8) & = & x + 8 \\
  3/5(72 - x) & = & x + 8 \\
  5(3/5(72 - x)) & = & 5(x + 8) \\
  3(72 - x) & = & 5x + 40 \\
  216 - 3x & = & 5x + 40 \\
  8x & = & 176 \\
  x & = & 22 \\
\end{eqnarray*}
Angie is 22 and her mother is 42.  In 8 years, Angie will be 30 and her mother will be 50.  \(3/5(50) = 30\) so the answer is correct.

\subsection{Pages 66-67}

\item[15]
\begin{eqnarray*}
  0.12t - 2.1 & = & 0.07t - 0.2 \\
  100(0.12t - 2.1) & = & 100(0.07t - 0.2) \\
  12t - 210 & = & 7t - 20 \\
  5t & = & 190 \\
  t & = & 38 \\
\end{eqnarray*}
check: \( 0.12(38) - 2.1 = 0.07(38) - 0.2\)

\item[16]
\begin{eqnarray*}
  0.13t - 3.4 & = & 0.08t - 0.4 \\
  100(0.13t - 3.4) & = & 100(0.08t - 0.4) \\
  13t - 340 & = & 8t - 40 \\
  5t &=& 300 \\
  t &=& 60 \\
\end{eqnarray*}
check: \( 0.13(60) - 3.4  = 0.08(60) - 0.4 \)

\item[17]
\begin{eqnarray*}
  0.92 + 0.9(x - 0.3) &=& 2x - 5.95 \\
  100(0.92 + 0.9(x - 0.3)) &=& 100(2x - 5.95) \\
  92 + 90(x - 0.3) &=& 200x - 595 \\
  92 + 90x - 27 &=& 200x - 595 \\
  -110x &=& -660 \\
  x &=& 6
\end{eqnarray*}
check: \( 0.92 + 0.9(6 - 0.3) = 2(6) - 5.95 \)

\item[30]
\begin{eqnarray*}
  24 &=& x - 0.25x \\
  24 &=& .75x \\
  x &=& 32 \\
\end{eqnarray*}
check: \( \)

\item[31] \(x = 64 - .15(64) = 54.4 \)

\item[43] Let $x$ be the amount invested at 10\%.
\begin{eqnarray*}
  .1x + .11(x + 1,500) &=& 795 \\
  100(.1x + .11(x + 1,500)) &=& 100 \cdot 795 \\
  10x + 11(x + 1,500) &=& 79,500 \\
  21x + 16,500 &=& 79,500 \\
  21x &=& 63,000 \\
  x &=& 3,000 \\
\end{eqnarray*}
check: \( .1(3,000) + .11(3,000 + 1,500) = 795\)

\subsection{Pages 77-79}

\item[1] Using \(i = Prt\), \(300)(.08)(5) = 120 \).

\item[5] Using \(r = \frac{i}{Pt}\), \(\frac{90}{600 \cdot 2.5} = 0.06 \).

\item[24]
\begin{eqnarray*}
  A &=& \frac{h}{2} (b_1 + b_2) \\
  2A &=& h(b_1 + b_2) \\
  h &=& \frac{2A}{b_1 + b_2} \\
\end{eqnarray*}
check:
\[
 \frac{1}{2} \left( \frac{2A}{b_1 + b_2} \right) (b_1 + b_2) = 2A 
\]

\item[28]
\begin{eqnarray*}
  x/a + y/b &=& 1 \\
  a(x/a + y/b) &=& a \cdot 1  \\
  x + ya/b &=& a  \\
  x &=& a - ya/b \\
\end{eqnarray*}
check: \( (a - ya/b)/a + y/b = 1 - y/b + y/b = 1 \)

An alternate form of the solution is: \( \displaystyle \frac{ab -ay}{b} \).

\item[32]
\begin{eqnarray*}
  x(a-b) &=& m(x-c) \\
  xa-xb &=& mx-mc \\
  xa-xb-mx &=& -mc \\
  x(a-b-m) &=& -mc \\
  x &=& \frac{-mc}{a-b-m} \\
\end{eqnarray*}
check: \( \frac{-mc}{a-b-m} (a-b) = m \left( \frac{-mc}{a-b-m} - c \right) \)

An alternate form of the solution is: \( \displaystyle \frac{mc}{m - a + b} \).

\item[35]
\begin{eqnarray*}
  \frac{1}{3}x + a &=& \frac{1}{2}b \\
  \frac{1}{3}x &=& \frac{1}{2}b - a \\
  3(\frac{1}{3}x) &=& 2(\frac{1}{2}b - a) \\
  x &=& \frac{3}{2}b - 3a \\
\end{eqnarray*}
check: \( \frac{1}{3} \left( \frac{3}{2}b - 3a \right) + a = \frac{1}{2}b - a + a = \frac{1}{2}b \)

An alternate form of the solution is: \( \frac{3b - 6a}{2} \).

\item[36]
\begin{align*}
  \frac{2}{3}x - \frac{1}{4}a &= b \\
  \frac{2}{3}x  &= b + \frac{1}{4}a \\
  3 \left( \frac{2}{3} x \right)  &= 3 \left( b + \frac{1}{4} a \right) \\
  2x  &= 3b + \frac{3}{4}a \\
  x  &= \frac{3}{2}b + \frac{3}{8}a \\
\end{align*}

check: \( \frac{2}{3} \left( \frac{3}{2}b + \frac{3}{8}a \right) - \frac{1}{4}a = b + \frac{1}{4}a - \frac{1}{4}a = b \)

An alternate form of the solution is: \( \frac{12b + 3a}{8} \).

\item[52] We can use \( i = Prt \) with $r = 0.1$ and \(i = 2P\).  We need three times as much money, and we already
  have the principal so we need the interest to supply twice the principal.  I think you could also interpret this
  problem as meaning you want the interest to be three times the principal.

\begin{eqnarray*}
  t = \frac{2P}{P \cdot 0.1} \\
  t = \frac{2}{0.1} \\
  t = 20 \\
\end{eqnarray*}
check: \( i = P(0.1)(20) = 2P \)

\item[54] The two cyclists are approaching each other at the sum of their two speeds, or \(18 + 14 = 32\) miles per
  hour.  The distance is 112 miles.  Using \(t = d/r\), \(t = 112/32 = 3.5\) hours.

  To check, in 3.5 hours the faster cyclist travels 63 miles and the slower cyclist travels 49 miles.  \(63 + 49 =
  112\), so the answer is correct.

\item[57] Let $x$ be the time at 20 mph and $4.5 - x$ (the rest of the time) be the time he spent at 12 mph.  Since he
  traveled 70 miles altogether:

\begin{eqnarray*}
  20x + 12(4.5 - x) = 70 \\
  20x + 54 - 12x = 70 \\
  8x + 54 = 70 \\
  x = 2 \\
\end{eqnarray*}

So he spent 2 hours at 20 mph and 2.5 hours at 12 mph.  

The question asks how far he traveled at each speed, not how
long he spent at each speed.  So we need to plug these numbers into \(d = rt\), which gives 40 miles at 20 mph and 30
miles at 12 mph.  \(30 + 40 = 70\), so the answer seems correct.

\end{description}

\else
\begin{itemize*}
\item pp. 58-59: 30-35, 37, 40, 42, 45, 49
\item pp. 66-67: 15-17, 30, 31, 43
\item pp. 77-79: 1, 5, 24, 28, 32, 35, 36, 52, 54, 57
\end{itemize*}

\fi

\section{Additional Problems}

\begin{questions}

\question
It would be a shame to let Super Bowl week go by without a prediction.  Here's my prediction.

At half time the Colts will have scored as many field goals as the Saints have scored touchdowns.  In the second half,
the Colts will score three touchdowns while holding the Saints to 9 points.  The Colts will win the game by one point by
making a two point conversion after the final touchdown as time expires.  Of course, since this is the Super Bowl, nobody
misses extra points or gets safeties, and the only two point conversion will be the one I mentioned.

If my prediction comes true, what will the final score be?

\begin{solution}
Let {\em x} be the number of Colt field goals in the first half.

The total Colt score for the game will be: \(3x + 22\).

The total Saint score for the game will be: \(7x + 9\).

The Colts win the game by a point, so:
\begin{eqnarray*}
  7x + 9 + 1 & = & 3x + 22 \\
  4x + 10 & = & 22 \\
  4x & = & 12 \\
  x  & = & 3 \\
\end{eqnarray*}
So the scoreboard might look something like:

\begin{tabular}{|c|c|c|c|c|c|}
\hline
           & 1  & 2 & 3 & 4 & Total \\
\hline
  Saints   & 14 & 7 & 6 & 3 & 30 \\
\hline
  Colts    & 0  & 9 & 14 & 8 & 31 \\
\hline
\end{tabular}

\end{solution}

\question

A judge in Pennsylvania was having trouble paying his bills with his meager judicial salary.  So when a
prison moved into town, he saw a golden business opportunity.  He worked out a deal with the prison where,
for a small fee, he would sentence juveniles to lengthy prison terms.  If a particular offense was usually
punished with a fine, he would instead issue the maximum punishment so the defendant would enjoy some time in
the new prison.

It was a terrific arrangement for everyone.  Even with the rising prices of yachts, the judge was able to make ends
meet.  The prison had plenty of residents.  And since they charged the state for each prisoner, the prison owners were
thrilled. The modest investment in bribes paid off handsomely.

\begin{parts}
  \part
The judge wanted to make \$1,300,000 from the scheme.  He decided a reasonable plan would be to charge an
initial down payment for his participation, and then a fee for each sentence handed out.  If the initial down
payment was \$200,000, and he charged a bribe of \$2,000 per case, how many cases would he need to process to 
reach his goal?

\begin{solution}
Letting $x$ be the number of cases:
\begin{eqnarray*}
  200,000 + 2,000x &=& 1,300,000 \\
  200 + 2x &=& 1,300 \\
  2x &=& 1,100 \\
  x &=& 550 \\
\end{eqnarray*}
To check, if he got bribes for 550 cases, and a \$200,000 down payment, he would make 
\(200,000 + 550 \cdot 2,000 = 1,300,000\). 

\end{solution}

\part
Judging is hard work, even when you've already decided the outcome in advance.  So the judge decided to retire a bit
early.  He still wanted to make \$1,300,000 from the scheme, but he wanted to retire after only taking in \$1,000,000,
investing his money to make up the difference.  

He found a bank that paid 5\% simple interest (\(i = Prt\)) per year. Assuming he can manage to survive without
spending any of the principal or interest, how long will he have to wait for his \$1,000,000 to turn into \$1,300,000?

\begin{solution}

Solving the equation for $t$ gives: \( t = \frac{i}{Pr} \).  He needs to make \$300,000 in interest to get to
\$1,300,000.  So we can plug in the numbers for $i$, $P$, and $r$, to get:

  \begin{eqnarray*}
    t & = & \frac{300,000}{1,000,000 \cdot .05}   \\
    t & = & \frac{300,000}{50,000}   \\
    t & = & 6 \\
  \end{eqnarray*}

  So after six years of retirement, he will have the full \$1,300,000.

\end{solution}

\end{parts}

\question 

The equation for a Celsius temperature in terms of a Farenheit temperature is:
\[C = \frac{5}{9} ( F - 32 ) \]

\begin{parts}
  \part
  What is the equation for a Farenheit temperature in terms of a Celsius temperature?

\begin{solution}
\begin{eqnarray*}
  C & = & \frac{5}{9} (F - 32) \\
  \frac{9}{5} C & = & F - 32 \\
  F & = & \frac{9}{5} C + 32 \\
\end{eqnarray*}
\end{solution}

  \part
  There is one temperature where the temperature in Celsius is the same as the temperature in Farenheit.  In other
  words: {\em x} degrees Farenheit = {\em x} degrees Celsius.

  What is this temperature?

\begin{solution}

You can use either form of the equation.

\begin{eqnarray*}
  x & = & \frac{5}{9} (x - 32) \\
  \frac{9}{5} x & = & x - 32 \\
  \frac{4}{5} x & = & -32 \\
  x & = & -40 \\
\end{eqnarray*}
\end{solution}

\end{parts}

\question

Two police cars depart from the same point, traveling in opposite directions towards different donut shops. One car
travels 6 mph faster than the other car.  Find the speed of each car if they are 176 miles apart at the end of 55
minutes.

\begin{solution}
Let {\em x} be the speed of the slower car.

\begin{eqnarray*}
  \frac{55}{60} x + \frac{55}{60}(x + 6) = 176 \\
  \frac{11}{12} (x + x + 6) = 176 \\
  2x + 6 = \frac{12}{11} \cdot 176 \\
  2x + 6 = 192 \\
  2x = 186 \\
  x = 93 \\
\end{eqnarray*}

So the slower car is traveling at 93 mph and the faster car is traveling at 99 mph.

\end{solution}

\question

Last week I was chatting with the new UBB English professor, Gillian Harkin.  She asked me if
we'd heard the poem by Robert Browning which begins: {\em ``A rose-red city, half as old as time...''}

``Wait!'' I interrupted,  ``That would make a nifty algebra problem.''  I thought about it for a minute or two and came up with
this poem:

\begin{verbatim}
A rose-red city, half as old as Time.
One billion years ago, the city's age
Was just two-fifths of what Time's age will be
A billion years from now.  Can you compute
How old the crimson city is today?
\end{verbatim}

She gave me a funny look and quickly changed the subject.

How old is the rose-red city?

\begin{solution}

Here's what we know about the ages:

\begin{tabular}{|c|c|}
\hline
  Time & T \\
  City & \( T/2 \) \\
  Time 1 billion years from now  & T + 1 \\
  City 1 billion years ago & \( T/2 - 1\) \\
  City 1 billion years ago & \(2/5 (T + 1) \) \\
\hline
\end{tabular}

\vspace{.2 cm}

Putting it all together:

\begin{eqnarray*}
  \frac{T}{2} - 1 & = & \frac{2}{5} (T + 1) \\
  10 \left( \frac{T}{2} - 1 \right) & = & 10 \cdot \frac{2}{5} (T + 1)  \\
  5T - 10 & = & 4(T + 1) \\
  5T - 10 & = & 4T + 4 \\
  T - 10 & = & 4 \\
  T & = & 14 \\
\end{eqnarray*}

So time is 14 billion years old.  The city is half as old as time, so it is 7 billion years old.

Astronomers estimate the universe actually is about 13.7 billion years old.  But the earth seems to only be about 4.5 billion years
old, so the age of the city may be slightly exaggerated.

\end{solution}

\section{Extra Credit}

\question

A commuter is in the habit of arriving at his suburban station each evening exactly at 5:00.  His wife always meets the
train and drives him home.  One day he takes an earlier train, arriving at the station at 4:00.  The weather is
pleasant, so instead of telephoning home he starts walking along the route always taken by his wife.  They meet
somewhere along the way.  He gets into the car and they drive home, arriving at their house ten minutes earlier than
usual.  

Assuming that the wife always drives at a constant speed, and that on this occasion she left just in time to
meet the 5:00 train, how long did the husband walk before he was picked up?

\begin{solution}
  The wife took 10 minutes off her usual trip.  So she must have picked her husband up 5 minutes early, saving 5 minutes
  on the way to the station and 5 minutes on the way back.  She planned to pick him up at 5:00, so she must have instead
  picked him up at 4:55.  He started walking at 4:00, so he walked for 55 minutes.
\end{solution}

\end{questions}


\end{document}


