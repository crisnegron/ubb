\documentclass[fleqn,addpoints]{exam}
\usepackage{amsmath}
\usepackage{graphicx}

\title{Math 113 Placement Exam}
\author{}
\date{January 13, 2010}

% \oddsidemargin 0in
% \topmargin -0.5in
% \textwidth 6.5in

\extrawidth{-1 in}
\setlength{\mathindent}{0in}

\printanswers

\begin{document}

\maketitle

\vspace{0.2in}
\makebox[\textwidth]{Name:\enspace\hrulefill}
\vspace{0.2in}

\begin{center}
\gradetable[h][pages]
\end{center}

\section{Introduction}

This exam is intended to help you determine if you are enrolled in the appropriate math course.  You should be familiar
with most of the material on this exam before taking Math 113.  

According to the Ohio University {\em Teacher's Guide}, if you score:

\begin{itemize}
  \item 60\%-100\%--you should be OK
  \item 40\% - 59\%--you may be OK if you are can put in some additional effort at the beginning to catch up to the other
    students. 
  \item 0\% - 40\%--you should consider taking Math 101 instead.
\end{itemize}

No calculators.

If you need more space or scratch paper, ask Carol.

\ifprintanswers

\vspace{0.5 in}

  \begin{em}
    Math is hard enough, so you should do everything you can to make it easier.  Here are some things you can often do:
    \begin{itemize}
      \item Reorder the equations a bit before doing any calculations.  When you do this, you may find that
            some terms cancel out or you do less work finding common denominators, and you end up doing less work in
            the end. 

      \item Do as much simplification as you can before plugging in actual values for the variables.  For example,
        suppose the problem is to evaluate this equation for \( n = 5 \):
        
        \[ 17n^2 - 3n - 16n^2 \]

        If you just work from left to right, you have to figure out what \( 17 \cdot 25 \) is, then subtract \( 15 \),
        then figure out what \( 16 \cdot 25 \) is and subtract that.  If you do all this correctly, you will end up with
        the right answer.  But there is a lot of opportunity to make an arithmetic mistake.  So a better approach is to
        turn the original equation into:

        \[ 17n^2 - 16n^2 - 3n = n^2 - 3n \]

        Then you just have to figure out what \( 25 - 15 \) is, which you can easily do in your head.
    \end{itemize}

    Reordering equations is good.  But don't try to take to many shortcuts in your head, as you are more likely to make
    a mistake.  Paper is pretty cheap, so don't try to save it by doing several steps at once in your head.  When you do
    this, you are likely to forget a term, write a term twice, etc.

    An exponent goes with whatever it is next to and nothing else.  If the exponent is next to a parenthisized
    expression, it applies to whole expression:

    \begin{eqnarray*}
      -2^2 & = & - (2 \cdot 2) = -4 \\
      (-2)^2 & = & (-2) \cdot (-2) = 4 \\
      2x^2 & = & 2 \cdot x \cdot x \\
      (2x)^2 & = & (2 \cdot x) \cdot (2 \cdot x) = 4x^2 \\
    \end{eqnarray*}

    You should always show your work.  If you show your work and make a minor arithmetic error, you usually get most of
    the credit.  If you don't show your work and just write an incorrect answer, you won't get any credit, since I won't
    have any idea how you arrived at that answer.  If you need more room and use a separate piece of paper, write your
    name on that paper and make it possible to identify which work goes with which problem.

  \end{em}

\fi

\section{Questions}

\subsection{Numerical Expressions}

For problems \ref{ne:first}-\ref{ne:last}, simplify or evaluate each numerical expression.

\begin{questions}

\question[5]
\label{ne:first}
\( -7 + 4 - (-5) + 7 + 10 \)

\begin{solution}[3 cm]
  \begin{eqnarray*}
    -7 + 4 - (-5) + 7 + 10 & = & 7 - 7 + 4 + 5 + 10 \\
                           & = & 4 + 5 + 10 \\
                           & = & 19 \\
  \end{eqnarray*}
\end{solution}

\question[5]
\( \displaystyle
  \frac{1}{3} + \frac{1}{2} 
\)

\begin{solution}[3 cm]
  \begin{eqnarray*}
    \frac{1}{3} + \frac{1}{2} & = & \frac{2}{6} + \frac{3}{6} \\
                              & = & \frac{5}{6} \\
  \end{eqnarray*}
\end{solution}

\question[5]
\( \displaystyle
  5 (-\frac{1}{3}) - 3(-\frac{1}{2}) - 2(\frac{2}{3}) + 1 
\)
\begin{solution}[3 cm]
  \begin{eqnarray*}
    5 (-\frac{1}{3}) - 3(-\frac{1}{2}) - 2(\frac{2}{3}) + 1 & = & -\frac{5}{3} - \frac{4}{3} + \frac{3}{2} + 1 \\
    & = & \frac{3}{2} - \frac{9}{3} + 1 \\
    & = & \frac{3}{2} - 3 + 1 \\
    & = & \frac{3}{2} - 2 \\
    & = & \frac{3}{2} - \frac{4}{2} \\
    & = & -\frac{1}{2} \\
  \end{eqnarray*}
\end{solution}

\question[5]
\( \displaystyle
( \frac{1}{2} + \frac{3}{5} ) + \frac{1}{5} + (\frac{1}{2} \cdot 8 + \frac{1}{2})
\)
\begin{solution}[3 cm]
  \begin{eqnarray*}
    (\frac{1}{2} + \frac{3}{5} ) + \frac{1}{5} + (\frac{1}{2} \cdot 8 + \frac{1}{2})
      & = & \frac{1}{2} + \frac{1}{2} + \frac{3}{5} + \frac{1}{5} + \frac{8}{2} \\
      & = & 1 + \frac{4}{5} + 4 \\
      & = & 5\ 4/5 \\
  \end{eqnarray*}
\end{solution}

\question[5]
\(
  [57 + (-93)] + (-58)
\)
\begin{solution}[3 cm]
  \begin{eqnarray*}
    [57 + (-93)] + (-58)
      & = & 57 - 58 - 93 \\
      & = & -1 - 93 \\
      & = & -94 \\
  \end{eqnarray*}
\end{solution}

\question[5]
\(
  3(-2)^3 + 4(-2)^2 -9(-2) - 14
\)
\begin{solution}[3 cm]
  \begin{eqnarray*}
    3(-2)^3 + 4(-2)^2 -9(-2) - 14
      & = & 3(-8) + 4(4) + 18 - 14 \\
      & = & -24 + 16 + 4 \\
      & = & -24 + 20 \\
      & = & -4 \\
  \end{eqnarray*}
\end{solution}

\question[5] Simplify \( 6x^2 - 7x - 2x^2 - 9x - 2 \) by combining similar terms.
\begin{solution}[3 cm]
  \begin{eqnarray*}
    6x^2 - 7x - 2x^2 - 9x - 2
      & = & 4x^2 -7x -9x -2 \\
      & = & 4x^2 -16x -2 \\
  \end{eqnarray*}
\end{solution}

\question[5] \label{ne:last} Simplify \( 3(3x + 2) -5(2x - 3) + 2(-3x - 1) \) by combining similar terms.
\begin{solution}[3 cm]
  \begin{eqnarray*}
    3(3x + 2) -5(2x - 3) + 2(-3x - 1)
      & = & 9x + 6 - 10x + 15 - 6x - 2 \\
      & = & 9x -10x - 6x + 6 + 15 - 2 \\
      & = & -7x + 19 \\
  \end{eqnarray*}
\end{solution}


\subsection{Evaluating Algebraic Expressions}


For problems \ref{ae:first}-\ref{ae:last}, evaluate each algebraic expression for the given values of the variables.

\question[5] \label{ae:first}
\( 5x + 2y \) for \( x = -6 \) and \( y = 5 \)
\begin{solution}[3 cm]
  \begin{eqnarray*}
    5x + 2y & = & 5(-6) + 2(5)    \\
      & = & -30 + 10  \\
      & = & -20 \\
  \end{eqnarray*}
\end{solution}


\question[5]
\( 8a^2 - 2b^2 \) for \( \displaystyle a = -\frac{3}{4} \) and \( \displaystyle b = \frac{1}{2} \)

\begin{solution}[3 cm]
  \begin{eqnarray*}
    8a^2 -2b^2  & = & 8(-\frac{3}{4})^2 - 2(\frac{1}{2})^2    \\
      & = & 8(\frac{9}{16}) - 2(\frac{1}{4})  \\
      & = & \frac{9}{2} - \frac{1}{2} \\
      & = & \frac{8}{2} \\
      & = & 4
  \end{eqnarray*}
\end{solution}

\question[5]
\( -5n^2 - 6n + 7n^2 + 5n - 1 \) for \( n = -6 \)

\begin{solution}[3 cm]
  \begin{eqnarray*}
    -5n^2 - 6n + 7n^2 + 5n - 1  & = & 7n^2 - 5n^2 + 5n - 6n - 1 \\
      & = & 2n^2 - n - 1 \\
      & = & 2(-6)^2 - n - 1 \\
      & = & 2(36) - (-6) - 1 \\
      & = & 72 + 6 - 1 \\
      & = & 77
  \end{eqnarray*}
\end{solution}

\question[5]
\( -2xy - x + 4y \) for \( x = -3 \) and \( y = 9 \)

\begin{solution}[3 cm]
  \begin{eqnarray*}
     -2xy - x + 4y  & = & -2(-3)(9) - (-3) + 4(9) \\
      & = & 6(9) + 3 + 36  \\
      & = & 54 + 3 + 36 \\
      & = & 90 + 3 \\
      & = & 93 \\
  \end{eqnarray*}
\end{solution}

\question[5]
\( 4(n^2 + 1) - (2n^2 + 3) -2(n^2 + 3n) \) for \( n = -4 \)

\begin{solution}[3 cm]
  \begin{eqnarray*}
     4(n^2 + 1) - (2n^2 + 3) -2(n^2 + 3n)   & = & 4n^2 + 4 - 2n^2 - 3 -2n^2 - 6n \\
      & = & 4n^2 - 2n^2 - 2n^2 -6n + 4 - 3 \\
      & = & -6n + 1 \\
      & = & -6(-4) + 1 \\
      & = & 25 \\
  \end{eqnarray*}
\end{solution}

\question[5]  \label{ae:last}
The formula to convert Farenheit to Celsius is: \(\displaystyle C = \frac{5}{9}(F - 32) \). 

What is the Celsius value for \(104^{\circ}\) Farenheit? 

\begin{solution}[3 cm]
  \begin{eqnarray*}
      \frac{5}{9}(F - 32) & = & \frac{5}{9}(104 - 32) \\
      & = & \frac{5}{9}(72) \\
      & = & 5(\frac{72}{9}) \\
      & = & 5(8) \\
      & = & 40 \\
  \end{eqnarray*}
\end{solution}

\subsection{English to Algebraic Expression}

For problems \ref{eae:first} to \ref{eae:last} answer each question with an algebraic expression.

Example:

\hspace{1 cm} {\em x} is 10 more than half {y}.

Solution:

\hspace{1 cm} \(\displaystyle x = \frac{y}{2} + 10 \)

\vspace{1 cm}

\question[5] \label{eae:first}
The product of {\em x} and {\em y} is 85.  What is {\em y} in terms of {\em x}?

\begin{solution}[3 cm]
  \begin{eqnarray*}
      xy & = & 85 \\
      y  & = & \frac{85}{x} \\
  \end{eqnarray*}
\end{solution}

\question[5] 
Joe has {\em n} nickels, {\em d} dimes, and {\em q} quarters.  How much money, in cents, does he have?

\begin{solution}[3 cm]
  \begin{eqnarray*}
      m & = & 5n + 10d + 25q 
  \end{eqnarray*}
\end{solution}


\question \label{eae:last} 
Although the United States only has 5\% of the world's population, it has a very impressive 25\% of the world's prison
population.

\ifprintanswers
  \begin{em}
  25\% is the same as \( 25/100 \) which reduces to \( 1/4 \).  Many people seemed confused about this point, so the
  problem would probably have been much easier if it had been phrased as:  

  {\em The US has \( 1/4 \) of the world's prison population}.

  The 5\% number is just interesting background information which is not used in the solution.

  \end{em}
\fi

\begin{parts}
\part[5] If the US prison population is {\em U} and the world prison population is {\em W}, what is {\em W} in terms of
  {\em U}?


\begin{solution}[3 cm]

  \begin{eqnarray*}
      U & = & \frac{25}{100} W \\
        & = & \frac{1}{4} W \\
      W & = & 4U
  \end{eqnarray*}

\end{solution}

\part[5]
If the US prison population is {\em U} and the rest of the world's prison population (excluding
the US) is {\em R}, what is {\em R} in terms of {\em U}? 

\begin{solution}[3 cm]

  \begin{eqnarray*}
      U & = & \frac{25}{100} W \\
      R & = & \frac{75}{100} W \\
      \frac{R}{U} & = & \frac{75}{100} \cdot \frac{100}{25} \\
      \frac{R}{U} & = & \frac{75}{25} \\
      \frac{R}{U} & = & 3 \\
      R           & = & 3U \\
  \end{eqnarray*}
\end{solution}

\end{parts}


\noaddpoints

\subsection{Extra Credit}

\question
Alice and Carol run a 100 meter race and Alice wins by 10 meters.  To make the race more interesting the next time,
they decide that Alice should start 10 meters behind the starting line.

\begin{parts}
\part[5]
Is the second race a tie?  If not, who wins and why?

\begin{solution}[6 cm]

Alice also wins the second race.  Alice runs 100 meters in the same time Carol runs 90 meters.  In the second race Alice
starts 10 meters behind.  So after Carol has run 90 meters, the runners are tied.  Alice is the faster runner, so she
runs the last 10 meters faster and wins the race.

\end{solution}[
\part[5]
If the second race is not a tie, how much does the winner win by?

\begin{solution}

Since Alice runs 100 meters in time it takes Carol to run 90 meters, Alice runs 10 meters in the time it takes Carol to run
9 meters.  Since the runners are tied after 90 meters, Carol has only run 9 meters more when Alice finishes.  So Carol loses the
race by a single meter.

Another way to look at it is that Carol only runs at \( 9/10 \) of Alice's speed.  Alice needs to run 110 meters.  In this
time, Carol can run 
\begin{eqnarray*}
  \frac{9}{10} \cdot 110 \text{ meters} & = & 9 (\frac{110}{10}) \text{ meters} \\
    & = & 9 \cdot 11 \text{ meters} \\
    & = & 99 \text{ meters} \\
\end{eqnarray*}

Carol has only run 99 meters when Alice finishes the race.  So Carol loses by a single meter.

\end{solution}

\end{parts}

\end{questions}

\end{document}


