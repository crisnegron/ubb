\documentclass{exam}

\usepackage{units} 
\usepackage{graphicx}
\usepackage[fleqn]{amsmath}
\usepackage{cancel}
\usepackage{float}
\usepackage{mdwlist}
\usepackage{booktabs}
\usepackage{cancel}
\usepackage{polynom}
\usepackage{caption}
\usepackage{fullpage}
\usepackage{xfrac}
\usepackage{enumerate}

\newcommand{\dg}{\ensuremath{^\circ}} 
\everymath{\displaystyle}

\printanswers

\title{Math 142 Notes \\ Section 7.3}

\date{\today}

\begin{document}

  \maketitle
  \tableofcontents

  \section{Double Angle Formulas}

  \subsection{$\cos 2s$}
  \begin{align*}
    \cos(s + s) & = \cos s \cos s - \sin s \sin s \\
    \cos 2s     & = \cos^2 s - \sin^2 s \\
                & = 1 - 2 \sin^2 s \\
                & = 2 \cos^2 s - 1 \\
  \end{align*}

  \subsection{$\sin 2s$}
  \begin{align*}
    \sin(s + s) & = \sin s \cos s + \cos s \sin s \\
    \sin 2s     & = 2 \cos s \sin s \\
  \end{align*}

  \subsection{$\tan 2s$}
  \begin{align*}
    \tan 2s & = \frac{\sin 2s}{\cos 2s} \\
            & = \frac{2 \cos s \sin s}{\cos^2 s - \sin^2 s} \\
            & = \frac{2 \cos s \sin s}{\cos^2 s - \sin^2 s} \cdot \frac{\sfrac{1}{\cos^2 s}}{\sfrac{1}{\cos^2 s}} \\
            & = \frac{2 \tan s}{1 - \tan^2 s} \\
  \end{align*}

  \section{Power Lowering Formulas}

  \subsection{$\sin^2 s$}
  \begin{align*}
    \cos 2s  & = 1 - 2 \sin^2 s \\
    \sin^2 x & = \frac{1 - \cos 2x}{2} \\
  \end{align*}

\end{document}
