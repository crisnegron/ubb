
\documentclass{exam}

\usepackage{graphicx}
\usepackage[fleqn]{amsmath}
\usepackage{cancel}
\usepackage{polynom}
\usepackage{float}
\usepackage{mdwlist}
\usepackage{booktabs}
\usepackage{cancel}
\usepackage{polynom}
\usepackage{caption}

\newcommand{\degree}{\ensuremath{^\circ}} 

% \oddsidemargin .5in
% \topmargin -1in
\textwidth 6.5 in

\printanswers

\ifprintanswers 
\usepackage{2in1, lscape} 
\fi

\title{Physics \\ Homework Four}
\date{October 3, 2011}
% \author{Ed Tellman}

\begin{document}

\maketitle

\section{From the Book}

\begin{itemize*}
  \item Read Chapter 7
  \item Chapter 7
    \begin{itemize*}
      \item questions 2, 6, 7, 9, 12, 19, 24, and 32
      \item exercises 1, 2, 4, 5, 8, 11, and 14
      \item synthesis problems 2 and 3
    \end{itemize*}
\end{itemize*}

\section{Extra Credit}

\begin{questions}

\question 
An object initially at rest breaks into two pieces as the result of an explosion.  One piece has twice the kinetic
energy of the other piece.  What is the ratio of the masses of the two pieces?  Which piece has the larger mass? ({\em James Walker}).

\begin{solution}
Here's what we know about the kinetic energy of the two pieces:
\[
  \frac{1}{2} m_1v_1^2 = 2 \left( \frac{1}{2} m_2v_2^2 \right) = m_2v_2^2
\]

Since the initial momentum before the explosion was zero, the momentum after the explosion is also zero:
\[
  m_1v_1 + m_2v_2 = 0
\]

We can solve the momentum equation for one of the velocities and plug that into the kinetic energy equation:
\begin{align*}
  m_1v_1 + m_2v_2 &= 0 \\
  v_1 &= - \frac{m_2v_2}{m1} \\
  \\
  \frac{1}{2} m_1 \left(- \frac{m_2v_2}{m1} \right)^2 &= m_2 v_2^2 \\
  \frac{1}{2} \left(\frac{m_2^2v_2^2}{m1} \right) &= m_2 v_2^2 \\
  m_2 &= 2m_1 \\
\end{align*}

If, you put the masses back in the original equation, you find that 
\[
  v_2 = - \frac{v_1}{2}
\]

The larger object has twice the mass and half the kinetic energy as the smaller object.

\end{solution}

\end{questions}

\ifprintanswers

\section{Chapter 7}
\begin{description}

\item[Q2]
If the impulses are equal, the force which acted for a longer time must have been smaller.

\item[Q6]
\begin{description}
\item[a] Since the ball reverses direction, the momentum is the same magnitude but also in the opposite direction.
\item[b] Since there is a change in momentum, there is also an impulse.
\end{description}

\item[Q7]
Since $I = Ft$, keeping the ball in contact for a longer time will result in a larger impulse, which will make the ball go farther.

\item[Q9]
In a collision, the passenger decelerates rapidly from driving speed to zero.  If the passenger decelerates by
contacting the dashboard or windshield, the impulse will be delivered over a very short time with a very large force,
causing a great deal of damage to the passenger.  If the passenger declerates by contacting the air bag, the impulse
will be delivered over a longer time with a smaller force, causing less damager to the passenger.

\item[Q12]
Both vehicles have the same velocity, but the truck has larger mass so it also has larger momentum.  Impulse is the
change in momentum, so the truck requires a larger impulse.

\item[Q19]
Momentum is conserved in a closed system.  The initial momentum was zero, so the momentum after the push is also zero.
The momentum of one skater exactly cancels the momentum of the other skater.

\item[Q24]
Before the collision, the momentum was $p_{before} = m_1v_1$.  After the collision, the momentum is 
$p_{after} = (m_1 + m_2) v$  These two momentums must be equal since momentum is conserved.  And since the
post-collision mass is larger, the post-collision velocity must be smaller.

\item[Q32]
Since most of the motion after the collision is to the right, most of the momentum after the collision is to the right.
So the red car must have most of the momentum before the collision.  Since the cars have the same mass, the red car must
have been moving faster before the collision.

\item[E1]
\begin{description}
\item[a]
\[
  I = Ft = (300 \text{ N})(0.04 \text{ s}) = 12 \text{ N s} 
\]

\item[b]
\[
  I = \Delta p = 12 \text{ kg m/s} 
\]
\end{description}

\item[E2]
\[
  p = mv = (1,200 \text{ kg})(27 \text{ m/s}) = 32,400 \text{ kg m/s}
\]

\item[E4]
\begin{description}
\item[a] 
\[
  I = Ft = (45 \text{ N})(0.2 \text{ s}) = 9 \text{ N s}
\]

\item[b] 
The final momentum is the same as the impulse: $9 \text{ kg m/s}$.

\end{description}

\item[E5]
\[
  I = \Delta p = - (0.12 \text{ kg})(40 \text{ m/s}) = -4.8 \text{ N s}
\]

\item[E8]
\begin{description}
\item[a]
\[
  \Delta p = (2.5 \text{ kg m/s}) - (- 2.5 \text{ kg m/s}) = 5 \text{ kg m/s}
\]

\item[b] from part a, $5 \text{ kg m/s}$
\end{description}

\item[E11]
\begin{description}
\item[b] 
Since the initial momentum was zero and no external forces acted on the two skaters, the momentum after they push
off is also zero.

\item[c]

Since the total momentum is zero:
\begin{align*}
  (80 \text{ kg}) & (3 \text{ m/s}) + (32 \text{ kg}) v = 0 \\
  v &= - 7.5 \text{ m/s} \\
\end{align*}

\end{description}

\item[E14]
\begin{description}
\item[a]
\[
  p = mv = (12,000 \text{ kg})(12 \text{ m/s}) = 144,000 \text{ kg m/s}
\]

\item[b]
\begin{align*}
  144,000 \text{ kg m/s} &= (12,000 + 18,000 \text{ kg}) v \\
  v &= 4.8 \text{ m/s} \\
\end{align*}

\end{description}

\item[SP1]

I typed the solution to this one and then I realized that it wasn't one of the assigned problems \ldots

\begin{description}
\item[a]
If we consider the direction from the pitcher to the batter to be positive and from the batter to the pitcher to be negative:
\begin{itemize*}
  \item the initial momentum of the ball is 
    \[ p = mv = (.120 \text{ kg})(40 \text{ m/s}) = 4.8 \text{ kg m/s} \]
  \item the final momentum of the ball is 
    \[ p = mv = (.120 \text{ kg})(-60 \text{ m/s}) = -7.2 \text{ kg m/s} \]
  \item the change is $4.8 - (-7.2) \text{ kg m/s} = 12 \text{ kg m/s}$
\end{itemize*}

\item[b]
The change is greater than the final momentum because the ball reversed direction.

\item[c]
The magnitude of the required impulse is the same as the change in the momentum.

\item[d]
\[
  F = \frac{I}{\Delta t} = \frac{12 \text{ kg m/s}}{0.04 \text{ s}} = 300 \text{ N}
\]

\end{description}

\item[SP2]

You have to retain quite a few significant digits in this problem for all the numbers to work out exactly.

\begin{description}

\item[a]
The initial momentum of the bullet is 
\[ p = mv = (0.005 \text{ kg})(500 \text{ m/s}) = 2.5 \text{ kg m/s} \]

Since momentum is conserved, this is also the momentum of the bullet/block combination after the collision.  So the
velocity of the bullet/block combination is:
\begin{align*}
  p &= mv \\
  2.5 \text{ kg m/s} &= (1.2 \text{ kg} + 0.005 \text{ kg}) v \\
  v &\approx 2.0747 \text{ m/s} \\
\end{align*}

\item[b]
Since the block of wood starts out stationary, the change in momentum for the block of wood is the same as its final
momentum: 
\[
  p = mv = (1.2 \text{ kg})(2.0747 \text{ m/s}) \approx 2.4896 \text{ kg m/s}
\]
This is also the impulse that acts on the block of wood.

\item[c]

The final momentum for the bullet is
\[
  p = mv = (0.005 \text{ kg})(2.0747 \text{ m/s}) \approx 0.0104 \text{ kg m/s}
\]

The change in momentum for the bullet is 
\[
\Delta p = 2.5 \text{ kg m/s} - 0.0104 \text{ kg m/s} = 2.4896 \text{ kg m/s}
\]
which matches the momentum of the block of wood.

This makes sense, because any momentum the bullet lost must have been transferred to the block for the momentum of the bullet/block
system to remain constant.

\end{description}

\item[SP3]
\begin{description}
\item[a]
In case A the ball ends up with zero momentum and the change is from its initial momentum to zero.  In case B, the
ball reverses direction so the change is from its initial momentum to negative its initial momentum, which is twice as
big of a change.

\item[b]
Case B, because that is the larger change in momentum.

\item[c]
Momentum is always conserved, but you have to include the change in the wall's momentum which is probably very difficult to
observe.

\end{description}

\end{description}

\fi


\vspace{3 in}

\ifprintanswers
\else
\begin{em}
% The ancients studied in order to rule themselves.  These days, people study in order to rule others.
Adept Kung was forever comparing and criticizing people.  The Master said: ``To have time for such things, Kung must have
already perfected himself completely.  As for me, I am not so lucky.''
\end{em}

\vspace{.2 cm}
\hspace{1 cm} --The Analects of Confucius

\fi

\end{document}

