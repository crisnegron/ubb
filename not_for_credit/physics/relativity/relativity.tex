
\documentclass[fleqn]{article}

\usepackage{siunitx}
\usepackage{graphicx}
\usepackage{float}
\usepackage{amsmath}
\usepackage{cancel}
\usepackage{polynom}
\usepackage{caption}
\usepackage{mdwlist}

\newcommand{\degree}{\ensuremath{^\circ}} 

\everymath{\displaystyle}

% \usepackage{2in1, lscape} 

\title{Special Relativity}
\date{\today}

\begin{document}

\maketitle

In 1905 when he was a 25-year old clerk at the patent office, Albert Einstein discovered the theory of Special Relativity in
his spare time.

Einstein proposed two simple rules:
\begin{itemize*}
  \item as long as you travel in a straight line, the laws physics are the same, regardless of how fast you are moving
  \item the speed of light in a vacuum is one of the laws of physics
\end{itemize*}

He applied a little algebra and a lot of imagination and discovered that if you follow these two rules, there are some
very interesting consequences:
\begin{itemize*}
  \item time travels more slowly for fast-moving objects
  \item fast moving objects shrink
  \item energy and mass are really two aspects of the same thing ($E=mc^2$)
\end{itemize*}

Why should you care?  Relativity brings us:
\begin{itemize*}
  \item nuclear power
  \item the atom bomb
  \item the Global Position System (GPS)
  \item black holes
\end{itemize*}

No math is required, although a little knowledge of algebra will be helpful at a few points.

\end{document}
