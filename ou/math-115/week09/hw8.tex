
\documentclass[fleqn,addpoints]{exam}

\usepackage{graphicx}
\usepackage{float}
\usepackage{amsmath}
\usepackage{cancel}
\usepackage{polynom}
\usepackage{caption}

% \printanswers

\ifprintanswers \usepackage{2in1, lscape} \fi

\title{Math 115 Homework 8}
\date{December 7, 2010}

\begin{document}

\maketitle
 

``Mathematicians are curious birds,'' the police commissioner said to his wife. ``You see, we had all those partly filled
glasses lined up in rows on a table in the hotel kitchen. Only one contained poison, and we wanted to know which one,
before searching that glass for fingerprints. Our laboratory could test the liquid in each glass, but the tests take
time and money, so we wanted to make as few of them as possible. We phoned the university and they sent over a
mathematics professor to help us. He counted the glasses, smiled, and said:''

``Pick any glass you want, commissioner. We'll test it first.''

``But won't that waste a test?'' I asked.

``No,'' he said. ``It's part of the best procedure. We can test one glass first. It doesn't matter which one.''

``How many glasses were there to start with?'' the commissioner's wife asked.

``I don't remember. Somewhere between 100 and 200.''

\begin{questions}
\question What was the exact number of glasses? (It is assumed that any group of glasses can be tested simultaneously by taking a
small sample of liquid from each, mixing the samples, and making a single test of the mixture.)

\question The mathematician was actually wrong, and testing a single glass isn't the most efficient way to
proceed.  What is the best procedure? 

\end{questions}

\ifprintanswers

\fi

\ifprintanswers
\else
\vspace{8 cm}

{\em Imagine}

\vspace{.1 cm}
\hspace{1 cm} --John Lennon

\fi

\end{document}

