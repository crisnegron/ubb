
\documentclass{exam}

\usepackage{siunitx} 
\usepackage{graphicx}
\usepackage[fleqn]{amsmath}
\usepackage{cancel}
\usepackage{polynom}
\usepackage{float}
\usepackage{mdwlist}
\usepackage{booktabs}
\usepackage{cancel}
\usepackage{polynom}
\usepackage{caption}

\newcommand{\degree}{\ensuremath{^\circ}} 
\everymath{\displaystyle}

% \oddsidemargin .5in
% \topmargin -1in
\textwidth 6.5 in

\printanswers

\ifprintanswers 
\usepackage{2in1, lscape} 
\fi

\title{Physics \\ Homework Ten \\ Magnetism}
\date{November 28, 2011}

\begin{document}

\maketitle

\section{Homework}

\begin{itemize*}
  \item Read Chapter 14, section 4
  \item Chapter 14
    \begin{itemize*}
      \item questions 24, 26, 28
      \item exercises 11-15
      \item synthesis problems: 3-4
    \end{itemize*}
\end{itemize*}

\section{Extra Credit}

\begin{questions}
\question A loop of wire is dropped between the poles of a horseshoe magnet, where the north pole is on the left.  Is
the induced current in the loop clockwise or counterclockwise when the loop is (a) above the magnet (b) below the
magnet?

\begin{solution}
Because of Lenz's law, the induced current will create a magnetic field which opposes the change in flux.  When the loop
is entering the field from above the magnet, the flux through the loop is increasing and the current will be
clockwise to create an opposing magnetic field.  When the loop is exiting the field below the magnet, the current
changes direction and moves counterclockwise to create a magnetic field in the same direction as the original field. 
\end{solution}

\question
Suppose instead of dropping the loop, you attach it to a string and lower it between the poles of the magnet.  Is the
tension in the string greater than, less than, or equal to the weight of the loop when the loop is (a) above the magnet
(b) below the magnet? 

\begin{solution}
When the loop is above the magnet, the current in the loop generates a magnetic field which opposes the original field.
The opposing magnetic forces decrease the tension on the rope when the loop is lowered.

When the loop is below the magnet, the current in the loop opposes the motion by generating a magnetic field which is
attracted to the original field.  The tension on the rope is again decreased.

\end{solution}

\question
With the loop still attached to the string, you now raise it between the poles instead of lowering it.  Is the
tension in the string greater than, less than, or equal to the weight of the loop when the loop is (a) above the magnet
(b) below the magnet? 

\begin{solution}
When the loop is below the magnet, the current in the loop generates a magnetic field which opposes the original field.
The opposing magnetic forces increase the tension on the rope.

When the loop is above the magnet, the current in the loop opposes the motion by generating a magnetic field which is
attracted to the original field.  The tension on the rope is again increased.

\end{solution}

\end{questions}

\ifprintanswers

\section{Questions}

\begin{description}

\item[Q24]
The current is clockwise, so that it creates a field which opposes the change in the flux.

\item[Q28]
A simple generator produces an alternating current because the rate of the change in the flux alternates between
positive (increasing flux) and negative (decreasing flux).

\end{description}

\section{Exercises}

\begin{description}

\item[E11]
\[
  V = \frac{\Delta \Phi}{\Delta t} = \frac{6 \ \tesla \cdot \meter^2}{0.25 \ \second} = 24 \ \volt
\]

\item[E12]
\begin{description}
\item[a]
\[
  \Phi = NBA = 60 \cdot (1.5 \ \tesla)(0.02 \ \meter^2) = 1.8 \ \tesla \cdot \meter^2
\]

\item[b]
\[
  V = \frac{\Delta \Phi}{\Delta t} = \frac{0.03 \ \tesla \cdot \meter^2}{0.2 \ \second} = 9 \ \volt
\]

\end{description}

\item[E13]
\begin{description}
\item[a]
Since the number of turns on the secondary coil is larger than the number of turns on the primary coil, this is a
step-up transformer.

\item[b]
\[
  V_s = V_p \cdot \frac{N_s}{N_p} = (110 \ \volt) \cdot \frac{60}{15} = 440 \ \volt
\]

\end{description}

\item[E14]
\begin{align*}
  V_s I_s &= V_p I_p \\
  I_s &= \frac{V_p}{V_s} \cdot I_p \\
      &= \frac{1}{4} \cdot (6 \ \ampere) \\
      &= 1.5 \ \ampere \\
\end{align*}



\item[E15]
\begin{align*}
  V_s &= V_p \cdot \frac{N_s}{N_p} \\
  N_s &= N_p \cdot \frac{V_s}{V_p} \\
      &= 300 \cdot \frac{6 \ \volt}{120 \ \volt} \\
      &= 15 \ \text{ turns} \\
\end{align*}

\end{description}

\section{Synthesis Problems}

\begin{description}
\item[SP3]
\begin{description}

\item[a]
\[
  A = lw = (0.03 \ \meter)(0.06 \ \meter) = 0.0018 \ \meter^2
\]

\item[b]
\[
  \Phi = BA =60 \cdot  (0.4 \ \tesla)(0.0018 \ \meter^2) = 0.0432 \ \tesla \cdot \meter^2
\]

\item[c]
The minimum value is zero when the coil is parallel to the field.

\item[d]
With one complete turn each second, it takes 0.25 seconds for the loop to move from perpendicular to parallel.

\item[e]
\[
  V = \frac{\Delta \Phi}{\Delta t} = \frac{0.0432 \ \tesla \cdot \meter^2}{0.25 \ \second} = 0.173 \ \volt
\]

\end{description}

\item[SP4]
\begin{description}
\item[a]
\[
  N_s = \frac{V_s}{V_p} \cdot N_p = \frac{22 \ \volt}{110 \ \volt} \cdot 400 = 80 \ \text{ turns}
\]

\item[b]
\[
  I_s = I_p \cdot \frac{V_p}{V_s} = (5 \ \ampere) \frac{110 \ \volt}{22 \ \volt} = 25 \ \ampere
\]

\item[c]
If the transformer gets warm, some of the original energy is being converted to heat.  The energy remaining is reduced
and the current in the secondary coil will be less than $25 \ \ampere$.

\end{description}

\end{description}

\else

\vspace{4.5 cm}

{\em 
Voting is easy and marginally useful, but it is a poor substitute for democracy, which requires direct action by concerned citizens.
} 

\vspace{.2 cm}

\hspace{1 cm} --Howard Zinn

\fi

\end{document}

