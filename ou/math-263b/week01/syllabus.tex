\documentclass[fleqn, onecolumn]{article}

\usepackage{fullpage}
\usepackage{graphicx}
\usepackage{float}
\usepackage{amsmath}
\usepackage{amssymb}
\usepackage{polynom}
\usepackage{caption}
\usepackage{mdwlist}
\usepackage{parskip}


\newcommand{\degree}{\ensuremath{^\circ}} 

\everymath{\displaystyle}
\setlength{\mathindent}{1 cm}

\title{Math 263B Syllabus}
\date{\today}

\begin{document}

\maketitle

\section{Overview}
Math 263A covered derivatives.  Math 263B covers the inverse operation which is {\em integrals}. Derivatives allow you
to find the rate at which a function changes from a function.  Integrals allow you to find a function, given the rate at
which it changes.


The text book is {\em Calculus, 7th edition} by Varberg and Purcell.  Here are the sections we'll cover, as described in
the OU course guide:

\begin{quote}
Math 263B covers Chapters 5, 7, 8, 9, and the first two sections of Chapter 6. The following sections are omitted: 5.2,
5.10, 6.3-6.8, 7.10, 8.7, 9.6. In addition, you can omit the Mean Value Theorem for Integrals, page 281
of the text, and the material on Inverse Hyperbolic Functions, beginning on page 393 of the text.
\end{quote}

\section{Homework and Exams}

You should expect to spend two or three hours each week doing homework.

Math is like learning piano, basketball, or bicycle mechanics.  Watching someone else do it or reading about it in a
book is helpful.  But you can't actually learn how to do it yourself unless you've practiced on your own.

Each chapter will be followed by an in-class test.  The primary reason to have in-class tests is so that you can
practice for the OU exam which is the way you receive credit at the end of the course.  Taking tests is like any other
skill---you get better at it with practice.

\section{Course Overview}

\subsection{Chapter Five--The Integral}

%% Many mathemetical operations have inverses.  Examples are plus/minus, multiply/divide, square root/square, etc.  You can
%% apply the first operation and then undo it by applying the second operation. Of course, you don't usually want to apply
%% an operation and its inverse right sequentially, because then you are left right where you started and haven't
%% accomplished anything.

Derivitaves are useful because given a function, you can get a new function which describes the rate at which the
original function changes, or its slope.  For example, given a function which describes the position of an object as a
function of time, you can find functions which describe its velocity and its acceleration as a function of time.

For example, if $x(t)$ provides the position of something at time $t$:
\begin{align*}
  x(t) &= 5t^2 \\
  v(t) &= x'(t) = 10t \\
  a(t) &= v'(t) = 10 \\
\end{align*}

% As we learned last semester, you can use the slope to find minimums, maximums, related rates, etc.  

Integrals are useful in going the opposite direction.  Given a function which describes the acceleration of an object as
a function of time, you can find functions which describe its velocity and position as a function of time.

\begin{align*}
  a(t) &= 10 \\
  v(t) &= \int 10 \, \mathrm{d}t = 10 t \\
  s(t) &= \int 10t \, \mathrm{d}t = 5t^2 \\
\end{align*}

In this chapter we learn what an integral is, how integrals are related to derivatives, and how to calculate some integrals.

\subsection{Chapter Six--Applications of the Integral}

We learn how to use integrals to compute the area and volume of objects.

\subsection{Chapter 7--Transendental Functions}

{\em Transcendental Functions} are functions like sine and cosine which can't be written as polynomials.

Two other transendental functions we learned about in pre-calculus were $f(x) = e^x$ and $f(x) = \ln x$.  There is a
natural way to define these functions using integrals which we learn about in this chapter.  

This chapter also includes a review of derivatives and integrals of trigonometric functions and introduces hyperbolic functions.

\subsection{Chapter 8--Techniques of Integration}

Doing derivatives required learning some rules about what to do when faced with various situations.  Integrals are the
same way, and this chapter includes some techniques to use for integrating various expressions.

\subsection{Chapter 9--Indeterminate Forms and Improper Integrals}

In this chapter we learn:
\begin{itemize}
\item Easier and more powerful ways to solve expressions which result in $0/0$ or $\infty/\infty$.  For example, we
  learn an easy way to evaluate $\lim_{x \to 0} \frac{\sin x}{x}$.

\item What to do when one of the limits of integration is infinite.
\end{itemize}

%% Last semester we learned about the derivative.  This semester we'll learn about its inverse operation--the integral.  

%% Subjects include:
%% \begin{itemize*}
%% \item antiderivatives (indefinite integrals).  An antiderivative or ind
%% \item differential equations 
%% \item sums
%% \item using integrals to calculate area
%% \item definite integrals
%% \item the Fundamental Theorem of Calculus
%% \end{itemize*}

\end{document}

