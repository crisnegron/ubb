\documentclass{exam}

\usepackage{units} 
\usepackage{graphicx}
\usepackage[fleqn]{amsmath}
\usepackage{cancel}
\usepackage{float}
\usepackage{mdwlist}
\usepackage{booktabs}
\usepackage{cancel}
\usepackage{polynom}
\usepackage{caption}
\usepackage{fullpage}
\usepackage{xfrac}
\usepackage{enumerate}

\newcommand{\dg}{\ensuremath{^\circ}} 
\everymath{\displaystyle}

\printanswers

\title{Math 142 Notes \\ Section 7.5}

\date{\today}

\begin{document}

  \maketitle
  \tableofcontents

  \section{Solving Trigonometric Equations}
  \begin{align*}
    2 \cos x - \sqrt{2} & = 0 \\
    x                   & = \left\{ \frac{\pi}{4}, \frac{7\pi}{4}, \frac{9\pi}{4}, etc. \right\} \\
                        & = \left\{ \frac{\pi}{4} + 2k \pi, \frac{7\pi}{4} + 2k \pi \right\} \\
  \end{align*}

  \begin{itemize*}
    \item isolate trigonometric function on left side
    \item find solutions between 0 and period ($2 \pi$ or $\pi$)
    \item remaining solutions are found by adding multiples of period 
  \end{itemize*}

  \section{Quadratic Functions}
  \begin{align*}
    4 \cos^2 x - 1               & = 0 \\
    (2 \cos x - 1)(2 \cos x + 1) & = 0 \\
    \\
    \cos x & = \frac{1}{2} \\
    x      & = \left\{ \frac{\pi}{3}, \frac{5 \pi}{3} \right\} \\
    \\
    \cos x & = - \frac{1}{2} \\
    x      & = \left\{ \frac{2 \pi}{3}, \frac{4 \pi}{3} \right\} \\
    \\
    x      & = \boxed{ \left\{ \frac{\pi}{3} + k \pi, \frac{2 \pi}{3} + k \pi \right\} } \\
  \end{align*}

\end{document}
