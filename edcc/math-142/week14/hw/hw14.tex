\documentclass{exam}

\usepackage{units} 
\usepackage{graphicx}
\usepackage[fleqn]{amsmath}
\usepackage{cancel}
\usepackage{float}
\usepackage{mdwlist}
\usepackage{booktabs}
\usepackage{cancel}
\usepackage{polynom}
\usepackage{caption}
\usepackage{fullpage}
\usepackage{comment}
\usepackage{enumerate}
\usepackage{xfrac}
\usepackage{parskip}

\newcommand{\dg}{\ensuremath{^\circ}} 
\everymath{\displaystyle}

\printanswers
\excludecomment{comment}

\ifprintanswers 
  \usepackage{2in1, lscape} 
\fi

\author{}
\date{December 31, 2014}
\title{Math 142 \\ Homework Fourteen}

\begin{document}

  \maketitle

  \section{Homework}
  Section 7.3:

  \section{Extra Credit}
  Section 7.3: 

  \ifprintanswers
    \pagebreak
    \begin{description}

      \item[87] 
        \begin{align*}
          (\tan x + \cot x)^4 & = \left[ (\tan x + \cot x)^2 \right]^2 \\
                              & = \left( \tan^2 x + 2 \tan x \cot x + \cot^2 x \right)^2 \\
                              & = \left( \tan^2 x + 2 + \cot^2 x \right)^2 \\
                              & = \left( \sec^2 x - 1 + 2 + \csc^2 x - 1 \right)^2 \\
                              & = \left( \sec^2 x + \csc^2 x \right)^2 \\
                              & = \left( \frac{1}{\cos^2 x} + \frac{1}{\sin^2 x} \right)^2 \\
                              & = \left( \frac{\sin^2 x + \cos^2 x}{\sin^2 x \cos^2 x} \right)^2 \\
                              & = \left( \frac{1}{\sin^2 x \cos^2 x} \right)^2 \\
                              & = \csc^4 x \sec^4 x \\
        \end{align*}

    \end{description}
  \fi

  \section{Review}

  \ifprintanswers
    \section{Section 7.1}
    \begin{description}

      \item[1] 
        \begin{tabular}[H]{ccc}
          \toprule
          $\sin x$           & $\cos x$           & $\tan x$      \\
          $\sfrac{5}{13}$    & $\sfrac{12}{13}$   & $\sfrac{5}{12}$ \\
          \midrule
          $\sin 2x$          & $\cos 2x$          & $\tan 2x$      \\
          $\sfrac{120}{169}$ & $\sfrac{119}{169}$ & $\sfrac{120}{119}$ \\
          \bottomrule
        \end{tabular}

      \item[2] 
        \begin{tabular}[H]{ccc}
          \toprule
          $\sin x$          & $\cos x$         & $\tan x$      \\
          $\sfrac{4}{5}$    & $- \sfrac{3}{5}$ & $-\sfrac{4}{3}$ \\
          \midrule
          $\sin 2x$         & $\cos 2x$        & $\tan 2x$      \\
          $-\sfrac{24}{25}$ & $-\sfrac{7}{25}$ & $-\sfrac{24}{7}$ \\
          \bottomrule
        \end{tabular}

      \item[9]
        \begin{align*}
          \sin^4 x & = \left( \sin^2 x \right)^2 \\
                   & = \left( \frac{1 - \cos 2x}{2} \right)^2 \\
                   & = \frac{1 - 2 \cos 2x + \cos^2 2x}{4} \\
                   &= \frac{1}{4} - \frac{\cos 2x}{2} + \frac{\cos^2 2x}{4} \\
                   &= \frac{1}{4} - \frac{\cos 2x}{2} + \frac{1 + \cos 4x}{8} \\
        \end{align*}

      \item[10]
        \begin{align*}
          \cos^4 x & = \left( \cos^2 x \right)^2 \\
                   & = \left( \frac{1 + \cos 2x}{2} \right)^2 \\
                   & = \frac{1 + 2 \cos 2x + \cos^2 2x}{4} \\
                   &= \frac{1}{4} + \frac{\cos 2x}{2} + \frac{\cos^2 2x}{4} \\
                   &= \frac{1}{4} + \frac{\cos 2x}{2} + \frac{1 + \cos 4x}{8} \\
        \end{align*}

      \item[11]
        \begin{align*}
          \cos^2 x \sin^4 x & = \cos^2 x \left( \sin^2 x \right)^2 \\
                            & = \left( \frac{1 + \cos 2x}{2} \right) \left( \frac{1 - \cos 2x}{2} \right) \left( \frac{1 - \cos 2x}{2} \right) \\
                            & = \left( \frac{1 - \cos^2 2x}{4} \right) \left( \frac{1 - \cos 2x}{2} \right) \\
                            & = \frac{1 - \cos^2 2x - \cos 2x + \cos^3 2x}{8} \\
                            & = \frac{1}{8} \left( 1 - \frac{1 + \cos 4x}{2} - \cos 2x + \cos 2x \left( \frac{1 + \cos 4x}{2} \right) \right) \\
                            & = \frac{1}{8} \left( 1 - \frac{1}{2} - \frac{\cos 4x}{2} - \cos 2x + \frac{\cos 2x}{2} + \frac{\cos 2x \cos 4x}{2} \right) \\
                            & = \frac{1}{8} \left( \frac{1}{2} - \frac{\cos 2x}{2} - \frac{\cos 4x}{2} + \frac{\cos 2x \cos 4x}{2} \right) \\
                            & = \frac{1}{16} ( 1 - \cos 2x - \cos 4x + \cos 2x \cos 4x ) \\
        \end{align*}

      \item[12]
        \begin{align*}
          \cos^4 x \sin^2 x & = \left( \cos^2 x \right)^2 \sin^2 x \\
                            & = \left( \frac{1 + \cos 2x}{2} \right) \left( \frac{1 + \cos 2x}{2} \right) \left( \frac{1 - \cos 2x}{2} \right) \\
                            & = \left( \frac{1 + \cos 2x}{4} \right) \left( \frac{1 - \cos^2 2x}{2} \right) \\
                            & = \frac{1}{8} \left( 1 - \cos^2 2x + \cos 2x - \cos^3 2x \right) \\
                            & = \frac{1}{8} \left( 1 + \cos 2x - \frac{1 + \cos 4x}{2}  - \frac{\cos 2x}{2} \left( \frac{1 + \cos 4x}{2} \right) \right) \\
                            & = \frac{1}{8} \left( 1 + \cos 2x - \frac{1}{2}  - \frac{\cos 4x}{2} - \frac{\cos 2x}{2} - \frac{\cos 2x \cos 4x}{2} \right) \\
                            & = \frac{1}{8} \left( \frac{1}{2} + \frac{\cos 2x}{2} - \frac{\cos 4x}{2}  - \frac{\cos 2x \cos 4x}{2} \right) \\
                            & = \frac{1}{16} \left( 1 + \cos 2x - \cos 4x  - \cos 2x \cos 4x \right) \\
        \end{align*}

      \item[13]
        \begin{align*}
          \cos^4 x \sin^4 x & = \left( \frac{1 - \cos 2x}{2} \right)^2 \left( \frac{1 + \cos 2x}{2} \right)^2 \\
                            & = \left( \frac{1 - \cos^2 2x}{4} \right)^2 \\
                            % & = \left( \frac{1}{4} - \frac{\cos^2 2x}{4} \right)^2 \\
                            & = \left( \frac{1}{4} - \frac{1 + \cos 4x}{8} \right)^2 \\
                            & = \left( \frac{1}{8} - \frac{\cos 4x}{8} \right)^2 \\
                            & = \frac{1}{64} \left( 1 - \cos 4x \right)^2 \\
                            & = \frac{1}{64} \left( 1 - 2 \cos 4x + \cos^2 4x \right) \\
                            & = \frac{1}{64} \left( 1 - 2 \cos 4x + \frac{1 + \cos 8x}{2} \right) \\
                            % & = \frac{1}{64} \left( \frac{3}{2} - 2 \cos 4x + \frac{\cos 8x}{2} \right) \\
                            & = \frac{1}{128} \left( 3 - 4 \cos 4x + \cos 8x \right) \\
        \end{align*}

      \item[15]
        \begin{align*}
          \sin 15 \dg & = \sin \frac{30 \dg}{2} \\
                      & = \sqrt{  \frac{1 - \sfrac{\sqrt{3}}{2}}{2} }
                      & = \frac{1}{2} \sqrt{2 - \sqrt{3}} \\
        \end{align*}

      \item[16]
        \begin{align*}
          \tan 15 \dg & = \tan \frac{30 \dg}{2} \\
                      & = \frac{ 1 - \sfrac{\sqrt{3}}{2} }{ \sfrac{1}{2} } \\
                      & = 2 - \sqrt{3}
        \end{align*}

      \item[17]
        \begin{align*}
          \tan 22.5 \dg & = \tan \frac{45 \dg}{2} \\
                        & = \frac{1 - \sfrac{\sqrt{2}}{2}}{\sfrac{\sqrt{2}}{2}} \\
                        & = \sqrt{2} - 1 \\
        \end{align*}

      \item[18]
        \begin{align*}
          \sin 75 \dg & = \sin \frac{150 \dg}{2} \\
                      & = \sqrt{ \frac{1 - (\sfrac{\sqrt{3}}{2})}{2} } \\
                      & = \frac{1}{2} \sqrt{2 + \sqrt{3}} \\
        \end{align*}

      \item[19]
        Since $165 \dg$ is in Q-II, $\cos 165 dg < 0$
        \begin{align*}
          \cos 165 \dg & = - \cos \frac{330 \dg}{2}  \\
                       & = - \sqrt{ \frac{1 + \sfrac{\sqrt{3}}{2}}{2} } \\
                       & = - \frac{1}{2} \sqrt{ 2 + \sqrt{3} } \\
        \end{align*}

      \item[23]
        \begin{align*}
          \cos \frac{\pi}{12} & = \cos \frac{\sfrac{\pi}{6}}{2} \\
                              & = \sqrt{ \frac{1 + \sfrac{\sqrt{3}}{2}} {2} } \\
                              & = \frac{1}{2} \sqrt{2 + \sqrt{3}} \\
        \end{align*}

      \item[24]
        \[
          \tan \frac{5 \pi}{12} = \tan \frac{\sfrac{5 \pi}{6}}{2} = 2 + \sqrt{3}
        \]

      \item[25]
        \begin{align*}
          \sin \frac{9 \pi}{8} & = \sin \frac{\sfrac{9 \pi}{4}}{2} \\
                               & = \pm \sqrt{ \frac{1 - \sfrac{1}{\sqrt{2}}}{2} } \\
                               % & = \pm \sqrt{ \frac{2 - \sqrt{2}}{4} } \\
                               & = \pm \frac{1}{2} \sqrt{ 2 - \sqrt{2} } \\
        \end{align*}

        Since $\frac{9 \pi}{8}$ is in Q-III:
        \[
          \sin \frac{9 \pi}{8} = - \frac{1}{2} \sqrt{ 2 - \sqrt{2} } 
        \]

    \end{description}

  \else
    \vspace{5 cm}

    \begin{quote}
      \begin{em}
      \end{em}
    \end{quote}
    \hspace{1 cm} --TO DO
  \fi

\end{document}

