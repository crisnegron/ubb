\documentclass[fleqn,addpoints]{exam}
\usepackage{amsmath}
\usepackage{graphicx}
\usepackage{cancel}
\usepackage{polynom}

\printanswers

\ifprintanswers
\usepackage{2in1, lscape}
\fi

\title{Math 113 Ohio University Exam Problems}
\author{}
\date{October 23, 2010}

% \oddsidemargin 0in
% \topmargin -0.5in
% \textwidth 6.5in

% \extrawidth{-1 in}
% \setlength{\mathindent}{0in}

\begin{document}

\maketitle

\begin{questions}

\question
Perform the indicated division.
\[
  (x^5 - 32) \div (x-2)
\]

\begin{solution}[1cm]

\[ \polylongdiv{x^5 - 32}{x-2} \]

\end{solution}


\question
Perform the indicated operation and write the result in simplest form.

\[
  \frac{36x^2-49}{-x+5} \cdot \frac{x^2-10x+25}{14-2x}
\]

\begin{solution}[1cm]
\begin{align*}
    \frac{36x^2-49}{-x+5} \cdot \frac{x^2-10x+25}{14-2x} &= \frac{(6x+7)(6x-7)(x-5)\cancel{(x-5)}}{-2\cancel{(x-5)}(7-x)} \\
        &= \frac{(6x+7)(6x-7)(x-5)}{2(x-7)} \\
\end{align*}

\end{solution}

\question
Solve the absolute inequality.

\[
  \left| \frac{4x+3}{2} \right| - 2 \leq 5
\]

\begin{solution}[1cm]

\begin{align*}
  \left| \frac{4x+3}{2} \right| - 2 &\leq 5 \\
  \left| \frac{4x+3}{2} \right|  &\leq 7 \\
\end{align*}

\begin{align*}
  -7 &\leq \frac{4x+3}{2} \leq 7 \\
  -14 &\leq 4x+3 \leq 14 \\
  -17 &\leq 4x \leq 11 \\
  -\frac{17}{4} &\leq x \leq \frac{11}{4} \\
\end{align*}

\[
  x = \left[-\frac{17}{4}, \frac{17}{4} \right]
\]
\end{solution}

\question
Solve the system of equations.

\begin{align*}
  7x + 2  &= 5y \\
  3x + 2y &= 11 \\
\end{align*}

\begin{solution}[1cm]
\begin{align*}
  7x + 2  &= 5y \\
  3x + 2y &= 11 \\
  \\
  7x - 5y &= -2 \\
  3x + 2y &= 11 \\
  \\
  14x - 10y &= -4 \\
  15x + 10y &= 55 \\
  \\
  29x &= 51
  \\
  x &= \frac{51}{29}
\end{align*}

\begin{align*}
  3 \left( \frac{51}{29} \right) + 2y &= 11 \\
  y &= \frac{83}{29} \\
\end{align*}

\end{solution}

\question
Find the equation of a line passing through $(3, -2)$ and parallel to the line $2x-y+10 = 0$

\begin{solution}[1cm]
First we should put the original line in slope-intercept form so we know what slope to look for:
\begin{align*}
  2x-y+10 &= 0 \\
  y &= 2x + 10 \\
\end{align*}

So we need a line with a slope of 2 which passes through $(3, -2)$.

\begin{align*}
  y - (-2) &= 2(x - 3) \\
  y +2 &= 2x- 6 \\
  y &= 2x - 8 \\
\end{align*}

\end{solution}

\question
Factor the following expressions.

\begin{parts}

\part $20x^4 - 17x^2 - 63$
\begin{solution}[1cm]
\begin{align*}
  20x^4 - 17x^2 - 63 &= (5x^2+7)(4x^2-9) \\
  &= (5x^2+7)(2x+3)(2x-3) \\
\end{align*}

\end{solution}

\part $x^2y - 3x^2w + 2xy - 6xw$
\begin{solution}[1cm]
\begin{align*}
  x^2y - 3x^2w + 2xy - 6xw &= x^2(y-3w) + 2x(y-3w) \\
    &= (x^2 + 2x)(y-3w) \\
    &= x(x + 2)(y-3w) \\
\end{align*}

\end{solution}

\end{parts}

\ifprintanswers
\pagebreak
\fi

\question
Solve for x:
\[
  2x^2 - 6x + 5 = 0
\]

\begin{solution}[1cm]
\begin{align*}
  x = \frac{6 \pm \sqrt{36-4 \cdot 2 \cdot 5}}{4} = \frac{3}{2} \pm \frac{i}{2}
\end{align*}

\end{solution}

\question
Simplify the following radicals

\begin{parts}
\part $\sqrt[5]{- \dfrac{81}{16} x^6y^{-3}}$

\begin{solution}[1cm]
\begin{align*}
  \sqrt[5]{- \dfrac{81}{16} x^6y^{-3}} &= -\sqrt[5]{\dfrac{81x^6}{16y^3}} \\
  &= -\sqrt[5]{\dfrac{81x^6}{16y^3}} \left( \frac{2y^2}{2y^2} \right) \\
  &= - \frac{x \sqrt[5]{162xy^2}}{2y} 
\end{align*}

\end{solution}

\part $\sqrt{x^4+x^6y^2}$

\begin{solution}[1cm]
\[
  \sqrt{x^4+x^6y^2} = \sqrt{x^4(1+x^2y^2)} = x^2 \sqrt{1+x^2y^2}
\]
\end{solution}

\end{parts}

\question
Solve for x:
\[
\frac{1}{x-1} = 3 + \frac{2}{x-3}
\]

\begin{solution}[1cm]
\begin{align*}
  \frac{1}{x-1} &= 3 + \frac{2}{x-3} \\
  \frac{1}{x-1} &= \frac{3x-9+2}{x-3} \\
  \frac{1}{x-1} &= \frac{3x-7}{x-3} \\
  x-3 &= (x-1)(3x-7) \\
  3x^2 - 11x + 10 &= 0 \\
  (3x-5)(x-2) &= 0
\end{align*}

$x = \left\{\dfrac{5}{3}, 2 \right\}$

\end{solution}

\question
Find the quotient and express the answer in the standard form of a complex number
\[
\frac{2+3i}{3-2i}
\]

\begin{solution}[1cm]
\begin{align*}
  \frac{2+3i}{3-2i} &= \frac{2+3i}{3-2i} \left( \frac{3+2i}{3+2i} \right) \\
  &= \frac{6 + 9i + 4i + 6i^2}{9-4i^2} \\
  &= \frac{6 + 13i -6}{9+4} \\
  &= \frac{13i}{13} \\
  &= i \\
\end{align*}

\end{solution}

\ifprintanswers
\pagebreak
\fi


\question
Perform the indicated operations and reduce the result to lowest terms.
\[
x^2 + x + 1 + \frac{1}{x-1}
\]

\begin{solution}[1cm]
\begin{align*}
  x^2 + x + 1 + \frac{1}{x-1} &= \frac{(x-1)(x^2+x+1) + 1}{x-1} \\
  &= \frac{x^3 + x^2 + x - x^2 - x - 1 + 1}{x-1} \\
  &= \frac{x^3}{x-1} \\
\end{align*}

\end{solution}

\question
Solve for x:
\[
  12x^{-2} - 17x^{-1} - 5 = 0
\]

\begin{solution}[1cm]
\begin{align*}
  12x^{-2} - 17x^{-1} - 5 &= 0 \\
  12 - 17x - 5x^2 &= 0 \\
  5x^2 + 17x - 12 &= 0 \\
  (5x-3)(x+4) &= 0 \\
\end{align*}

$x = \left\{ -4, \dfrac{3}{5} \right\}$

\end{solution}

\ifprintanswers
\pagebreak
\fi


\question
Solve for y in terms of x:
\[
  xy - 7y = 4xy + 3x - 2
\]

\begin{solution}[1cm]
\begin{align*}
  xy - 7y &= 4xy + 3x - 2 \\
  xy - 7y - 4xy &= 3x - 2 \\
  - 3xy - 7y  &= 3x - 2 \\
  y(-3x -7) &= 3x - 2 \\
  y &= \frac{3x - 2}{- 3x -7} \\
\end{align*}

\end{solution}

\question
Howard lives 2 miles from school.  When he rode his bicycle, he traveled 5 mph faster than when he walked, and thus saved
25 minutes.  What was is average rate of walking? 

\begin{solution}[1cm]
Here are the facts from the equation, putting everything in hours:
\begin{itemize}
  \item $d = 2$
  \item $r_b = r_w + 5$
  \item $t_b = t_w - \dfrac{25}{60} = t_w - \dfrac{5}{12}$
\end{itemize}

Using $r=\dfrac{d}{t}$, $r_b = \dfrac{2}{t_b}$ and $r_w = \dfrac{2}{t_w}$.  We can plug these into the second equation
to get: $dfrac{2}{t_b} = \dfrac{2}{t_w} + 5$.  We can substitute the third equation into this one to get
\[
  2\left(t_w - \dfrac{5}{12} \right) = \dfrac{2}{t_w} + 5
\]

With a little work, this turns into:
\begin{align*}
  60t_w^2 - 25t_w - 10 &= 0 \\
  (4t_w + 1)(3t_w - 2) &= 0 \\
\end{align*}

So $t_w$ is either $- \dfrac{1}{4}$ or $\dfrac{2}{3}$.  A negative number doesn't make sense for a time, so $t_w =
\dfrac{2}{3}$.  This means the rate walking was 3 mph and the rate biking was 8 mph.  

8 mph is a pretty slow biking rate, but perhaps it was uphill both ways.

\end{solution}

\question
Separate 83 into two parts such that if the larger is divided by the smaller, the quotient is 4 with a remainder of 8.

\begin{solution}[1cm]
From the problem statement, we have the two equations:
\begin{align*}
  x+y &= 83 \\
  y &= 4x + 8 \\
\end{align*}

We can plug the second equation into the first one to find $x$:
\begin{align*}
  x+4x+8 &= 83 \\
  5x &= 75 \\
  x &= 15 \\
\end{align*}

So the two numbers are 15 and 68.

\vspace{.5 cm}

The hard part is getting the two equations.  Fortunately, there is another approach if you can't figure out the
equations.  If you try a few numbers, you can find the solution using trial and error.  For example, you might try:
\begin{itemize}
  \item 70 and 13 ($70/13$ is 6 remainder 2 so 70 is too big)
  \item 60 and 23 ($60/23$ is 2 remainder 14 so 60 is too small)
  \item 65 and 18 ($65/18$ is 3 remainder 11 so 65 is still a little too small)
  \item 68 and 15 ($68/15$ is 4 remainder 8 so this is the correct solution)
\end{itemize}

However, you can't just write $x = 15$ and $y = 68$ on your exam, because you are supposed to do some algebra to get the correct
solution.  However, knowing the correct solution, you may be able to work backwards to the equations.  What you want is
an equation which contains 68, 15, 4, and 8.  $68 = 4 \cdot 15 + 8$ works and makes sense with the problem statement.
Since you know $y=68$ and $x=15$, you can turn this into $y = 4x + 8$ and solve the system of equations:

\begin{align*}
  x+y &= 83 \\
  y &= 4x + 8 \\
\end{align*}

You should be able solve this system of equations pretty easily, especially when you know the solution in advance.

\end{solution}


\end{questions}
\end{document}


