\documentclass[fleqn, onecolumn]{article}

\usepackage{fullpage}
\usepackage{graphicx}
\usepackage{float}
\usepackage{amsmath}
\usepackage{amssymb}
\usepackage{polynom}
\usepackage{caption}
\usepackage{mdwlist}
\usepackage{parskip}

\newcommand{\degree}{\ensuremath{^\circ}} 

\everymath{\displaystyle}
\setlength{\mathindent}{1 cm}

\author{Ed Tellman}
\title{Math 142---Precalculus II}
\date{\today}

\begin{document}

  \maketitle

  \section{Overview}
  This course will cover chapters 5-7 of {\em Precalculus}, by Stewart, Redlin, and Watson.  

  \section{Homework and Exams}

  You should expect to spend three or four hours each week doing homework.  Feel free to work together with other students
  on the homework.

  Each chapter will be followed by an in-class test.  

  \section{Credit}
  Anyone who completes the course and is interesting receiving credit can take the final from Edmonds Community Collete.

  \section{Topics}

  \subsection{Chapter 5---Trigonometric Functions of Real Numbers}

  The simplest way to define trigonometric functions is as points on a circle with a radius of one.

  Topics include:
  \begin{itemize*}
    \item Measuring angles in radians
    \item Trigonometric functions (sine, cosine, tangent, etc.)
    \item Graphing 
    \item Applications 
  \end{itemize*}

  \subsection{Chapter 6---Trigonometric Functions of Angles}

  Another way to define trigonometric functions as as ratios of the lengths of sides of right triangles.

  Topics include:
  \begin{itemize*}
    \item Trigonometric functions of angles
    \item Law of Sines
    \item Law of Cosines
  \end{itemize*}

  \subsection{Chapter 7---Analytic Trigonometry}

  There are numerous formulas for simplifying and transforming trigonometric functions which allow you to simplify
  trigonometric equations.

  Topics include:
  \begin{itemize*}
    \item Addition and subtraction formulas
    \item Double-angle, half-angle, and sum-product formulas
    \item Inverse trigonometric functions
    \item Trigonometric equations
  \end{itemize*}

\end{document}

