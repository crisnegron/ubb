\documentclass[fleqn,addpoints]{exam}
\usepackage{amsmath}
\usepackage{graphicx}

\title{Math 113 Chapter Two Exam}
\author{}
\date{\today}

% \oddsidemargin 0in
% \topmargin -0.5in
% \textwidth 6.5in

% \printanswers

\ifprintanswers
\usepackage{2in1, lscape}
\fi

\extrawidth{-1 in}
\setlength{\mathindent}{0in}

\begin{document}

\maketitle

\ifprintanswers
\section{Strategies}

Getting good at taking tests takes practice, just like anything else.  Taking a test is a little different from doing
homework or solving real world problems mostly because of the time constraint.  Here are some things you might want to
keep in mind.

Your goal is not necessarily to solve every problem.  The goal is to maximize the number of points you get.  So you
should spend your limited amount of time in the way which is mostly likely to produce the 
most points.

For example, suppose you have 15 minutes left when you get to problem 20.  You take a glance at it and have no idea how to solve it.  If you spend the entire fifteen minutes working on problem 20, the best possible result is five more
points.  And there is a decent chance that you will end up with less than 5 points, since the problem looks tough.

So, instead, you might remember that you didn't check your work yet on problems 1-19.  Since you were working quickly and
trying to get everything done, you probably made a few mistakes.  Going back and plugging your answers back into the
original equations will probably take a minute or two per problem.  If you find 3 or 4 addition mistakes, signs swapped,
etc., you'll get 6 or 7 more points.  So in this case, checking your work on the earlier problems is probably a better
use of your time.

The strategy I use is to try to check a result immediately after I do it, especially if the result is an integer and the
original equation isn't too complicated.  In this case, you can often check the result in 30 seconds or less, and find a
mistake right away while you are still thinking about that problem.  If I'm pressed for time and checking the result
looks like it will take a bit longer, I'll move on and come back to check that result if I have time.

Basically, you want to make sure that you get full credit for all of the problems you actually know how to solve, and
don't lose points because of simple adding mistakes, etc.

Another thing to keep in mind is that I mentioned that calculators are not required and wouldn't be particularly
useful.  If you find yourself doing some complicated long division calculation, don't think to yourself: ``Wow, Ed's idea
of when a calculator is useful is way different from my idea of when a calculator would be useful.''  Instead, think to
yourself: ``I bet I messed something up in the steps leading up to this calculation.''

When you are working on a problem which starts out with a bunch of elaborate fractions and you get an integer solution,
you can usually feel a little more hopeful that your answer is correct.  It doesn't necessarily mean that your answer is
correct, of course, and the actual answer may be a fraction.  But often problems are constructed to have simple answers,
and it is a bit unlikely that you'll get a simple answer from a complicated start just by chance.

For instance, if I do a bunch of calculations and end up with something like 101/50 my immediate reaction is usually:
``I bet this problem was supposed to come out to 2,'' since 101/50 is almost exactly 2.  I'll view 101/50 with suspicion and go
back and double check all my calculations, looking for a place where I might have added in an extra 1.  Of course,
101/50 may be the correct answer, but it is just a warning sign that time spent double-checking that particular result
may be well spent and may result in a few more points.
 
\else
\vspace{0.2in}
\makebox[\textwidth]{Name:\enspace\hrulefill}
\vspace{0.2in}

\begin{center}
\gradetable[h][pages]
\end{center}

\section{Suggestions}

\begin{itemize}
  \item Check your answer by plugging your result back into the original equation (or original problem statement,
    for word problems).  We all make simple arithmetic mistakes, but you can usually catch them and fix them if you plug
    your answer back in to the original equation.
  \item Don't try to do too many steps at once in your head.  It's easy to make a mistake if you are juggling several
    numbers in your head at once.
  \item Show your work so I can give you partial credit if you went amiss somewhere along the line.
  \item If you need extra paper, let me know.
\end{itemize}

\fi

\section{Questions}

\subsection{Equalities}

For problems \ref{e:first}-\ref{e:last}, solve each equation.

\begin{questions}

\question[5]
\label{e:first}
\[ 4x - 1 = 2x + 7 \]
\begin{solution}[4 cm]
\begin{eqnarray*}
  4x - 1 &=& 2x + 7 \\
  2x - 1 &=& 7 \\
  2x     &=& 8 \\
  x      &=& 4 \\
\end{eqnarray*}
\end{solution}

\question[5]
\label{e:first}
\[ -2(3n - 1) + 3(n + 2) = -4(n - 3) \]
\begin{solution}[4 cm]
\begin{eqnarray*}
  -2(3n - 1) + 3(n + 2) &=& -4(n - 3) \\
  -6n + 2 + 3n + 6 &=& -4n + 12  \\
  -3n + 8 &=& -4n + 12  \\
  n + 8 &=& 12  \\
  n  &=& 4  \\
\end{eqnarray*}
\end{solution}

\question[5]
\label{e:first}
\[ 4(2a - 3) - 3(5a + 2) = 5(4a - 6) \]
\begin{solution}[4 cm]
\begin{eqnarray*}
  4(2a - 3) - 3(5a + 2) &=& 5(4a - 6) \\
  8a - 12 - 15a - 6 &=& 20a - 30 \\
  -7a - 18  &=& 20a - 30 \\
  -27a - 18 &=& -30 \\
  -27a      &=& -12 \\
  a         &=& \frac{12}{27} \\
  a         &=& \frac{4}{9} \\
\end{eqnarray*}
\end{solution}

\question[5]
\[ \frac{2}{5}(3x + 1) - \frac{1}{4}(2x - 2) = -\frac{3}{10} \]
\begin{solution}[4 cm]
\begin{eqnarray*}
  \frac{2}{5}(3x + 1) - \frac{1}{4}(2x - 2) &=& -\frac{3}{10} \\
  20 \left( \frac{2}{5}(3x + 1) - \frac{1}{4}(2x - 2) \right) &=& 20 \left( -\frac{3}{10} \right) \\
  8(3x + 1) - 5(2x - 2) &=& -6 \\
  24x + 8 - 10x + 10 &=& -6 \\
  14x + 18 &=& -6 \\
  14x &=& -24 \\
  x &=& -\frac{24}{14} \\
  x &=& -\frac{12}{7} \\
\end{eqnarray*}
\end{solution}

\question[5]
\[ \frac{1}{5}(5x + 10) + \frac{1}{2}(4x - 6) = 14 \]
\begin{solution}[4 cm]
\begin{eqnarray*}
  \frac{1}{5}(5x + 10) + \frac{1}{2}(4x - 6) &=& 14 \\
  x + 2 + 2x - 3 &=& 14 \\
  3x - 1 &=& 14 \\
  3x &=& 15 \\
  x  &=& 5 \\
\end{eqnarray*}
\end{solution}

\question[5]
\[ \frac{x - 2}{3} - \frac{x + 3}{4} = \frac{11}{6} \]
\begin{solution}[4 cm]
\begin{eqnarray*}
  \frac{x - 2}{3} - \frac{x + 3}{4} &=& \frac{11}{6} \\
  12 \left( \frac{x - 2}{3} - \frac{x + 3}{4} \right) &=& 12 \left( \frac{11}{6} \right) \\
  4(x - 2) - 3(x + 3) &=& 22 \\
  4x - 8 - 3x - 9 &=& 22 \\
  x - 17 &=& 22 \\
  x      &=& 39 \\
\end{eqnarray*}
\end{solution}

\question[5]
\[ \frac{2x + 3}{10} + \frac{x - 5}{6} - \frac{4x + 1}{15} = 1 \]
\begin{solution}[4 cm]
\begin{eqnarray*}
  \frac{2x + 3}{10} + \frac{x - 5}{6} - \frac{4x + 1}{15} &=& 1  \\
  30 \left( \frac{2x + 3}{10} + \frac{x - 5}{6} - \frac{4x + 1}{15} \right) &=& 30  \\
  3(2x + 3) + 5(x - 5) - 2(4x + 1) &=& 30  \\
  6x + 6 + 5x - 25 - 8x - 2 &=& 30  \\
  3x - 18 &=& 30  \\
  3x &=& 48  \\
  x &=& 16  \\  
\end{eqnarray*}

\end{solution}

\question[5]
\label{e:last}
\[ .1x + 0.12(x + 1000) =  560 \]
\begin{solution}[4 cm]
\begin{eqnarray*}
  .1x + 0.12(x + 1000) &=&  560 \\
  100(.1x + 0.12(x + 1000)) &=&  100 \cdot 560 \\
  10x + 12(x + 1000) &=&  56,000 \\
  10x + 12x + 12,000 &=&  56,000 \\
  22x + 12,000 &=&  56,000 \\
  22x &=&  44,000 \\
  x &=&  2,000 \\
\end{eqnarray*}
\end{solution}

\subsection{Formulas}

For problems \ref{formula:first}-\ref{formula:last}, solve each equation for $x$.

\question[5]
\label{formula:first}
\[ \frac{x}{3} + a = \frac{b}{2} \]
\begin{solution}[4 cm]
\begin{eqnarray*}
  \frac{x}{3} + a = \frac{b}{2} \\
  3 \left( \frac{x}{3} + a \right) = 3 \left( \frac{b}{2} \right) \\
  x + 3a = (\frac{3b}{2}) \\
  x = \frac{3b}{2} - 3a\\
\end{eqnarray*}

Some other ways to write the answer are: \( \displaystyle x = 3 \left(\frac{b}{2} - a \right) \) and 
\( \displaystyle x = \frac{3b - 6a}{2} \).

\end{solution}

\question[5]
\label{formula:last}
\[ a(x + b) = b(x - c) \]
\begin{solution}[4 cm]
\begin{eqnarray*}
  a(x + b) &=& b(x - c) \\
  ax + ab  &=& bx - bc  \\
  ax + ab - bx  &=& -bc  \\
  ax - bx  &=& -ab - bc  \\
  x(a - b)  &=& -ab - bc  \\
  x  &=& \frac{-ab - bc}{a - b}  \\
\end{eqnarray*}

Another way to write the solution is: 
\[ x = \frac{-ab - bc}{a - b} \cdot \frac{-1}{-1} = \frac{ab + bc}{b - a} \]

\end{solution}

\question

According to doctors, each night a young person should sleep 8 hours plus $1/4$ hour for each year that the person is
under 18 years old.  For example, a 14 year old should sleep 9 hours a night.

\begin{parts}
  \part[3] If age is $a$ and hours of sleep needed is $h$, write a formula expressing $h$ in terms of $a$.
    \begin{solution}[2 cm]
      \[ h = 8 + \frac{1}{4}(18 - a)\]
    \end{solution}

  \part[2]
    According to your formula, how many hours of sleep would a 4-year old need each night?

    \begin{solution}[2 cm]
      Plugging in \( a = 4 \):
      \[ h = 8 + \frac{1}{4}(18 - 4) = 8 + \frac{1}{4} \cdot 14 = \frac{23}{2}\]

      So a 4-year old should sleep 11 1/2 hours each night.
    \end{solution}

\end{parts}

\subsection{Inequalities}

For problems \ref{inequality:first}-\ref{inequality:last}, solve the inequalities and express the answer in interval form.

\question[5]
\label{inequality:first}
\( 2x + 1 < 3x \)
\begin{solution}[4 cm]
\begin{eqnarray*}
  2x + 1 &<& 3x \\
  1 &<& x \\
  x &>& 1 \\
\end{eqnarray*}
In interval form: \( (1, \infty) \).

\end{solution}

\question[5]
\[ \frac{x - 2}{3} - \frac{x + 1}{4} \leq \frac{3}{2} \]
\begin{solution}[4 cm]
\begin{eqnarray*}
  \frac{x - 2}{3} - \frac{x + 1}{4} &\leq& \frac{3}{2} \\
  12 \left( \frac{x - 2}{3} - \frac{x + 1}{4} \right) &\leq& 12 \left( \frac{3}{2} \right) \\
  4(x - 2) - 3(x + 1) &\leq& 18 \\
  4x - 8 - 3x - 3 &\leq& 18 \\
  x - 11 &\leq& 18 \\
  x &\leq& 29 \\
\end{eqnarray*}
In interval form: \( (-\infty, 29] \).

\end{solution}

\question[5]
\[ 2x + 1 < \frac{1}{3} \) or \( \displaystyle 3x - \frac{3}{2} > 1 \]
\begin{solution}[4 cm]
\begin{eqnarray*}
  2x + 1 < \frac{1}{3}  &or& 3x - \frac{3}{2} > 1 \\
  2x < -\frac{2}{3}  &or& 3x > \frac{5}{2} \\
  x < -\frac{1}{3}  &or& x > \frac{5}{6} \\
\end{eqnarray*}
In interval form: \( (-\infty, -1/3) \cup (5/6, \infty) \).

\end{solution}

\question[5]
\( \displaystyle \frac{-2x + 4}{3} \leq 2 \) and \( \displaystyle \frac{2x + 7}{9} < 3 \)
\begin{solution}[4 cm]
\begin{eqnarray*}
  \frac{-2x + 4}{3} \leq 2            &and& \frac{2x + 7}{9} \leq 3 \\
  3 \left( \frac{-2x + 4}{3} \right) \leq 3 \cdot 2 &and& 9 \left( \frac{2x + 7}{9} \right) \leq 9 \cdot 3 \\
  -2x + 4 \leq 6                      &and& 2x + 7 \leq 27 \\
  -2x \leq 2                          &and& 2x \leq 20 \\
  x \geq 1                            &and& x \leq 10 \\
\end{eqnarray*}
In interval form: \( [1, 10] \).
\end{solution}

\ifprintanswers
\else
\pagebreak
\fi

\question[5]
\[ \left| 2x - 3 \right| < 4 \]

\begin{solution}[4 cm]
\begin{eqnarray*}
  -(2x - 3) < 4 &and& 2x - 3 < 4 \\
  -2x + 3 < 4   &and& 2x < 7 \\
  -2x < 1       &and& x < 7/2 \\
  x > -1/2      &and& x < 7/2 \\
\end{eqnarray*}
In interval form: \( (-1/2, 7/2) \).
\end{solution}

\question[5]
\[ \left| \frac{x - 1}{5} \right| \geq 2 \]
\begin{solution}[4 cm]
\begin{eqnarray*}
  -\left( \frac{x - 1}{5} \right) \geq 2 &or& \frac{x - 1}{5} \geq 2 \\
  \frac{1 - x}{5} \geq 2                 &or& x - 1 \geq 10 \\
  1 - x \geq 10                          &or& x \geq 11 \\
  -x \geq 9                              &or& x \geq 11 \\
  x \leq -9                              &or& x \geq 11 \\
\end{eqnarray*}
In interval form: \( [-\infty, -9] \cup [11, \infty] \).

\end{solution}

\question[5] 
\label{inequality:last}

The UBB fund raiser is coming up in about a week.  Suppose the auction includes an equal number of UBB
mugs, artwork, and teddy bears and we would like the average price for these items to be at least \$50.  If UBB mugs
sell for \$15 and paintings sell for \$110, what would teddy bears have to sell for to reach the goal of \$50 per item
on average?  Express your answer in interval form.

\begin{solution}[4 cm]
Let $x$ be the price of teddy bears.

\begin{eqnarray*}
  \frac{15 + 110 + x}{3} &\geq& 50 \\
  \frac{125 + x}{3} &\geq& 50 \\
  125 + x &\geq& 150 \\
  x &\geq& 25 \\
\end{eqnarray*}

In interval form: \( [25, \infty) \).
\end{solution}

\pagebreak

\subsection{Word Problems}

For problems \ref{word:first}-\ref{word:last}, solve each problem by setting up and solving an appropriate algebraic equation.

\question[5]
\label{word:first}

Find three consecutive numbers such that three times the first plus two times the second is two more than four times
the third.

\begin{solution}[4 cm]
Let $x$ be the first number, $x + 1$ be the second number, and $x + 2$ be the third number.
\begin{eqnarray*}
  3x + 2(x + 1) &=& 2 + 4(x + 2) \\
  3x + 2x + 2 &=& 2 + 4x + 8 \\
  5x + 2 &=& 4x + 10 \\
  x + 2 &=& 10 \\
  x  &=& 8 \\
\end{eqnarray*}

The three numbers are 8, 9, and 10.  

To check, three times 8 is 24 and two times 9 is 18.  $24 + 18 = 42$ which is two more than four times 10.
\end{solution}

\question[5]
\label{word:last}
It's that magical time again when the Winter Olympics roll around and we all stare in bewilderment at our TVs and try
to figure out what the difference is between a bobsled and a luge and wonder why they don't liven up the curling a bit by allowing
the contestants to whack each other with the sticks.

Curiously, when the medal totals are printed in the paper, a bronze counts for just as much as a gold.  If I was running
the Olympics, I'd do things a little differently.  I'd count a gold medal as 4 points, a silver medal as 2 points, and a
bronze medal as 1 point.  For this question, we'll imagine that the Olympic committee decided to adopt my scoring
system.

Suppose the US and Canada are in heated competition over 10 skiing events.  In fact, the two countries are so dominant
that they take all the medals for these 10 events.  The countries are so evenly matched that they manage to
each get exactly the same number of total medals and the same number of bronze medals for these events.

Under the standard scoring system, they'd be tied, of course.  But under my scoring system, the US comes out 8 points
ahead.  How many gold and silver medals does each country take from the 10 events?

\begin{solution}[5 cm]
This problem is similar to the Super Bowl problem, with gold and silver medals instead of touchdowns and field goals.

The first thing to notice is that the number of gold medals the US wins is the same as the number of silver medals
Canada wins, since each event has one of each type of medal and the countries ended up with the same number of medals.
Similarly, the number of gold medals that Canada wins is the same as the number of silver medals that the US wins.

There are 10 events, so there are 10 gold medals, 10 silver medals, and 10 bronze medals.  It says in the problem
statement that each country got 5 bronze medals.  You can ignore the bronze medals, since each country got the same
number. 

If we let $x$ be the number of gold medals the US wins and the number
of silver medals Canada wins, the equation is:

\begin{eqnarray*}
  4x + 2(10 - x) &=& 4(10 - x) + 2x + 8\\
  4x + 20 - 2x   &=& 40 - 4x + 2x + 8\\
  2x + 20        &=& 48 - 2x \\
  4x + 20        &=& 48 \\
  4x             &=& 28 \\
  x              &=& 7 \\
\end{eqnarray*}

The US won 7 gold medals, 3 silver medals, and 5 bronze medals for \(28 + 6 + 5 = 39 \) points.  Canada won 3 gold
medals, 7 silver medals, and 5 bronze medals for \( 12 + 14 + 5 = 31 \) points.  The US wins by 8 points, so the
solution is correct.

If you can't come up with an equation right away, you might want to try some examples.  Experimenting with an example is
likely to lead you down the path towards the correct solution.

For example, you might say:  ``Suppose the US wins 6 events and Canada wins 4 events.''  You could then figure out the
score for that situation would be: \( us = 6 \cdot 4 + 4 \cdot 2 + 5 = 37 \) and 
\( ca = 4 \cdot 4 + 6 \cdot 2 + 5 = 33 \).

Looking at these numbers, the difference is only 4, so the 6 and 4 is not the correct answer.  But the equations for the
two countries look similar, so you might then realize that when the US wins $x$ golds, Canada wins $10 - x$ golds, which
might lead you to the correct equation.
 
\end{solution}

\pagebreak

\subsection{Extra Credit}

\noaddpoints

\question[5]
Two skaters, Jennie and Maude stood a mile apart on a frozen lake.  Then each skated directly to the spot where
the other had been standing.  With the help of a strong wind, Jennie performed the feat two and one-half times faster
than Maude and beat her by six minutes.  How long did it take each girl to skate the mile?

\begin{solution}[5 cm]
We can use $d = rt$ for this problem.  If we let Maude's rate be $r$ and Jennie's time be $t$, we know that Maude skated
the mile in $\frac{5}{2} rt$ and Jennie skated the mile in $r(t + 6)$.  Since they each skated a mile, these two equations must
be equal:
\begin{eqnarray*}
  \frac{5}{2} \cdot rt = r(t + 6) \\
  \frac{5t}{2} = t + 6 \\
  5t = 2t + 12 \\
  3t = 12 \\
  t = 4 \\
\end{eqnarray*}

So Jennie skated the mile in 4 minutes and Maude skated the mile in 10 minutes.  10 minutes is two and one-half times as
long as 4 minutes, so Jennie was two and one-half times faster and the solution is correct.

\end{solution}
\end{questions}

\end{document}


