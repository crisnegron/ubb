\documentclass{exam}

\usepackage{units} 
\usepackage{graphicx}
\usepackage[fleqn]{amsmath}
\usepackage{cancel}
\usepackage{float}
\usepackage{mdwlist}
\usepackage{booktabs}
\usepackage{cancel}
\usepackage{polynom}
\usepackage{caption}
\usepackage{fullpage}
\usepackage{comment}
\usepackage{enumerate}
\usepackage{xfrac}

\newcommand{\degree}{\ensuremath{^\circ}} 
\everymath{\displaystyle}

% \printanswers
\excludecomment{comment}

\ifprintanswers 
  \usepackage{2in1, lscape} 
\fi

\author{}
\date{October 9, 2013}
\title{Math 142 \\ Homework Six}

\begin{document}

  \maketitle

  A commuter is in the habit of arriving at his suburban station each evening exactly at 5:00.  His wife always meets
  the train and drives him home.  One day he takes an earlier train, arriving at the station at 4:00.  The weather is
  pleasant, so instead of telephoning home he starts walking along the route always taken by his wife.  They meet
  somewhere on the way.  He gets into the car and they drive home, arriving at their house ten minutes earlier than
  usual.  Assuming that the wife always drives at a constant speed, and that on this occasion she left just in time to
  meet the 5:00 train, can you determine how long the husband walked before he was picked up?

  \vspace{4 cm}
  \begin{quote}
    \begin{em}
      \begin{verse}
        The more prohibitions and rules, \\
        The poorer people become. \\
        The sharper people's weapons, \\
        The more they riot. \\
        The more skilled their techniques, \\
        The more grotesque their works. \\
        The more elaborate the laws, \\
        The more they commit crimes. \\
        \vspace{.2 cm}
        Therefore the Sage says: \\
        \vspace{.2 cm}
        I do nothing \\
        And people transform themselves. \\
        I enjoy serenity \\
        And people govern themselves \\
        I cultivate emptiness \\
        And people become prosperous. \\
        I have no desires \\
        And people simplify themselves. \\
      \end{verse}

    \end{em}
  \end{quote}
  \hspace{2 cm} --Lao Tzu
\end{document}

