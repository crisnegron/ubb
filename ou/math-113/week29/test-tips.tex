
\documentclass{article}
\usepackage{amsmath}
\usepackage{graphics}

\usepackage{2in1, lscape}

\title{Math 113 Test Notes}
\date{August 25, 2010}

\begin{document}

\maketitle

\section{Overview}
Here are some ideas of things you might want to keep in mind when taking the final.

\section{Check Your Work}

Even if you know how to do a problem, it's very easy to make a mistake with the sign, copying numbers from one step to
the next, adding numbers, etc.  The only way to catch this sort of thing is to check each answer.  Fortunately, checking
your work doesn't take much time, so you should always do it.

\subsection{Equations and Systems of Equations}
\begin{itemize}
  \item If you've solved an equation or system of equations, plug the results back in the equation
    to see if they actually turn the equations into true statements.
  \item If you've solved a quadratic equation and have a complicated solution like: $3 \pm \sqrt{17}$, check the answers
    using the ``sum of roots'' and ``product of roots'' techniques from {\em Chapter 6}.  The sum should be
    $-\frac{b}{a}$ and the product should be $\frac{c}{a}$.
  \item If you've solved a word problem, check the solution with the original problem statement to make sure it makes
    sense.
  \item If you've solved a problem by squaring both sides, you have to check the answer to make sure you haven't
    introduced any extraneous solutions.  Omit any solution that doesn't actually work.
\end{itemize}

\subsection{Lines}
\begin{itemize}
  \item If you've found an equation for a line using two points, plug both points into the equation so see if they work.
  \item If you've found an equation for a line using a point and slope, put the line in slope-intercept form to see if
    the slope is correct.  Also, plug the point in to see if it works.
  \item If you're trying to find perpendicular (or parallel) lines, make a quick graph to see if the lines look like
    they should.
\end{itemize}

\subsection{Other}
\begin{itemize}
  \item Every time you factor something, use FOIL to make sure when you multiply the factors together you get what you
    started with.
  \item Do the same problem in a different way and see if you get the same answer.
\end{itemize}

\section{Time Management}
\begin{itemize}
\item
Use all the time.  

I think you have about three hours to do about the same number of problems we've been doing in class
in about 2 hours.  There's no prize for finishing first.  If you get done with everything, go back and double check
your work, solve problems a different way, etc.  

\item
Don't spend too much time on one problem until you've looked at all the problems.  If you get stuck on a problem, skip
it for now and move on the next problem.  When you are done with the rest of the problems, don't forget to come
back to it and try again.

\item
Keep track of the time.  Don't be so leisurely that you run out of time before you've looked at all the problems.

\end{itemize}

\section{Common Mistakes}

\begin{itemize}
  \item In general, $(a+b)^c \ne a^c + b^c$.  For example:
    \begin{itemize}
      \item $(2^{-1} + 3^{-2})^2 \ne 2^{-2} + 3^{-4}$
      \item $(x^{-2} + y^{3})^{-2} \ne x^4 + y^{-6}$
    \end{itemize}
With this type of problem, you should generally first combine the terms in the parentheses and then apply the outer
exponent.

\item FOIL
  Whenever you need to multiply something that looks like: $(a+b)(c+d)$ you need to use FOIL.  For example:
  \begin{itemize}
    \item $(2 + 3\sqrt{2})(3 + \sqrt{5})$
    \item $(1 + 3i)(4 + 7i)$
  \end{itemize}

\end{itemize}

\section{Miscellaneous Suggestions}
\begin{itemize}

\item
Show your work.  Unlike the in-class tests, you get no credit for problems without work.  

Think of it from Ohio University's perspective.  They don't know you at all.  If they just see a correct answer, they
don't know if you got it from your programmable calculator, the book, etc.  So you should always write down enough so
that they can tell that you actually solved the problem yourself.

\item
It's an algebra test, not an arithmetic test.  The person that wrote the test probably wrote it so that most of the
answers are integers or simple fractions like $\frac{1}{2}$.  If you find that you are multiplying four digit numbers or working with
fractions like $\frac{131}{1,709}$, you may be on the right track.  But it is much more likely that you made an
arithmetic mistake earlier.  So if this happens, go back right away and double check your work to this point.

\item
Look for ways to make the numbers smaller.  If you find yourself working with an equation like:  $32x + 16y = 64$,
divide everything by $16$ to turn the equation into the much easier to work with: $2x + y = 4$

\item 
Memorize a few key formulas.  You should know:
\begin{itemize}
  \item the quadratic formula
  \item the formulas for the sum and difference of two cubes
  \item the formulas for checking the solutions for quadratic equation using the sum/product of roots
\end{itemize}

If you think you might forget them in the stress of the test, review them right before the test then write them down on
the back of the test or scratch paper immediately after the test is handed out.  Then you'll have one less thing to
worry about during the test.

\item
Try different approaches to the same problem.  For example, to solve a quadratic equation you have a choice of:
\begin{itemize}
  \item factoring
  \item completing the square
  \item quadratic formula
\end{itemize}

If you get stuck with one of the approaches, try a different one.



\end{itemize}
\end{document}
