
\documentclass[fleqn,addpoints]{exam}

\usepackage{siunitx}
\usepackage{graphicx}
\usepackage{float}
\usepackage{amsmath}
\usepackage{cancel}
\usepackage{polynom}
\usepackage{caption}
\usepackage{mdwlist}

\newcommand{\degree}{\ensuremath{^\circ}} 

\printanswers

\everymath{\displaystyle}

\ifprintanswers 
\usepackage{2in1, lscape} 
\fi

\title{Physics \\ Electromagnetism Exam}
\date{December 5, 2010}

\begin{document}

\maketitle

% \ifprintanswers
% \else
% \begin{center}
% \gradetable[h][pages]
% \bonusgradetable[h][pages]
% \end{center}
% \fi

\ifprintanswers
\else
\section{Notes}

\begin{itemize*}
  \item Feel free to use a calculator or borrow one from the UBB office.
  \item Use your textbook or notes for the required formulas.
  % \item If you get stuck on something, you can ask another student for assistance. 
  \item Some questions are marked ``bonus''.  These questions are a bit harder and are extra-credit.
  \item electron charge, proton mass, etc. are all on the front cover of the text book
  \item There's no class on Dec 12, so this is due on Dec 19, which will be the last day of class.
\end{itemize*}

\fi

\begin{questions}

\section{Electrostatics}

\question[5]
What is the magnitude of the electric field produced by a $5 \ \coulomb$ charge at a distance of $3 \ \meter$?
\begin{solution}
\[
  E = \frac{kq}{r^2} = \frac{(9 \cdot 10^9 \ \newton \cdot \meter^2 \per \coulomb^2)(5 \ \coulomb)}{(3 \ \meter)^2} 
    = 5 \cdot 10^9 \ \newton \per \coulomb
\]
\end{solution}
 
\question[5]
What is the force between an electron and a proton separated by $1 \cdot 10^{-5} \ \meter$?
\begin{solution}
\[
  F = \frac{kq_1q_2}{r^2} = \frac{(9 \cdot 10^9 \ \newton \cdot \meter^2 \per \coulomb^2)(1.6 \cdot 10^{-19} \ \coulomb)^2}{(2 \cdot 10^{-5} \ \meter)^2} 
    = 5.76 \cdot 10^{-19} \ \newton
\]
\end{solution}

\question[5]
An {\em ion} is an atom which is missing some electrons or has some extra electrons so that it has a net positive or
negative charge.  

When two identical ions are separated by a distance of $6.2 \cdot 10^{-10} \ \meter$, the electrostatic
force each exerts on the other is $5.4 \cdot 10^{-9} \ \newton$.  How many electrons are missing from each ion?

\begin{solution}
\begin{align*}
  F &= \frac{kq^2}{r^2} \\
  q &= r \sqrt{ \frac{F}{k} } \\
    &= (6.2 \cdot 10^{-10} \ \meter) \sqrt{ \frac{5.4 \cdot 10^{-9} \ \newton}{9 \cdot 10^9 \ \newton \cdot \meter^2 \per \coulomb^2} } \\
    &\approx 4.8 \cdot 10^{-19} \ \coulomb \\
\end{align*}

Each electron has a charge of $1.6 \cdot 10^{-19} \ \coulomb$, so the number of missing electrons is:
\[
  \frac{4.8 \cdot 10^{-19} \ \coulomb}{1.6 \cdot 10^{-19} \ \coulomb \per \text{ electron}} = 3 \text{ electron}
\]

\end{solution}

\question
A uniform electric field of magnitude $10 \ \volt \per \meter$ points in the positive $y$ direction.  What is the change
in electric potential of a $5 \ \coulomb$ charge as it moves from the origin to:
\begin{parts}
\part[3] $(0 \ \meter, 5 \ \meter)$
\begin{solution}
The charge is moving only in the $y$ direction, so the change in potential is the distance times the field:
\[
  V = Ed = (10 \ \volt \per \meter)(5 \ \meter) = 50 \ \volt
\]
\end{solution}

\part[3] $(5 \ \meter, 0 \ \meter)$
\begin{solution}
The charge is moving perpendicular to the field so its potential doesn't change.
\end{solution}

\part[3] $(5 \ \meter, 5 \ \meter)$
\begin{solution}
The only part of the movement that matters is the movement in the $y$ direction, so the result is the same as part a, or $50 \ \volt$.
\end{solution}

\end{parts}

\question[5]
A point charge of $q = -0.5 \ \coulomb$ is at the origin.  Where should you put a proton so that the force from the
charge at the origin exactly counteracts the proton's weight at the earth's surface ($g = 9.8 \ \meter\per\second^2$)?

\begin{solution}
We need the weight and the electrostatic force to match:
\begin{align*}
  mg &= \frac{kq_1q_2}{r^2} \\
  r  &= \sqrt{ \frac{kq_1q_2}{mg} } \\
     &= \sqrt{ \frac{ (9 \cdot 10^9 \ \newton \cdot \meter^2 \per \coulomb^2) (0.5 \ \coulomb)(1.6 \cdot 10^{-19})}
                    {(1.67 \cdot 10^{-27})(9.8 \cdot \meter \per \second^2)} } \\
     &= 2.1 \cdot 10^8 \ \meter \\
\end{align*}

Since the charge at the origin is negative, the proton should be placed below the origin so that the electrostatic force pulls the proton up.

\end{solution}

\bonusquestion[5]
A typical $12 \ \volt$ car battery can delivery $7.5 \cdot 10^5 \ \coulomb$ of charge.  If the energy supplied by the
battery could be converted to kinetic energy, what speed would it give a $1200 \ \kilogram$ car?

\begin{solution}
A $12 \ \volt$ battery is prepared to deliver $12 \ \joule$ of energy to each Coulomb of charge.  So this batter is
capable of of delivering:
\[
  (12 \ \volt)(7.5 \cdot 10^5 \ \coulomb) = 9 \cdot 10^6 \ \joule
\]

If all this energy goes into the car's kinetic energy, the velocity of the car is:
\begin{align*}
  K &= \frac{1}{2} mv^2 \\
  v &= \sqrt{\frac{2K}{m}} \\
    &= \sqrt{\frac{1.8 \cdot 10^7 \ \joule}{1200 \ \kilogram}} \\
    &= 122.5 \ \meter \per \second \\
\end{align*}

\end{solution}

\question[5] Two point charges of equal magnitude and opposite sign are $8 \ \centi \meter$ apart.  At the midpoint
of the line connecting them, their combined electric field has a magnitude of $45 \ \newton \per \coulomb$.  What is the
magnitude of the charges?

Each charge accounts for half the field.

\begin{solution}
\begin{align*}
  E &= \frac{kq}{r^2} \\
  q &= \frac{Er^2}{k} \\
    &= \frac{(22.5 \ \newton \per \coulomb)(0.04 \ \centi \meter)^2}
            {9 \cdot 10^9 \ \newton \cdot \meter^2 \per \coulomb^2} \\
    &= 4 \cdot 10^{-12} \ \coulomb \\
\end{align*}
\end{solution}

\bonusquestion Three point charges are placed on the vertices of an equilateral triangle with edges $5 \centi\meter$
long.  The charges are: $2 \ \micro\coulomb$ (top), $3 \ \micro\coulomb$ (bottom left), and $-1 \micro \coulomb$ (bottom
right).  What is the force on the $2 \ \micro\coulomb$ charge in the:

\begin{solution}
I find the equations easier to deal with with letters instead of actual numbers, so I'll use $q$ for one $\micro \coulomb$ of
charge and $r$ for the length of a side.  Then I can plug the numbers in at the end.

\end{solution}
\begin{parts}
\part[5] $x$ direction
\begin{solution}
In the $x$ direction the charge in the bottom right corner is pulling to the right and the charge in the bottom left
corner is pushing to the right.  The total force is:
\begin{align*}
  F_x &= \frac{2kq^2}{r^2} \sin 30 \degree + \frac{6kq^2}{r^2} \sin 30 \degree \\
      &= \frac{8kq^2}{r^2} \sin 30 \degree \\
      &= 14.4 \ \newton \text{ (to the right)} \\
\end{align*}

\end{solution}
\part[5] $y$ direction
\begin{solution}
In the $y$ direction the charge in the bottom right corner is pulling down and the charge in the bottom left
corner is pushing up.  The total force is:
\begin{align*}
  F_y &= - \frac{6kq^2}{r^2} \cos 30 \degree + \frac{2kq^2}{r^2} \sin 30 \degree \\
      &= -\frac{4kq^2}{r^2} \sin 30 \degree \\
      &= -12.5 \ \newton \text{ (down)}
\end{align*}
\end{solution}

\end{parts}

\section{Electric Current}
\question[5]
$4 \ \ohm$ and $6 \ \ohm$ resistors are connected in series to a $9 \ \volt$ battery.  How much current flows through the
circuit?
\begin{solution}
\[
  I = \frac{V}{R} = \frac{9 \ \volt}{10 \ \ohm} = 0.9 \ \ampere
\]
\end{solution}

\question[5]
An $2 \ \ohm$ resistor is connected in series with a $6 \ \ohm$ resistor.  There is a current of $0.5 \ \ampere$
through the two resistors.  What is the voltage of the battery to which they are connected?

\begin{solution}
  V = IR = (0.5 \ \ampere)(8 \ \ohm) = 4 \ \ampere
\end{solution}

\question[5]
A circuit containing two identical resistors connected in parallel has a current of $12 \ \ampere$ when connected to a
$6 \ \volt$ battery.  What is the resistance of each resistor?

\begin{solution}
The resistance of two identical resistors connected in parallel is one half the resistance of each.  So the current
through the circuit is:
\begin{align*}
  I &= \frac{V}{R/2} \\
    &= \frac{2V}{R} \\
  R &= \frac{2V}{I} \\
    &= \frac{12 \ \volt}{12 \ \ampere} \\
    &= 1 \ \ohm
\end{align*}

An alternate approach is to note that half the current goes through each resistor and each resistor has a voltage drop
of $6 \ \volt$.  For one resistor:
\[
  R = \frac{V}{I} = \frac{6 \ \volt}{6 \ \ampere} = 1 \ \ohm
\]

\end{solution}

\question
A $1 \ \ohm$ and a $2 \ \ohm$ resistor are connected to a battery.  Which resistor dissipates more power when:
\begin{parts}
\part[3] the resistors are connected in series?
\begin{solution}
When the resistors are in series, they both get the same current.  Since $P = I^2R$, the larger resistor dissapates more power.
\end{solution}

\part[3] the resistors are connected in parallel?
\begin{solution}
When the resistors are in parallel, they both get the same voltage.  Since $P = VI = \frac{V^2}{R}$, the smaller
resistor dissapates more power.
\end{solution}
\end{parts}

\question[5]
How much power is dissipated in a $25 \ \ohm$ electric heater connected to a $120 \ \volt$ outlet?
\begin{solution}
\[
  P = VI = V \cdot \frac{V}{R} = \frac{V^2}{R} = \frac{(120 \ \volt)^2}{25 \ \ohm} = 576 \ \watt
\]
\end{solution}

\section{Magnetism}
\question[5]
One proton moves with speed $v$ from Seattle to New York.  Another proton moves with the same speed from Seattle to San
Francisco.  Which proton experiences a larger magnetic force during its trip?

\begin{solution}
Since the earth's magnetic field runs N/S, the proton traveling from West to East is moving roughly perpendicular to the
earth's magnetic field while the other proton is moving roughly parallel to the earth's magnetic field.  Since a field
only affects charges moving perpendicular to the field, the proton going to New York experiences the larger force.
\end{solution}

\question[5]
An electron moving in the positive $x$ direction, at right angles to a magnetic field, experiences a magnetic force in
the positive $y$ direction.  What is the direction of the magnetic field?
\begin{solution}
Using the Right Hand Rule, in the positive $z$ direction.
\end{solution}

\question[5]
An electron moves at right angles to a magnetic field of $0.12 \ \tesla$.  What is its speed if the force exerted on it
is $9 \cdot 10^{-15} \ \newton$?
\begin{solution}

\begin{align*}
  F &= qvB \\
  v &= \frac{F}{qB} \\
    &= \frac{9 \cdot 10^{-15} \ \newton}{(1.6 \ \cdot 10^{-19} \ \coulomb)(0.12 \ \tesla)} \\
    &= 4.5 \cdot 10^5 \ \meter \per \second \\
\end{align*}

\end{solution}

\question[5]
What is the acceleration of a proton moving with a speed of $10 \ \meter \per \second$ moving at right angles to a
magnetic field of $2 \ \tesla$?
\begin{solution}

First find the force:
\[
  F = qvB = (1.6 \cdot 10^{-19})(10 \ \meter \per \second)(2 \ \tesla) = 3.2 \cdot 10^{-18} \ \newton
\]

Then use $F = ma$ to find the acceleration:
\[
  a = \frac{F}{m} = \frac{3.2 \cdot 10^{-18} \ \newton}{1.67 \cdot 10^{-27} \ \kilogram} 
    \approx 1.9 \cdot 10^9 \ \meter \per \second
\]
\end{solution}

\question[5]
What is the magnetic force exerted on a $2 \ \meter$ length of wire carrying a current of $1 \ \ampere$ perpendicular to
a magnetic field of $0.75 \ \tesla$?

\begin{solution}
\[
  F = IBl = (1 \ \ampere)(0.75 \ \tesla)(2 \ \meter) = 1.5 \ \newton
\]
\end{solution}

%% \question
%% A single circular loop of radius $0.25 \ \meter$ carries a current of $3 \ \ampere$ in a magnetic field of $1
%% \ \tesla$.  What is the maximum torque exerted on the loop?

\question[5]
A wire with a length of $3 \ \meter$ and a mass of $0.75 \ kg$ is in a magnetic field of $0.8 \ \tesla$.  What is the
minimum current needed to levitate the wire?

\begin{solution}
We need the wire's weight to match the force:
\begin{align*}
  mg &= IBl \\
  I  &= \frac{mg}{Bl} \\
     &= \frac{(0.75 \ \kg)(9.8 \ \meter \per \second)}{(0.8 \ \tesla)(3 \ \meter)} \\
     &\approx 3.06 \ \ampere \\
\end{align*}
\end{solution}

% \question[5]
% A coil with 375 turns and a radius of $5 \ \centi\meter$ has a magnetic flux through its core of $1.25
% \cdot 10^{-4} \ \tesla \cdot \meter^2$.  What is the current in the coil?

\question[5]
A bar magnet with its south pole pointing downward is falling toward the center of a horizontal conducting ring.  As
viewed from above, is the direction of the induced current in the ring clockwise or counterclockwise?

\begin{solution}
From the Right Hand Rule and Lenz's law, clockwise.
\end{solution}

\question[5]
A $0.25 \ \tesla$ magnetic field is perpendicular to a circular loop of wire with 50 turns and a radius of $15
\ \centi\meter$.  If the field is reduced to zero in $0.1 \ \second$, what is the magnitude of the induced emf?

\begin{solution}
\begin{align*}
  V &= \frac{\Delta \Phi}{\Delta t} \\
           &= \frac{- (0.25 \ \tesla)(50)(.15 \ \centi \meter)^2(\pi)}{0.1 \ \second} \\
           &\approx 0.707 \ \volt \\
\end{align*}
\end{solution}
 
\bonusquestion[5]
A single circular conducting loop has a radius $0.5 \ \meter$ and a resistance of $10 \ \ohm$.  Perpendicular
to the plane of the loop is a magnetic field of $0.25 \ \tesla$.  At what rate, in $\tesla \per \second$, must this
field change if the induced current in the loop is to be $0.2 \ \ampere$?

\begin{solution}
First we need to figure out what voltage is required:
\[
  V = IR = (0.2 \ \ampere)(10 \ \ohm) = 2 \ \volt
\]

Then we need to find the required change in flux:
\begin{align*}
  V &= \frac{\Delta \Phi}{\Delta t} \\
  V &= \frac{NA \Delta B}{\Delta t} \\
  V &= NA \cdot \frac{\Delta B}{\Delta t} \\
  \frac{\Delta B}{\Delta t} &= \frac{V}{NA} \\
  &= \frac{2 \ \volt}{(0.5 \ \meter)^2 \pi} \\
  &\approx 2.5 \ \tesla \per \second \\  
\end{align*}
\end{solution}

\end{questions}

\end{document}

