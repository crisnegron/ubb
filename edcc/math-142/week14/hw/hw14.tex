\documentclass{exam}

\usepackage{units} 
\usepackage{graphicx}
\usepackage[fleqn]{amsmath}
\usepackage{cancel}
\usepackage{float}
\usepackage{mdwlist}
\usepackage{booktabs}
\usepackage{cancel}
\usepackage{polynom}
\usepackage{caption}
\usepackage{fullpage}
\usepackage{comment}
\usepackage{enumerate}
\usepackage{xfrac}
\usepackage{parskip}

\newcommand{\dg}{\ensuremath{^\circ}} 
\everymath{\displaystyle}

\printanswers
\excludecomment{comment}

\ifprintanswers 
  \usepackage{2in1, lscape} 
\fi

\author{}
\date{December 31, 2014}
\title{Math 142 \\ Homework Fourteen}

\begin{document}

  \maketitle

  \section{Homework}
  Section 7.3:

  \section{Extra Credit}
  Section 7.3: 89

  \ifprintanswers
    Since the angles form a triangle, they add up to $180 \dg$:
    \begin{align*}
      A + B + C & = 180 \dg \\
      C         & = 180 \dg - (A + B) \\
      \\
      \cos C & = \cos (180 \dg - (A + B)) \\
             & = \cos 180 \dg \cos (A + B) + \sin 180 \dg \sin (A + B) \\
             & = - \cos (A + B) \\
             & = \sin A \sin B - \cos A \cos B \\
             \\
      \cos A &= \sin B \sin C - \cos B \cos C \\
      \cos B &= \sin A \sin C - \cos A \cos C \\
      \\
      \sin (A + B) & = \sin (180 \dg - C) \\
                   & = \sin C \\
      \\
      \cos (A + B) & = \cos (180 \dg - C) \\
                   & = - \cos C \\
    \end{align*}

    Some other useful facts are:
    \begin{align*}
      \sin (A + B) &= \sin A \cos B + \cos A \sin B \\
      \\
      \cos (A + B)  & = \cos A \cos B - \sin A \sin B \\
      \cos A \cos B & = \cos (A + B) + \sin A \sin B \\
    \end{align*}
    
    \pagebreak

    You can substitute the above expressions at the appropriate spots to get to the desired final expression:
    \begin{align*}
      \sin 2A & + \sin 2B + \sin 2C  = 2 \sin A \cos A + 2 \sin B \cos B + 2 \sin C \cos C \\
              & = 2 \sin A ( \sin B \sin C - \cos B \cos C ) \\
                & \qquad{} + 2 \sin B ( \sin A \sin C - \cos A \cos C ) \\
                & \qquad{} + 2 \sin C ( \sin A \sin B - \cos A \cos B ) \\
              & = 6 \sin A \sin B \sin C  - 2 (\sin A \cos B \cos C \\
                & \qquad{} + \cos A \sin B \cos C + \cos A \cos B \sin C) \\
              & = 6 \sin A \sin B \sin C  - 2 ( \cos C ( \sin A \cos B + \cos A \sin B ) \\
                & \qquad{} + \cos A \cos B \sin C) \\
              & = 6 \sin A \sin B \sin C  - 2 ( \cos C \sin (A + B) + \cos A \cos B \sin C) \\
              & = 6 \sin A \sin B \sin C  - 2 ( \cos C \sin (A + B) \\
                & \qquad{} + ( \cos (A + B) + \sin A \sin B ) \sin C) \\
              & = 6 \sin A \sin B \sin C \\
                & \qquad{} - 2 ( \cos C \sin C + ( - \cos C + \sin A \sin B ) \sin C) \\
              & = 6 \sin A \sin B \sin C \\
                & \qquad{} - 2 ( \cos C \sin C  - \cos C \sin C + \sin A \sin B \sin C) \\
              & = 6 \sin A \sin B \sin C  - 2 \sin A \sin B \sin C \\
              & = 4 \sin A \sin B \sin C \\
    \end{align*}
  \fi

  % \section{Review}

  \pagebreak

  \ifprintanswers
    \section{Section 7.3}
    \begin{description}

      \item[1] 
        \begin{tabular}[H]{ccc}
          \toprule
          $\sin x$           & $\cos x$           & $\tan x$      \\
          $\sfrac{5}{13}$    & $\sfrac{12}{13}$   & $\sfrac{5}{12}$ \\
          \midrule
          $\sin 2x$          & $\cos 2x$          & $\tan 2x$      \\
          $\sfrac{120}{169}$ & $\sfrac{119}{169}$ & $\sfrac{120}{119}$ \\
          \bottomrule
        \end{tabular}

      \item[2] 
        \begin{tabular}[H]{ccc}
          \toprule
          $\sin x$          & $\cos x$         & $\tan x$      \\
          $\sfrac{4}{5}$    & $- \sfrac{3}{5}$ & $-\sfrac{4}{3}$ \\
          \midrule
          $\sin 2x$         & $\cos 2x$        & $\tan 2x$      \\
          $-\sfrac{24}{25}$ & $-\sfrac{7}{25}$ & $-\sfrac{24}{7}$ \\
          \bottomrule
        \end{tabular}

      \item[9]
        \begin{align*}
          \sin^4 x & = \left( \sin^2 x \right)^2 \\
                   & = \left( \frac{1 - \cos 2x}{2} \right)^2 \\
                   & = \frac{1 - 2 \cos 2x + \cos^2 2x}{4} \\
                   &= \frac{1}{4} - \frac{\cos 2x}{2} + \frac{\cos^2 2x}{4} \\
                   &= \boxed{ \frac{1}{4} - \frac{\cos 2x}{2} + \frac{1 + \cos 4x}{8} } \\
        \end{align*}

      \item[10]
        \begin{align*}
          \cos^4 x & = \left( \cos^2 x \right)^2 \\
                   & = \left( \frac{1 + \cos 2x}{2} \right)^2 \\
                   & = \frac{1 + 2 \cos 2x + \cos^2 2x}{4} \\
                   &= \frac{1}{4} + \frac{\cos 2x}{2} + \frac{\cos^2 2x}{4} \\
                   &= \boxed{ \frac{1}{4} + \frac{\cos 2x}{2} + \frac{1 + \cos 4x}{8} } \\
        \end{align*}

      \item[11]
        \begin{align*}
          \cos^2 x & \sin^4 x = \cos^2 x \left( \sin^2 x \right)^2 \\
                   & = \left( \frac{1 + \cos 2x}{2} \right) \left( \frac{1 - \cos 2x}{2} \right) \left( \frac{1 - \cos 2x}{2} \right) \\
                   & = \left( \frac{1 - \cos^2 2x}{4} \right) \left( \frac{1 - \cos 2x}{2} \right) \\
                   & = \frac{1 - \cos^2 2x - \cos 2x + \cos^3 2x}{8} \\
                   & = \frac{1}{8} \left( 1 - \frac{1 + \cos 4x}{2} - \cos 2x + \cos 2x \left( \frac{1 + \cos 4x}{2} \right) \right) \\
                   & = \frac{1}{8} \left( 1 - \frac{1}{2} - \frac{\cos 4x}{2} - \cos 2x + \frac{\cos 2x}{2} + \frac{\cos 2x \cos 4x}{2} \right) \\
                   & = \frac{1}{8} \left( \frac{1}{2} - \frac{\cos 2x}{2} - \frac{\cos 4x}{2} + \frac{\cos 2x \cos 4x}{2} \right) \\
                   & = \boxed{ \frac{1}{16} ( 1 - \cos 2x - \cos 4x + \cos 2x \cos 4x ) } \\
        \end{align*}

      \item[12]
        \begin{align*}
          \cos^4 x & \sin^2 x = \left( \cos^2 x \right)^2 \sin^2 x \\
                   & = \left( \frac{1 + \cos 2x}{2} \right) \left( \frac{1 + \cos 2x}{2} \right) \left( \frac{1 - \cos 2x}{2} \right) \\
                   & = \left( \frac{1 + \cos 2x}{4} \right) \left( \frac{1 - \cos^2 2x}{2} \right) \\
                   & = \frac{1}{8} \left( 1 - \cos^2 2x + \cos 2x - \cos^3 2x \right) \\
                   & = \frac{1}{8} \left( 1 + \cos 2x - \frac{1 + \cos 4x}{2}  - \frac{\cos 2x}{2} \left( \frac{1 + \cos 4x}{2} \right) \right) \\
                   & = \frac{1}{8} \left( 1 + \cos 2x - \frac{1}{2}  - \frac{\cos 4x}{2} - \frac{\cos 2x}{2} - \frac{\cos 2x \cos 4x}{2} \right) \\
                   & = \frac{1}{8} \left( \frac{1}{2} + \frac{\cos 2x}{2} - \frac{\cos 4x}{2}  - \frac{\cos 2x \cos 4x}{2} \right) \\
                   & = \boxed{ \frac{1}{16} \left( 1 + \cos 2x - \cos 4x  - \cos 2x \cos 4x \right) } \\
        \end{align*}

      \item[13]
        \begin{align*}
          \cos^4 x & \sin^4 x = \left( \frac{1 - \cos 2x}{2} \right)^2 \left( \frac{1 + \cos 2x}{2} \right)^2 \\
                   & = \left( \frac{1 - \cos^2 2x}{4} \right)^2 \\
                   & = \left( \frac{1}{8} - \frac{\cos 4x}{8} \right)^2 \\
                   & = \frac{1}{64} \left( 1 - \cos 4x \right)^2 \\
                   & = \frac{1}{64} \left( 1 - 2 \cos 4x + \cos^2 4x \right) \\
                   & = \frac{1}{64} \left( 1 - 2 \cos 4x + \frac{1 + \cos 8x}{2} \right) \\
                   & = \boxed{ \frac{1}{128} \left( 3 - 4 \cos 4x + \cos 8x \right) } \\
        \end{align*}

      \item[15]
        \begin{align*}
          \sin 15 \dg & = \sin \frac{30 \dg}{2} \\
                      & = \sqrt{  \frac{1 - \sfrac{\sqrt{3}}{2}}{2} } \\
                      & = \boxed{ \frac{1}{2} \sqrt{2 - \sqrt{3}} } \\
        \end{align*}

      \item[16]
        \begin{align*}
          \tan 15 \dg & = \tan \frac{30 \dg}{2} \\
                      & = \frac{ 1 - \sfrac{\sqrt{3}}{2} }{ \sfrac{1}{2} } \\
                      & = \boxed{ 2 - \sqrt{3} } \\
        \end{align*}

      \item[17]
        \begin{align*}
          \tan 22.5 \dg & = \tan \frac{45 \dg}{2} \\
                        & = \frac{1 - \sfrac{\sqrt{2}}{2}}{\sfrac{\sqrt{2}}{2}} \\
                        & = \boxed{ \sqrt{2} - 1 } \\
        \end{align*}

      \item[18]
        \begin{align*}
          \sin 75 \dg & = \sin \frac{150 \dg}{2} \\
                      & = \sqrt{ \frac{1 - (\sfrac{\sqrt{3}}{2})}{2} } \\
                      & = \boxed{ \frac{1}{2} \sqrt{2 + \sqrt{3}} } \\
        \end{align*}

      \item[19]
        Since $165 \dg$ is in Q-II, $\cos 165 \dg < 0$
        \begin{align*}
          \cos 165 \dg & = - \cos \frac{330 \dg}{2}  \\
                       & = - \sqrt{ \frac{1 + \sfrac{\sqrt{3}}{2}}{2} } \\
                       & = \boxed{ - \frac{1}{2} \sqrt{ 2 + \sqrt{3} } } \\
        \end{align*}

      \item[23]
        \begin{align*}
          \cos \frac{\pi}{12} & = \cos \frac{\sfrac{\pi}{6}}{2} \\
                              & = \sqrt{ \frac{1 + \sfrac{\sqrt{3}}{2}} {2} } \\
                              & = \boxed{ \frac{1}{2} \sqrt{2 + \sqrt{3}} } \\
        \end{align*}

      \item[24]
        \begin{align*}
          \tan \frac{5 \pi}{12} & = \tan \frac{\sfrac{5 \pi}{6}}{2} \\
                                & = \frac{ 1 - \left( - \sfrac{\sqrt{3}}{2} \right) }{\sfrac{1}{2}} \\
                                & = \boxed{ 2 + \sqrt{3} } \\
        \end{align*}

      \item[25]
        Since $\sfrac{9 \pi}{8}$ is in Q-III, $\sin \sfrac{9 \pi}{8} < 0$.
        \begin{align*}
          \sin \frac{9 \pi}{8} & = \sin \frac{\sfrac{9 \pi}{4}}{2} \\
                               & = - \sqrt{ \frac{1 - \sfrac{1}{\sqrt{2}}}{2} } \\
                               & = \boxed{ - \frac{1}{2} \sqrt{ 2 - \sqrt{2} } } \\
        \end{align*}

      \item[26]
        Since $\sfrac{11 \pi}{12}$ is in Q-II, $\sin \sfrac{11 \pi}{12} > 0$.
        \begin{align*}
          \sin \frac{11 \pi}{12} & = \sin \frac{\sfrac{11 \pi}{6}}{2} \\
                                 & = \sqrt{ \frac{1 - \sfrac{\sqrt{3}}{\sqrt{2}}}{2} } \\
                                 & = \boxed{ \frac{1}{2} \sqrt{ 2 - \sqrt{3} } } \\
        \end{align*}

      \item[27]
        \begin{parts}
          \part 
            \[
              2 \sin 18 \dg \cos 18 \dg = \boxed{ \sin 36 \dg }
            \]

          \part 
            \[
              2 \sin 3 \theta \cos 3 \theta = \boxed{ \sin 6 \theta }
            \]
        \end{parts}

      \item[28]
        \begin{parts}
          \part 
            \[
              \frac{2 \tan 7 \dg}{1 - \tan^2 7 \dg} = \boxed{ \tan 14 \dg }
            \]

          \part 
            \[
              \frac{2 \tan 7 \theta}{1 - \tan^2 7 \theta} = \boxed{ \tan 14 \theta }
            \]
        \end{parts}

      \item[29]
        \begin{parts}
          \part 
            \[
              \cos^2 34 \dg - \sin^2 34 \dg = \boxed{ \cos 68 \dg }
            \]

          \part 
            \[
              \cos^2 5 \theta  - \sin^2 5 \theta = \boxed{ \cos 10 \theta }
            \]
        \end{parts}

      \item[30]
        \begin{parts}
          \part 
            \[
              \cos^2 \frac{\theta}{2}  - \sin^2 \frac{\theta}{2} = \boxed{ \cos \theta }
            \]

            \[
              2 \sin \frac{\theta}{2} \cos \frac{\theta}{2} = \boxed{ \sin \theta }
            \]
        \end{parts}

      \item[41]
        \[
          \sin 2x \cos 3x = \boxed{ \frac{1}{2} \left[ \sin 5x - \sin x \right] }
        \]

      \item[42]
        \[
          \sin x \sin 5x = \boxed{ \frac{1}{2} \left[ \cos 4x - \cos 6x \right] }
        \]

      \item[43]
        \[
          \sin 4x \cos x = \boxed{ \frac{1}{2} \left[ \sin 5x + \sin 3x \right] } 
        \]

      \item[44]
        \[
          \cos 5x \cos 3x = \boxed{ \frac{1}{2} \left[ \cos 8x + \cos 2x \right] } 
        \]

      \item[45]
        \[
          3 \cos 4x \cos 7x = \boxed{ \frac{3}{2} \left[ \cos 11x + \cos 3x \right] } 
        \]

      \item[46]
        \[
          11 \sin \frac{x}{2} \cos \frac{x}{4} = \boxed{ \frac{11}{2} \left[ \cos \frac{3x}{4} + \cos \frac{x}{4} \right] } 
        \]

      \item[47]
        \begin{align*}
          \sin 5x + \sin 3x & = 2 \sin \frac{5x + 3x}{2} \cos \frac{5x - 3x}{2} \\
                            & = \boxed{ 2 \sin 4x \cos x } \\
        \end{align*}

      \item[48]
        \begin{align*}
          \sin x - \sin 4x & = 2 \cos \frac{x + 4x}{2} \sin \frac{x - 4x}{2} \\
                           & = - 2 \cos \frac{5x}{2} \sin \left( -\frac{3x}{2} \right) \\
                           & = \boxed{ 2 \cos \frac{5x}{2} \sin \frac{3x}{2} } \\
        \end{align*}

      \item[49]
        \begin{align*}
          \cos 4x - \cos 6x & = - 2 \sin \frac{4x + 6x}{2} \sin \frac{4x - 6x}{2} \\
                            & = -2 \sin 5x \sin (-x) \\
                            & = \boxed{ 2 \sin 5x \sin x } \\
        \end{align*}

      \item[50]
        \begin{align*}
          \cos 9x + \cos 2x & = 2 \cos \frac{9x + 2x}{2} \cos \frac{9x - 2x}{2} \\
                            & = \boxed{ 2 \cos \frac{11x}{2} \cos \frac{7x}{2} } \\
        \end{align*}

      \item[61]
        \begin{align*}
          (\sin x + \cos x)^2 & = \sin^2 x + 2 \sin x \cos x + \cos^2 x \\
                              & = 1 + 2 \sin x \cos x \\
                              & = 1 + \sin 2x \\
        \end{align*}

      \item[62]
        \begin{align*}
          \frac{2 \tan x}{1 + \tan^2 x} & = \frac{2 \sfrac{\sin x}{\cos x}}{1 + \sfrac{\sin^2 x}{\cos^2 x}} \\
                                        & = \frac{2 \sin x \cos x}{\cos^2 x + \sin^x x} \\
                                        & = 2 \sin x \cos x \\
                                        & = \sin 2x \\
        \end{align*}

      \item[63]
        \begin{align*}
          \frac{\sin 4x}{\sin x} & = \frac{2 \sin 2x \cos 2x}{\sin x} \\
                                 & = \frac{4 \sin x \cos x \cos 2x}{\sin x} \\
                                 & = 4 \cos x \cos 2x \\
        \end{align*}

      \item[64]
        \begin{align*}
          \frac{1 + \sin 2x}{\sin 2x} & = \frac{1}{\sin 2x} + 1 \\
                                      & = 1 + \frac{1}{2 \sin x \cos x} \\
                                      & = 1 + \frac{1}{2} \sec x \csc x \\
        \end{align*}

      \item[70]
        \begin{align*}
          \cos \left( x + \frac{\pi}{2} \right) & = \cos x \cos \frac{\pi}{2} - \sin x \sin \frac{\pi}{2} \\
                                                & = - \sin x \\
          \\
          \tan^2 \left( \frac{x}{2} + \frac{\pi}{4} \right) 
            &= \frac{1 - \cos \left( x + \sfrac{\pi}{2} \right)}{1 + \cos \left( x + \sfrac{\pi}{2} \right)} \\
            &= \frac{1 - (- \sin x)}{1 + (- \sin x)} \\
            &= \frac{1 + \sin x}{1 - \sin x} \\
        \end{align*}

      \item[71]
        \begin{align*}
          \frac{\sin x + \sin 5x}{\cos x + \cos 5x} & = \frac{2 \sin (3x) \cos (-2x)}{2 \cos(3x) \cos (-2x)} \\
                                                    & = \frac{\sin 3x}{\cos 3x} \\
                                                    & = \tan 3x \\
        \end{align*}

      \item[72]
        \begin{align*}
          \frac{\sin 3x + \sin 7x}{\cos x + \cos 5x} & = \frac{2 \sin 3x \cos 2x}{2 \cos3x \cos (2x)} \\
                                                    & = \frac{\sin 3x}{\cos 3x} \\
                                                    & = \tan 3x \\
        \end{align*}

      \item[73]
        \begin{align*}
          \frac{\sin 10x}{\sin 9x + \sin x} & = \frac{2 \sin 5x \cos 5x}{2 \sin 5x \sin 4x} \\
                                            & = \frac{\cos 5x}{\sin 4x} \\
        \end{align*}

      \item[74]
        \begin{align*}
          \frac{\sin x + \sin 3x + \sin 5x}{\cos x + \cos 3x + \cos 5x} & = \frac{\sin 3x + (\sin x + \sin 5x)}{\cos 3x + (\cos x + \cos 5x)} \\
                                                                        & = \frac{\sin 3x + 2 \sin 3x \cos 2x}{\cos 3x + 2 \cos 3x \cos 2x} \\
                                                                        & = \frac{\sin 3x (1 + 2 \cos 2x)}{\cos 3x (1 + 2 \cos 2x)} \\
                                                                        & = \frac{\sin 3x}{\cos 3x} \\
                                                                        & = \tan 3x \\
        \end{align*}

      \item[75]
        \begin{align*}
          \frac{\sin x + \sin x}{\cos x + \cos x} & = \frac{2 \sin \sfrac{(x + y)}{2} \cos \sfrac{(x - y)}{2}}{2 \cos \sfrac{(x + y)}{2} \cos \sfrac{(x - y)}{2}} \\
                                                  & = \frac{\sin \sfrac{(x + y)}{2}}{\cos \sfrac{(x + y)}{2}} \\
                                                  & = \tan \frac{x + y}{2} \\
        \end{align*}

      \item[76]
        \begin{align*}
          & \frac{\sin (x + y) - \sin (x - y)}{\cos (x + y) - \cos (x - y)} \\
            & = \frac{2 \cos \left( \cfrac{x + y + x - y}{2} \right) \sin \left( \cfrac{x + y - (x - y)}{2} \right)}
                     {2 \cos \left( \cfrac{x + y + x - y}{2} \right) \cos \left( \cfrac{x + y - (x - y)}{2} \right)} \\
            & = \frac{2 \cos x \sin y}{2 \cos x \cos y} \\
            & = \tan y \\
        \end{align*}

    \end{description}

  \else
    \vspace{5 cm}

    \begin{quote}
      \begin{em}
      \end{em}
    \end{quote}
    \hspace{1 cm} --TO DO
  \fi

\end{document}

