\documentclass{exam}

\usepackage{units} 
\usepackage{graphicx}
\usepackage[fleqn]{amsmath}
\usepackage{cancel}
\usepackage{float}
\usepackage{mdwlist}
\usepackage{booktabs}
\usepackage{cancel}
\usepackage{polynom}
\usepackage{caption}
\usepackage{fullpage}
\usepackage{comment}
\usepackage{enumerate}
\usepackage{xfrac}

\newcommand{\dg}{\ensuremath{^\circ}} 
\everymath{\displaystyle}

% \begin{figure}[H]
%   \centering
%   \includgraphics[scale=.3]{question7.eps}
%   \caption*{question 7}
% \end{figure}

% \begin{tabular}{cc}
%   \toprule
%   period & amplitude \\
%     $\pi$ & $2$ \\
%   \bottomrule
% \end{tabular}

\printanswers
\excludecomment{comment}

\ifprintanswers 
  \usepackage{2in1, lscape} 
\fi

\author{}
\date{\today}
\title{Math 142 \\ Homework Seven}

\begin{document}

  \maketitle

  \section{Homework}
  Section 6.2: 

  \section{Extra Credit}
  Section 6.2: 43-44

  \ifprintanswers

    \section{Section 6.2}
    \begin{description}

      \item[1] 
        \begin{tabular}[H]{cccccc}
          \toprule
          $\sin$         & $\cos$         & $\tan$         & $\sec$         & $\csc$         & $\cot$ \\
          % \midrule
          $\sfrac{4}{5}$ & $\sfrac{3}{5}$ & $\sfrac{4}{3}$ & $\sfrac{5}{3}$ & $\sfrac{5}{4}$ & $\sfrac{3}{4}$ \\
          \bottomrule
        \end{tabular}

      \item[2] 
        \begin{tabular}[H]{cccccc}
          \toprule
          $\sin$          & $\cos$           & $\tan$          & $\sec$           & $\csc$          & $\cot$ \\
          % \midrule
          $\sfrac{7}{25}$ & $\sfrac{24}{25}$ & $\sfrac{7}{24}$ & $\sfrac{25}{24}$ & $\sfrac{25}{7}$ & $\sfrac{24}{7}$ \\
          \bottomrule
        \end{tabular}

      \item[3] 
        Find the missing side:
        \begin{align*}
          x^2 + 40^2 & = 41^2 \\
          x          & = 9 \\
        \end{align*}

        \begin{tabular}[H]{cccccc}
          \toprule
          $\sin$          & $\cos$           & $\tan$          & $\sec$           & $\csc$          & $\cot$ \\
          % \midrule
          $\sfrac{40}{41}$ & $\sfrac{9}{41}$ & $\sfrac{40}{9}$ & $\sfrac{41}{9}$ & $\sfrac{41}{40}$ & $\sfrac{9}{40}$ \\
          \bottomrule
        \end{tabular}

      \item[4] 
        Find the missing side:
        \begin{align*}
          15^2 + 8^2 & = r^2 \\
          r          & = 17 \\
        \end{align*}

        \begin{tabular}[H]{cccccc}
          \toprule
          $\sin$          & $\cos$           & $\tan$          & $\sec$           & $\csc$          & $\cot$ \\
          % \midrule
          $\sfrac{15}{17}$ & $\sfrac{8}{17}$ & $\sfrac{15}{8}$ & $\sfrac{17}{8}$ & $\sfrac{17}{15}$ & $\sfrac{8}{15}$ \\
          \bottomrule
        \end{tabular}

      \item[5] 
        Find the missing side:
        \begin{align*}
          3^2 + 2^2 & = r^2 \\
          r         & = \sqrt{13} \\
        \end{align*}

        \begin{tabular}[H]{cccccc}
          \toprule
          $\sin$          & $\cos$           & $\tan$          & $\sec$           & $\csc$          & $\cot$ \\
          % \midrule
          $\sfrac{2}{\sqrt{13}}$ & $\sfrac{3}{\sqrt{13}}$ & $\sfrac{2}{3}$ & $\sfrac{\sqrt{13}}{3}$ & $\sfrac{\sqrt{13}}{2}$ & $\sfrac{3}{2}$ \\
          \bottomrule
        \end{tabular}

      \item[6] 
        Find the missing side:
        \begin{align*}
          x^2 + 7^2 & = 8^2 \\
          x         & = \sqrt{15} \\
        \end{align*}

        \begin{tabular}[H]{cccccc}
          \toprule
          $\sin$          & $\cos$           & $\tan$          & $\sec$           & $\csc$          & $\cot$ \\
          % \midrule
          $\sfrac{7}{8}$ & $\sfrac{\sqrt{15}}{8}$ & $\sfrac{7}{\sqrt{15}}$ & $\sfrac{8}{\sqrt{15}}$ & $\sfrac{8}{7}$ & $\sfrac{\sqrt{15}}{7}$ \\
          \bottomrule
        \end{tabular}

      \item[7]
        Find the missing side:
        \begin{align*}
          3^2 + 5^2 & = r^2 \\
          x         & = \sqrt{34} \\
        \end{align*}

        \begin{align*}
          \sin \alpha & = \cos \beta = \frac{3}{\sqrt{34}} \\
          \tan \alpha & = \cot \beta = \frac{3}{5} \\
          \sec \alpha & = \csc \beta = \frac{\sqrt{34}}{5} \\
        \end{align*}

      \item[8]
        Find the missing side:
        \begin{align*}
          x^2 + 4^2 & = 7^2 \\
          x         & = \sqrt{33} \\
        \end{align*}

        \begin{align*}
          \sin \alpha & = \cos \beta = \frac{4}{7} \\
          \tan \alpha & = \cot \beta = \frac{4}{\sqrt{33}} \\
          \sec \alpha & = \csc \beta = \frac{7}{4} \\
        \end{align*}

      \item[9]
        \begin{align*}
          \sin 30 \dg & = \frac{x}{25} \\
          \frac{1}{2}     & = \frac{x}{25} \\
          x               & = \frac{25}{2} \\
        \end{align*}

      \item[10]
        \begin{align*}
          \sin 45 \dg    & = \frac{12}{x} \\
          \frac{\sqrt{2}}{2} & = \frac{12}{x} \\
          x                  & = \boxed{ 12 \sqrt{2} } \\
        \end{align*}

      \item[11]
        \begin{align*}
          \sin 60 \dg    & = \frac{x}{13} \\
          \frac{\sqrt{3}}{2} & = \frac{x}{13} \\
          x                  & = \boxed{ \frac{13 \sqrt{3}}{2} } \\
        \end{align*}

      \item[12]
        \begin{align*}
          \tan 30 \dg    & = \frac{4}{x} \\
          \frac{\sqrt{3}}{3} & = \frac{4}{x}  \\
          x                  & = \boxed{ 4 \sqrt{3} } \\
        \end{align*}

      \item[15]
        \begin{align*}
          \sin \theta & = \frac{y}{28} \\
          y           & = \boxed{ 28 \sin \theta } \\
          \\
          \cos \theta & = \frac{x}{28} \\
          x           & = \boxed{ 28 \cos \theta } \\
        \end{align*}

      \item[16]
        \begin{align*}
          \tan \theta & = \frac{x}{4} \\
          x           & = \boxed{ 4 \tan \theta } \\
          \\
          \sec \theta & = \frac{y}{4} \\
          y           & = \boxed{ 4 \sec \theta } \\
        \end{align*}

      \item[17] 
        Find the missing side:
        \begin{align*}
          x^2 + 3^2 & = 5^2 \\
          x         & = 4 \\
        \end{align*}

        \begin{tabular}[H]{cccccc}
          \toprule
          $\sin$          & $\cos$           & $\tan$          & $\sec$           & $\csc$          & $\cot$ \\
          $\sfrac{4}{5}$ & $\sfrac{3}{5}$ & $\sfrac{4}{3}$ & $\sfrac{5}{3}$ & $\sfrac{5}{4}$ & $\sfrac{3}{4}$ \\
          \bottomrule
        \end{tabular}

      \item[18] 
        Find the missing side:
        \begin{align*}
          9^2 + y^2 & = 40^2 \\
          x         & = 7 \sqrt{31} \\
        \end{align*}

        \begin{tabular}[H]{cccccc}
          \toprule
          $\sin$                    & $\cos$          & $\tan$                   & $\sec$          & $\csc$                    & $\cot$ \\
          $\sfrac{7 \sqrt{91}}{40}$ & $\sfrac{9}{40}$ & $\sfrac{7 \sqrt{91}}{9}$ & $\sfrac{40}{9}$ & $\sfrac{40}{7 \sqrt{91}}$ & $\sfrac{9}{7 \sqrt{91}}$ \\
          \bottomrule
        \end{tabular}

      \item[23]
        \begin{align*}
          \sin \frac{\pi}{6} + \cos \frac{\pi}{6} & = \frac{1}{2} + \frac{\sqrt{3}}{2} \\
                                                  & = \boxed{ \frac{1 + \sqrt{3}}{2} } \\
        \end{align*}

      \item[24]
        \begin{align*}
          \sin 30 \dg + \csc 30 \dg & = \frac{1}{2} + 2 \\
                                            & = \boxed{ \frac{5}{2} } \\
        \end{align*}

      \item[25]
        \begin{align*}
          \sin 30 \dg \cos 60 \dg + \sin 60 \dg \cos 30 \dg               & =
            \frac{1}{2} \times \frac{1}{2} + \frac{\sqrt{3}}{2} \times \frac{\sqrt{3}}{2} \\
                                                                                          & = \frac{1}{4} + \frac{3}{4} \\
                                                                                          &= \boxed{ 1 } \\
        \end{align*}

      \item[26]
        \begin{align*}
          ( \sin 60 \dg )^2 + ( \cos 60 \dg )^2 & = ( \frac{\sqrt{3}}{2} )^2 + ( \frac{1}{2} )^2 \\
                                                        & = \frac{3}{4} + \frac{1}{4} \\
                                                        & = \boxed{ 1 } \\
        \end{align*}

      % \item[27]
      %   \begin{align*}
      %     ( \cos 30 \dg )^2 - ( \sin 30 \dg )^2 & = ( \frac{\sqrt{3}}{2} )^2 - ( \frac{1}{2}  )^2 \\
      %                                                   & = \frac{3}{4} + \frac{1}{4} \\
      %                                                   & = \boxed{ \frac{1}{2} } \\
      %   \end{align*}
  
      % \item[28]
      %   \begin{align*}
      %     \left( \sin \frac{\pi}{3} \cos \frac{\pi}{4} - \sin \frac{\pi}{4} \cos \frac{\pi}{3} \right)^2 & = 
      %     \left( \frac{\sqrt{3}}{2} \frac{\sqrt{2}}{2} - \frac{\sqrt{2}}{2} \frac{1}{2} \right)^2 \\
      %                % & = ( \frac{\sqrt{6}}{4} - \frac{\sqrt{2}}{4} \right)^2 \\
      %                % & = ( \frac{\sqrt{6} - \sqrt{2}}{4} \right)^2 \\
      %                % & = ( \frac{6 - 2 \sqrt{12} + 4}{16} \right)^2 \\
      %   \end{align*}
  
      \item[29]
        \begin{align*}
          \sin 45 \dg & = \frac{16}{r} \\
          r           & = \boxed{ 16 \sqrt{2} } \\
          \\
          \cos 45 \dg & = \frac{x}{16 \sqrt{2}} \\
          x           & = \boxed{ 16 } \\
        \end{align*}

      \item[30]
        \begin{align*}
          \sin 75 \dg & = \frac{100}{r} \\
          r           & \approx \boxed{ 105.53 } \\
          \\
          \tan 75 \dg & = \frac{100}{x} \\
          x           & \approx \boxed{ 26.795 } \\
        \end{align*}

      \item[31]
        \begin{align*}
          \cos 52 \dg & = \frac{35}{r} \\
          r           & \approx \boxed{ 56.849 } \\
          \\
          \tan 52 \dg & = \frac{y}{35} \\
          r           & \approx \boxed{ 44.798 } \\
        \end{align*}

      \item[32]
        \begin{align*}
          \sin 68 \dg & = \frac{y}{1000} \\
          y           & \approx \boxed{ 927.18 } \\
          \\
          \cos 68 \dg & = \frac{x}{1000} \\
          x           & \approx \boxed{ 374.61 } \\
        \end{align*}

      \item[39]
        Divide $x$ into a left and right half and then add them up to get the total.

        \begin{align*}
          \tan 60 \dg & = \frac{100}{x_l} \\
          \sqrt{3}    & = \frac{100}{x_l} \\
          x_l         & = \frac{100 \sqrt{3}}{3} \\
          \\
          \tan 30 \dg        & = \frac{100}{x_r} \\
          \frac{1}{\sqrt{3}} & = \frac{100}{x_r} \\
          x_r                & = 100 \sqrt{3} \\
          \\
          x & = \frac{100 \sqrt{3}}{3} + 100 \sqrt{3} \\
            & = \frac{400 \sqrt{3}}{3} \\
            & \approx \boxed{ 230.9 }
        \end{align*}

      \item[40]
        \begin{align*}
          \tan 30 \dg & = \frac{100}{x_1} \\
          x_1         & = 85 \sqrt{3} \\
          \\
          \tan 60 \dg & = \frac{85}{x_2} \\
          x_2         & = \frac{85 \sqrt{3}}{3} \\
          \\
          x  & = x_1 - x_2 \\
             & = 85 \sqrt{3} - \frac{85 \sqrt{3}}{3} \\
             & = \frac{170 \sqrt{3}}{3} \\
             & \approx \boxed{ 98.1 } \\
        \end{align*}

      \item[41]
        \begin{align*}
          \sin 60 \dg & = \frac{50}{r} \\
          r           & = \frac{100}{\sqrt{3}} \\
          \\
          \sin 65 \dg & = \frac{ \sfrac{100}{\sqrt{3}} }{x} \\
          x          & \approx \boxed{ 63.7 } \\
        \end{align*}

      \item[42]
        \begin{align*}
          \sin 30 \dg & = \frac{5}{y} \\
          y           & = 10 \\
          \\
          \tan 30 \dg & = \frac{x}{10} \\
          x          & \approx \boxed{ 5.8 } \\
        \end{align*}

      \item[45]
        Using feet for the height:
        \begin{align*}
          \tan 11 \dg & = \frac{x}{5280} \\
          x           & \approx \boxed{ \unit[1026]{ft} } \\
        \end{align*}

      \item[46]
        \begin{enumerate}[(a)]
          \item 
            \begin{align*}
              \sin 22 \dg & = \frac{35,000}{d} \\
              d           & \approx \unit[93,431]{ft} \\
                          & \approx \boxed{ \unit[17.7]{mi} } \\
            \end{align*}

          \item 
            \begin{align*}
              \tan 22 \dg & = \frac{35,000}{d} \\
              d           & \approx \unit[86,628]{ft} \\
                          & \approx \boxed{ \unit[16.4]{mi} } \\
            \end{align*}

        \end{enumerate}

      \item[47]
        \begin{enumerate}[(a)]
          \item 
            \begin{align*}
              \tan 0.5 \dg & = \frac{d}{240,000} \\
              d           & \approx \unit[2094]{mi} \\
            \end{align*}

          \item 
            no---it misses by about 1000 miles
        \end{enumerate}

      \item[48]
        \begin{align*}
          \tan 23 \dg & = \frac{200}{d} \\
          d           & \approx \unit[471]{ft} \\
        \end{align*}

    \end{description}

  \else
    \vspace{1 cm}
    \begin{quote}
      \begin{em}
        TO DO
      \end{em}
    \end{quote}
    \hspace{1 cm} --Shunryu Suzuki
  \fi

\end{document}

