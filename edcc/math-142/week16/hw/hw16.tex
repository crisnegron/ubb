\documentclass{exam}

\usepackage{units} 
\usepackage{graphicx}
\usepackage[fleqn]{amsmath}
\usepackage{cancel}
\usepackage{float}
\usepackage{mdwlist}
\usepackage{booktabs}
\usepackage{cancel}
\usepackage{polynom}
\usepackage{caption}
\usepackage{fullpage}
\usepackage{comment}
\usepackage{enumerate}
\usepackage{xfrac}
\usepackage{parskip}

\newcommand{\dg}{\ensuremath{^\circ}} 
\everymath{\displaystyle}

\printanswers
\excludecomment{comment}

\ifprintanswers 
  \usepackage{2in1, lscape} 
\fi

\author{}
\date{\today}
\title{Math 142 \\ Homework Sixteen}

\begin{document}

  \maketitle

  \section{Homework}
  Section 7.5:

  \section{Extra Credit}
  Section 7.5: 

  \ifprintanswers
    \begin{description}
      \item[48]
        \begin{align*}
          \sin x &= \pm \frac{1}{\sqrt{1 + \sfrac{1}{\tan^2 x}}} \\
          \cos x &= \pm \frac{1}{\sqrt{1 + \tan^2 x}} \\
          \\
          \sin(x - y) &= \sin x \cos y - \cos x \sin y \\
                      &= \begin{cases}
                            - \sqrt{ \frac{1 - x^2}{1 + \sfrac{1}{x^2}}} + \frac{x}{\sqrt{1 + x^2}} & \mbox{if } -1 \leq x < 0 \\
                            0                                                                       & \mbox{if } x = 0 \\
                            \sqrt{ \frac{1 - x^2}{1 + \sfrac{1}{x^2}}} + \frac{x}{\sqrt{1 + x^2}}   & \mbox{if } 0 < x \leq 1 \\
                          \end{cases}
        \end{align*}

    \end{description}
  \fi

  \ifprintanswers
    \section{Section 7.5}
    \begin{description}

      \item[1] 
        \begin{align*}
          \cos x + 1 & = 0 \\
          \cos x     & = -1 \\
          x          & = \boxed{ \pi \pm 2 k \pi } \\
        \end{align*}

      \item[2] 
        \begin{align*}
          \sin x + 1 & = 0 \\
          \sin x     & = -1 \\
          x          & = \boxed{ \frac{3 \pi}{2} \pm 2 k \pi } \\
        \end{align*}

      \item[3] 
        \begin{align*}
          2 \sin x - 1 & = 0 \\
          \sin x       & = \frac{1}{2} \\
          x            & = \boxed{ \left\{ \frac{\pi}{6} \pm 2 k \pi, \frac{5 \pi}{6} \pm 2 k \pi \right\} } \\
        \end{align*}

      \item[4] 
        \begin{align*}
          \sqrt{2} \cos x - 1 & = 0 \\
          \cos x              & = \frac{\sqrt{2}}{2} \\
          x                   & = \boxed{ \left\{ \frac{\pi}{4} \pm 2 k \pi, \frac{7 \pi}{4} \pm 2 k \pi \right\} } \\
        \end{align*}

      \item[5] 
        \begin{align*}
          \sqrt{3} \tan x + 1 & = 0 \\
          \tan x              & = - \frac{\sqrt{3}}{3} \\
          x                   & = \boxed{ \frac{5 \pi}{6} \pm k \pi } \\
        \end{align*}

      \item[6] 
        \begin{align*}
          \cot x + 1 & = 0 \\
          \cot x     & = -1 \\
          x          & = \boxed{ \frac{3 \pi}{4} \pm k \pi } \\
        \end{align*}

      \item[7] 
        \begin{align*}
          4 \cos^2 x - 1 & = 0 \\
          \cos x         & = \pm \frac{1}{2} \\
          x              & = \boxed{ \left\{ \frac{\pi}{3} \pm k \pi, \frac{2 \pi}{3} \pm k \pi \right\} } \\
        \end{align*}

      \item[8] 
        \begin{align*}
          2 \cos^2 x - 1 & = 0 \\
          \cos x         & = \pm \frac{\sqrt{2}}{2} \\
          x              & = \boxed{ \left\{ \frac{\pi}{4} \pm k \pi, \frac{3 \pi}{4} \pm k \pi \right\} } \\
        \end{align*}

      \item[9] 
        \begin{align*}
          \sec^2 x - 2 & = 0 \\
          \sec x         & = \pm 2 \\
          \cos x         & = \pm \frac{\sqrt{2}}{2} \\
          x              & = \boxed{ \left\{ \frac{\pi}{4} \pm k \pi, \frac{3 \pi}{4} \pm k \pi \right\} } \\
        \end{align*}

      \item[10] 
        \begin{align*}
          \csc^2 x - 4 & = 0 \\
          \csc x         & = \pm 2 \\
          \sin x         & = \pm \frac{1}{2} \\
          x              & = \boxed{ \left\{ \frac{\pi}{6} \pm k \pi, \frac{5 \pi}{6} \pm k \pi \right\} } \\
        \end{align*}

      \item[11] 
        \begin{align*}
          3 \csc^2 x - 4 & = 0 \\
          \csc x         & = \pm \frac{2}{\sqrt{3}} \\
          \sin x         & = \pm \frac{\sqrt{3}}{2} \\
          x              & = \boxed{ \left\{ \frac{\pi}{3} \pm k \pi, \frac{2 \pi}{3} \pm k \pi \right\} } \\
        \end{align*}

      \item[12] 
        \begin{align*}
          1 - \tan^2 & = 0 \\
          \tan x         & = \pm 1 \\
          x              & = \boxed{ \left\{ \frac{\pi}{4} \pm k \pi, \frac{3 \pi}{4} \pm k \pi \right\} } \\
        \end{align*}

      \item[13] 
        \begin{align*}
          \cos x (2 \sin x + 1) & = 0 \\
          \\
          \cos x & = 0 \\
          x      & = \left\{ \frac{\pi}{2} \pm k \pi, \frac{3 \pi}{2} \pm k \pi \right\} \\
          \\
          2 \sin x + 1 & = 0 \\
          \sin x       & = - \frac{1}{2} \\
          x            & = \left\{ \frac{7 \pi}{6} \pm k \pi, \frac{13 \pi}{6} \pm k \pi \right\} \\
          \\
          x & = \boxed{ \left\{ \frac{\pi}{2} \pm k \pi, \frac{3 \pi}{2} \pm k \pi,
                        \frac{7 \pi}{6} \pm k \pi, \frac{13 \pi}{6} \pm k \pi \right\} } \\
        \end{align*}

      \item[14] 
        \begin{align*}
          \sec x (2 \cos x - \sqrt{2}) & = 0 \\
          \\
          2 \cos x - \sqrt{2} & = 0 \\
          \cos x              & = \frac{\sqrt{2}}{2} \\
          x                   & = \boxed{ \left\{ \frac{\pi}{4} \pm k \pi, \frac{7 \pi}{4} \pm k \pi \right\} } \\
        \end{align*}

    \end{description}

  \else
    \vspace{5 cm}

    \begin{quote}
      \begin{em}
      \end{em}
    \end{quote}
    \hspace{1 cm} --TO DO
  \fi

\end{document}

