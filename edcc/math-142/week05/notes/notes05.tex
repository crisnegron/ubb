\documentclass{exam}

\usepackage{units} 
\usepackage{graphicx}
\usepackage[fleqn]{amsmath}
\usepackage{cancel}
\usepackage{float}
\usepackage{mdwlist}
\usepackage{booktabs}
\usepackage{cancel}
\usepackage{polynom}
\usepackage{caption}
\usepackage{fullpage}
\usepackage{xfrac}
\usepackage{enumerate}

\newcommand{\degree}{\ensuremath{^\circ}} 
\everymath{\displaystyle}

\printanswers

\title{Math 142 Notes \\ Section 5.5}

\date{\today}

\begin{document}

  \maketitle
  \tableofcontents

  \pagebreak

  \section{Homework}
  \begin{itemize*}
    \item be careful about offset vs. amplitude
    \item problem 75 is period is 20 seconds, not 20 feet
    \item absolute value graphs should have sharper corners at x-axis
  \end{itemize*}

  \section{Simple Harmonic Motion}

  \subsection{Definitions}
    Draw picture of spring and graph of motion.
    \begin{align*}
      p      & = \unit[ \frac{2 \pi}{\omega} ]{s} \\
      f      & = \frac{1}{p} = \unit[ \frac{\omega}{2 \pi} ]{cycles/s} \\
      \\
      \omega & = \unit[ \frac{2 \pi}{p} ]{\frac{rad}{s}} \\
             & = \unit[ 2 \pi f ]{\frac{rad}{s}} \\
    \end{align*}

    Draw velocity and acceleration graphs below position graph.

  \subsection{Examples}
  \begin{enumerate}
    \item amplitude $\unit[5]{m}$, period $\unit[2]{s}$, displacement 0 at $t = 0$
    \item amplitude $\unit[10]{cm}$, frequency $\unit[100]{Hz}$, displacement maximum at $t = 0$ 
  \end{enumerate}

  \section{Masses on Springs}
  \subsection{Definitions}
  \[
    T = 2 \pi \sqrt{\frac{m}{k}}
  \]
          
  \subsection{Examples}
  \begin{enumerate}
    \item mass on a spring:
      \[
        y = 3 \cos \frac{2}{3} \pi t
      \]
      \begin{parts}
        \part what is the period?
        \begin{solution}
          \[
            p = \frac{2 \pi}{\sfrac{2}{3}} = \unit[3 \pi]{s}
          \]
        \end{solution}

        \part what is the first time the mass will be at position $y = 0$?
        \begin{solution}
          one fourth of the way through, or $\unit[\sfrac{3 \pi}{4}]{s}$
        \end{solution}

      \end{parts}

    \item mass oscillates on a spring with a period of 1.5 seconds and an amplitude of 6 cm.  
      If it starts at 6 cm at $t = 0$, what is the equation?

      \begin{solution}
        \begin{align*}
          \omega & = \frac{2 \pi}{1.5} = \frac{4}{3} \pi \\
          y(t)   & = 6 \cos \frac{4}{3} \pi t \\
        \end{align*}
      \end{solution}

    \item a mass on a spring has a period 8 and an amplitude of 10.  If it starts out moving up at position and time 0,
      where is it at times 1, 2, 4, and 6 seconds?

      \begin{solution}
        \begin{align*}
          \omega & = \frac{2 \pi}{8} \\
                 & = \frac{\pi}{4} \\
          \\
          y(t)   & = 10 \sin \frac{\pi}{4} t \\
        \end{align*}

        \begin{tabular}[H]{lr}
          \toprule
          time & position \\
          \midrule
          1 & $5 \sqrt{2}$ \\
          2 & $10$ \\
          4 & $0$ \\
          6 & $-10$ \\
          \bottomrule
        \end{tabular}
      \end{solution}

    \item A mass on a spring with period T.  In the time $\sfrac{3T}{2}$, the object moves a total distance of 12 cm.  
      What is the amplitude?

      \begin{solution}
        In one and a half cycles, the object will have traveled 6 times the amplitude.

        \begin{align*}
          12 & = 6 A \\
          A  & = \unit[2]{cm} \\
        \end{align*}
        
      \end{solution}
    \item A mass oscillates with a period of 3 seconds and starts from rest at a position $y = \unit[5]{cm}$.
      \begin{parts}
        \part where is it at time $t = \unit[7]{s}$?
          \begin{solution}
            \begin{align*}
              y(t) &= 5 \cos \frac{2 \pi}{3} t \\
              \\
              \frac{14 \pi}{3} = 4 \pi + \frac{2 \pi}{3} \\
              y(7) &= - \frac{5}{2} \\
            \end{align*}
          \end{solution}

        \part is it moving up or down at this time?
        \begin{solution}
          it hasn't yet reached the bottom, which happens half way between 6 and 9, so it's moving down
        \end{solution}
      \end{parts}

    \item car engine at 2,500 rpm.  What is the period and frequency?
      \begin{solution}
        \begin{align*}
          \omega &= 2500 \cdot 2 \pi \\
          f &= 2,500 \\
          p &= 
        \end{align*}
      \end{solution}
  \end{enumerate}

  \section{Pendulums}

  \subsection{Definitions}
  \[
    T = 2 \pi \sqrt{\frac{L}{g}}
  \]

  \subsection{Examples}
  \begin{enumerate}
    \item How long is a pendulum with a period of 2 seconds?
      \begin{solution}
        \begin{align*}
          2 & = 2 \pi \sqrt{\frac{L}{9.81}} \\
          L & \approx \unit[0.994]{m} \\
        \end{align*}
      \end{solution}

    \item Should you shorten or lengthen the pendulum on a clock that is running slow?
      \begin{solution}
        You want to decrease the period, so you need to decrease the length.
      \end{solution}

    \item Would a pendulum clock run fast or slow on the moon?
      \begin{solution}
        Since $g$ is lower on the moon, the period is longer and the clock runs slower than a similar clock on earth.
      \end{solution}

    \item There is a pendulum at the UN building with a 90 kg weight and 23 meter length.  If the maximum height of the
      swing is 2 meters, Find an equation for its motion.

      \begin{solution}
        The mass doesn't matter.

        \begin{align*}
          T      & = 2 \pi \sqrt{\frac{23}{9.8}} \\
                 & = \unit[9.62]{s} \\
            \\
          \omega & = \frac{2 \pi}{9.62} \\
                 & \approx 0.653 \\
            \\
          h(t)   & = 2 \cos 0.653 t \\
        \end{align*}
      \end{solution}
  \end{enumerate}

  \section{Damped Harmonic Motion}

  \subsection{Description}
  Real springs don't oscillate forever:
  \[
    y = k e^{-ct} \cos \omega t
  \]

  amplitude function (instead of amplitude constant):
  \[
    a(t) = k e^{-ct}
  \]

  \begin{description*}
    \item[k] initial amplitude
    \item[c] damping constant
  \end{description*}

  Draw graph of $e^{-ct}$, graph of sine, envelope

  \subsection{Examples}

  \begin{enumerate}
    \item A spring's oscillation amplitude has dropped from 8 cm to 4 cm after 5 seconds.  What is the damping constant?
      \begin{solution}
        \begin{align*}
          4       & = 8 e^{-5k} \\
          e^{-5k} & = \frac{1}{2} \\
          -5k     & = \ln \frac{1}{2} \\
          k       & = 0.1386 \\
        \end{align*}
      \end{solution}

    \item Spring with damping factor 0.1.
      \begin{enumerate}[(a)]
        \item What is the equation if spring is released 5 cm above equilibrium at $t = 0$ and the frequency is 2 cycles
          per second.

          \begin{solution}
            \[
              y(t) = 5 e^{-0.1 t} \cos 2 \pi t
            \]
          \end{solution}

        \item What is the amplitude after 2 seconds?
          \begin{solution}
            \[
              y(2) = 5 e^{-0.2} \approx \unit[4.09]{cm}
            \]
          \end{solution}

        \item when will the amplitude be one fifth the starting amplitude?
          \begin{align*}
            5 e^{-0.1 t} & = 1 \\
            e^{-0.1 t}   & = \frac{1}{5} \\
            -0.1 t       & = \ln \frac{1}{5}
            t            & \approx 16 \\
          \end{align*}
      \end{enumerate}
  \end{enumerate}

  \section{RMS}

  \subsection{General Formula}
  \[
    y_{rms} = \sqrt{\frac{1}{n} \cdot \left( y_0^2 + y_1^2 + \ldots + y_n^2 \right) }
  \]

  \subsection{Square Wave}
  \begin{align*}
    y_{rms} & = \sqrt{\frac{1}{n} \cdot \left( a^2 + a^2 + \ldots + a^2 \right) } \\
            & = \sqrt{\frac{1}{n} \cdot na^2 } \\
            & = a \\
  \end{align*}

  \subsection{Sawtooth Wave}
  \begin{align*}
    y_{rms} & = \sqrt{\frac{1}{n} \cdot \left[ \left( \frac{a}{n} \right)^2 + \left( \frac{2a}{n} \right)^2 
        + \ldots + \left( \frac{na}{n} \right)^2 \right ] } \\
            & = \sqrt{\frac{1}{n^3} \cdot \left[ a^2 ( 1 + 4 + 9 + \dots + n^2 ) \right] } \\
            & = \sqrt{\frac{a^2}{n^3} ( 1 + 4 + 9 + \dots + n^2 ) ] } \\
            & = \sqrt{\frac{a^2}{n^3} \cdot \frac{n(n + 1)(2n + 1)}{6} } \\
            & = \sqrt{\frac{a^2}{n^2} \cdot \frac{(n + 1)(2n + 1)}{6} } \\
            & = \sqrt{\frac{a^2}{n^2} \cdot \frac{2n^2 + 3n + 1}{6} } \\
            & = a \sqrt{ \frac{2n^2 + 3n + 1}{6n^2} } \\
            & \approx a \sqrt{ \frac{1}{3} } \\
            & = \frac{a}{\sqrt{3}} \\
  \end{align*}

  \subsection{Sine/Cosine Function}
  Show pairs which add up to 1 in sine/cosine pairs.

  \begin{align*}
    y_{rms} & = \sqrt{\frac{1}{n} \cdot \frac{na^2}{2} } \\
            & = \frac{a}{\sqrt{2}} \\
  \end{align*}

  Draw graph with all three.

  \section{Discrete Time Fourier Transform}

\end{document}

