% \documentclass[fleqn,addpoints,landscape]{exam}
\documentclass[fleqn,addpoints]{exam}
\usepackage{amsmath}
\usepackage{graphics}
\usepackage{mdwlist}

\everymath{\displaystyle}

\title{Math 113 Homework Seven}
\author{}
\date{\today}

% \printanswers

\ifprintanswers
\usepackage{2in1, lscape}
\fi

\begin{document}

\maketitle

\section{From the Book}


\begin{itemize*}
  \item Read pages 144-157.
  \item pp. 151-152: 1-5, 15-24, 51-55
  \item pp 157-158: 11-15, 48-55, 61, 68
\end{itemize*}

\section{Extra Credit}

An airplane flies in a straight line from airport A to airport B, then back in a straight line from B to A.  It travels
with a constant engine speed and there is no wind.  Will its travel time for the same round trip be greater, less, or
the same if, throughout both flights, at the same engine speed, a constant wind blows from A to B?  Explain.

\begin{solution}
The flight with the wind blowing takes longer.  There are several ways to look at it.

The plane is slowed down from A to B and is sped up from B to A.  The distance is the same in both cases, but the trip
from A to B takes longer, since the plane is going slower.  So the average speed for the round trip is longer because
the plane spends a longer time going slower than it spends going faster.

One idea for problems like this is to plug in some extreme numbers and see what happens.  This will sometimes give a
hint towards the solution.  Suppose the distance between the cities is 100 miles, the plane travels 100 mph without
wind, and the wind is blowing at 99 mph.  Without wind, the round trip takes 2 hours.  With the wind, the plane only
goes 1 mph from A to B, so the trip from A to B takes 100 hours.  On the way back, the plane goes 199 mph, so the return
trip takes slightly more than 1/2 hour.  So for this example, the total time for the trip with the wind would be about
100.5 hours.

More generally, let $r$ be the rate without the wind, $w$ be the wind speed, and $d$ be the distance between the two
cities.  

Using $t = \frac{d}{r}$:
\begin{itemize*}
\item Without wind, the trip takes: $t_{no wind} = \frac{2d}{r}$
\item With wind, the trip takes: $t_{wind} = \frac{d}{r - w} + \frac{d}{r + w}$
\end{itemize*}

As $w$ gets close to $r$, the first term gets close to $\infty$ and the second term gets close to $\frac{d}{2r}$.  Of
course, $w$ can never be greater than or equal to $r$ or the plane would never be able to get from A to B.

\end{solution}

\ifprintanswers

\section{Solutions}

\subsection{Pages 151-152}

\begin{description}

\item[1]
\( x^2 + 9x + 20 = (x + 5)(x + 4) \)

\item[2]
\( x^2 + 11x + 24 = (x + 8)(x + 3) \)

\item[3]
\( x^2 - 11x + 28 = (x - 7)(x - 4) \)

\item[4]
\( x^2 - 8x + 12 = (x - 6)(x - 2) \)

\item[5]
\( a^2 + 5a - 36 = (a + 9)(a - 4) \)

\item[15]
\( x^2 + 15xy + 36y^2 = (x + 12y)(x + 3y) \)

\item[16]
\( x^2 - 14xy + 40y^2 = (x - 10y)(x - 4y) \)

\item[17]
\( a^2 - ab - 56b^2 = (a + 7b)(a - 8b) \)

\item[18]
\( a^2 + 2ab - 63b^2 = (a + 9b)(a - 7b) \)

\item[19]
\begin{align*}
15x^2 + 23x + 6 &= 15x^2 + 5x + 18x + 6 \\
 &= 5x(3x + 1) + 6(3x + 1) \\
 &= (3x + 1)(5x + 6)  \\
\end{align*}

\item[20]
\begin{align*}
  9x^2 + 30x + 16 &= 9x^2 + 6x + 24x + 16 \\
  &= 3x(3x + 2) + 8x(3x + 2) \\
  &= (3x + 2)(3x + 8) \\
\end{align*}

\item[21]
\begin{align*}
  12x^2 - x - 6 &= 12x^2 - 9x + 8x - 6 \\
  &= 4x(3x + 2) - 3(3x + 2) \\
  &= (4x - 3)(3x + 2) \\
\end{align*}

\item[22]
\begin{align*}
  20x^2 - 11x - 3 &= 20x^2 - 15x + 4x - 3 \\
  &= 5x(4x - 3) + (4x - 3) \\
  &= (4x - 3)(5x + 1) \\  
\end{align*}

\item[23]
\begin{align*}
  4a^2 + 3a - 27 &= 4a^2 + 12a - 9a - 27 \\
  &= 4a(a + 3) - 9(a + 3) \\
  &= (a + 3)(4a - 9) \\
\end{align*}

\item[24]
\begin{align*}
  12a^2 + 4a - 5 &= 12a^2 - 6a + 10a - 5 \\
  &= 6a(2a - 1) + 5(2a - 1) \\
  &= (2a - 1)(6a + 5) \\  
\end{align*}


\item[51]
\begin{align*}
  x^4 - 9x^2 + 8 &= (x^2 - 1)(x^2 - 8) \\
  & = (x + 1)(x - 1)(x^2 - 8) 
\end{align*}

\item[52]
\begin{align*}
  x^4 - x^2 - 12 &= (x^2 + 3)(x^2 - 4) \\
  &= (x^2 + 3)(x + 2)(x - 2) \\
\end{align*}

\item[53]
\begin{align*}
  18n^4 + 25n^2 - 3 &= 18n^4 - 2n^2 + 27n^2 - 3 \\
  &= 2n^2(9n^2 - 1) + 3(9n^2 - 1) \\
  &= (9n^2 - 1)(2n^2 + 3) \\
  &= (3n + 1)(3n - 1)(2n^2 + 3) \\  
\end{align*}

\item[54]
\begin{align*}
  4n^4 + 3n^2 - 27 &= 4n^4 + 12n^2 - 9n^2 - 27 \\
  &= 4n^2(n^2 + 3) - 9(^2 + 3) \\
  &= (n^2 + 3)(4n^2 - 9) \\
  &= (n^2 + 3)(2n + 3)(2n - 3) \\
\end{align*}

\item[55]
\begin{align*}
  x^4 - 17x^2 + 16 &= (x^2 - 1)(x^2 - 16) \\
  &= (x + 1)(x - 1)(x + 4)(x - 4) \\
\end{align*}

\end{description}
\subsection{Pages 157-158}

\begin{description}

% \item[1]
% \( x^2 + 4x + 3 = (x + 3)(x + 1) \)

% \item[2]
% \( x^2 + 7x + 10 = (x + 2)(x + 5) \)

% \item[3]
% \( x^2 + 18x + 72 = (x + 6)(x + 12) \)

% \item[4]
% \( n^2 + 20n + 91 = (n + 7)(n + 13) \)

% \item[5]
% \( n^2 - 13n + 36 = (n - 9)(n - 4) \)

\item[11]
\begin{align*}
  n^2 + 25n + 156 &= (n + 12)(n + 13)
\end{align*}

$\{-12, -13\}$.

\item[12]
\begin{align*}
  n(n - 24) &= -128 \\
  n^2 - 24n + 128 &= 0 \\
  (n - 16)(n - 8)
\end{align*}

$\{16, 8\}$

\item[13]
\begin{eqnarray*}
  3t^2 + 14t - 5 &= & 0 \\
  (3t - 1)(t + 5) &=& 0 \\
  \\
  3t - 1 = 0 &or& t + 5 = 0 \\
  3t = 1 &or& t = -5 \\
  t = 1/3 &or& t = -5 \\
\end{eqnarray*}

\item[14]
\begin{eqnarray*}
  4t^2 - 19t - 30 &=& 0 \\
  4t^2 - 24t + 5t - 30 &=& 0 \\
  4t(t - 6) + 5(t - 6) &=& 0 \\
  (4t + 5)(t - 6) &=& 0 \\
  \\
  4t + 5 = 0 &or& t - 6 = 0 \\
  4t = -5 &or& t = 6 \\
  t = -5/4 &or& t = 6 \\
\end{eqnarray*}

\item[15]
\begin{eqnarray*}
  6x^2 + 25x + 14 &=& 0 \\
  (3x + 2)(2x + 7) &=& 0 \\
  \\
  3x + 2 = 0 &or& 2x + 7 = 0 \\
  3x  = -2 &or& 2x = -7 \\
  x  = -2/3 &or& x = -7/2 \\
\end{eqnarray*}

\item[48]
\begin{eqnarray*}
  x^3 + 5x^2 - 36x &=& 0 \\
  x(x^2 + 5x - 36) &=& 0 \\
  x(x + 9)(x - 4) &=& 0 \\
\end{eqnarray*}

$\{-9, 0, 4\}$

\item[49]
\begin{eqnarray*}
  12x^3 + 46x^2 + 40x &=& 0 \\
  2x(6x^2 + 23x + 20) &=& 0 \\
  2x(6x^2 + 23x + 20) &=& 0 \\
  2x(6x^2 + 15x + 8x + 20) &=& 0 \\
  2x( 3x(2x + 5) + 4(2x + 5)) &=& 0 \\
  2x(3x + 4)(2x + 5) &=& 0 \\
  x(3x + 4)(2x + 5) &=& 0 \\
\end{eqnarray*}

$\{0, -4/3, -5/2\}$

\item[50]
\begin{eqnarray*}
  5x(3x - 2) = 0 \\
  5x = 0 &or& 3x - 2 = 0 \\
  x = 0 &or& 3x  = 2 \\
  x = 0 &or& x  = 2/3 \\
\end{eqnarray*}

\item[51]
\begin{eqnarray*}
  (3x - 1)^2 - 16 &=& 0 \\
  (3x - 1 + 4)(3x - 1 - 4) &=& 0 \\
  (3x + 3)(3x - 5) &=& 0 \\
  \\
  3x + 3 = 0 &or& 3x - 5 = 0 \\
  x + 1 = 0 &or& 3x  = 5 \\
  x = -1 &or& x  = 5/3 \\
\end{eqnarray*}

\item[52]
\begin{eqnarray*}
  (x + 8)(x - 6) &=& -24 \\
  x^2 + 2x - 48 &=& -24 \\
  x^2 + 2x - 24 &=& 0 \\
  (x + 6)(x - 4) &=& 0 \\
\end{eqnarray*}

$\{-6, 4\}$

\item[53]
\begin{eqnarray*}
  4a(a + 1) &=& 3  \\
  4a^2 + 4a - 3 &=& 0 \\
  4a^2 + 6a - 2a - 3 &=& 0 \\
  2a(2a + 3) - (2a + 3) &=& 0 \\
  (2a - 1)(2a + 3) &=& 0 \\
  \\
  2a - 1 = 0 &or& 2a + 3 = 0 \\
  2a = 1 &or& 2a = -3 \\
  a = 1/2 &or& a = -3/2 \\
\end{eqnarray*}

\item[54]
\begin{eqnarray*}
  -18n^2 - 15n + 7 &=& 0 \\
  -1(-18n^2 - 15n + 7) &=& -1(0) \\
  18n^2 + 15n - 7 &=& 0 \\
  18n^2 -6n + 21n - 7 &=& 0 \\
  6n(3n - 1) + 7(3n - 1) &=& 0 \\
  (6n + 7)(3n - 1) &=& 0 \\
\end{eqnarray*}

$\{-7/6, 1/3\}$

\item[55]
\begin{eqnarray*}
  x(x + 1) &=& 72 \\
  x^2 + x - 72 &=& 0 \\
  (x + 9)(x - 8) &=& 0 \\
\end{eqnarray*}

The solution set is: $\{-9, 8\}$.  Since the question doesn't say the integers have to be positive, there are two pairs of integers that work: $\{-9, -8\}$ and $\{8, 9\}$.

\item[61]
\begin{eqnarray*}
  x^2 + (x + 1)^2 &=& 85 \\
  x^2 + x^2 + 2x + 1 &=& 85 \\
  2x^2 + 2x + 1 &=& 85 \\
  2x^2 + 2x - 84 &=& 0 \\
  x^2 + x - 42 &=& 0 \\
  (x + 7)(x - 6) &=& 0 \\
\end{eqnarray*}

The squares are 49 and 36 and $49 + 36 = 8$.  The two possible pairs are $\{-7, -6\}$ and $\{6, 7\}$.

\item[68]

The three numbers are $x$, $x+2$, and $x + 4$.  The hypotenuse is always the longest side, so it is $x + 4$.

\begin{eqnarray*}
  x^2 + (x + 2)^2 &=& (x + 4)^2 \\
  x^2 + x^2 + 4x + 4 &=& x^2 + 8x + 16 \\
  2x^2 + 4x + 4 &=& x^2 + 8x + 16 \\
  x^2 - 4x - 12 &=& 0 \\
  (x + 2)(x - 6) &=& 0 \\
\end{eqnarray*}

The solution set is: $\{-2, 6\}$.  $-2$ doesn't make sense as a length, so the sides are 6, 8, and 10.  Checking, 
$6^2 + 8^2 = 36 + 64 = 100 = 10^2$.

\end{description}


% \section{Review Problems}

% To prepare for the Chapter 3 exam, you should do some of the problems in the review problem set on pp. 161-162 and the
% sample test on p. 163.  You don't have to hand any of them in---just do them for practice and check the answers in the
% back of the book.

\else
\vspace{9 cm}
{\em I'm for truth, no matter who tells it. I'm for justice, no matter who it's for or against.}

\vspace{0.1 in}

\hspace{0.5 in} --Malcolm X

\fi

\end{document}
