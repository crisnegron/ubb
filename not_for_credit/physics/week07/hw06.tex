
\documentclass{exam}

\usepackage{graphicx}
\usepackage[fleqn]{amsmath}
\usepackage{cancel}
\usepackage{polynom}
\usepackage{float}
\usepackage{mdwlist}
\usepackage{booktabs}
\usepackage{cancel}
\usepackage{polynom}
\usepackage{caption}

\newcommand{\degree}{\ensuremath{^\circ}} 
\everymath{\displaystyle}

% \oddsidemargin .5in
% \topmargin -1in
\textwidth 6.5 in

\printanswers

\ifprintanswers 
\usepackage{2in1, lscape} 
\fi

\title{Physics \\ Homework Six \\ Rotational Dynamics}
\date{October 17, 2011}
% \author{Ed Tellman}

\begin{document}

\maketitle

\section{From the Book}

\begin{itemize*}
  \item Read Chapter 8, Section 2-5
  \item Chapter 8
    \begin{itemize*}
      \item questions 11, 13-14, 16, 20, 22-23, 26-27, 29, 31
      \item exercises 8-9, 11-13, 15-16, 18
      \item synthesis problems: 1, 3-4
    \end{itemize*}
\end{itemize*}

\section{Extra Credit}

A disk of mass {\em m} and radius {\em r} has a string wrapped around its circumference.  The upper end of the string is
held fixed, and the disk is allowed to fall.  Show that its linear acceleration is: $a = \dfrac{2}{3} g$.  ({\em James
  Walker})

\vspace{.1 in}

Some values of {\em I} you may (or may not) find useful are:

\vspace{.1 in}

\begin{tabular}{cc}
\toprule
disk with axis of rotation at center & $I = \dfrac{1}{2} mr^2$ \\
\midrule
disk with axis of rotation at edge & $I = \dfrac{3}{2} mr^2$. \\
\bottomrule
\end{tabular}

\begin{solution}

There are two ways to look at this problem.

One approach is to think of the disk as rolling around its center point.  It has the force of gravity pulling down on it
and the force of the string pulling up.  These are the only two forces, so the total force is:
\begin{align*}
  F_{total} &= mg - F_{string} \\
  ma &= mg - F_{string} \\
  F_{string} &= mg - ma \\
\end{align*}
Since $a = \alpha r$ this is equal to: $F_{string} = mg - m \alpha r$.

The force from the string applies the torque, so if you multiply both sides by $r$, you get the torque:
\[
  F_{string} r = \tau = mgr - m \alpha r^2
\]
The angular acceleration is $\alpha = \frac{\tau}{I}$ and we know $I = \dfrac{1}{2} mr^2$, so:
\begin{align*}
  \alpha &= \frac{mgr - m \alpha r^2 }{\left(\cfrac{1}{2} mr^2 \right)} \\
  \alpha &= \frac{2g - 2 \alpha r }{r} \\
  \alpha &= \frac{2g}{r} - 2 \alpha \\
  3 \alpha &= \frac{2g}{r} \\
  \alpha &= \frac{2}{3r} g \\  
  \alpha r &= \frac{2}{3} g \\
  a &= \frac{2}{3} g \\  
\end{align*}

The other approach requires a little less algebra.  You can also think of the torque as being applied from the
weight of the disk on the center of the disk.  In this case, the axis of rotation is the point where the disk meets the
string and you need to use the other value for $I$.
The torque is:
\[
  \tau = mgr
\]

Using the value for $I$ for rotation with the axis at the edge:
\begin{align*}
  \alpha &= \frac{\tau}{I} \\
  &= \frac{mgr}{\left( \cfrac{3}{2} mr^2 \right)} \\
  &= \frac{2}{3r} g \\  
\end{align*}

This is the same value we got before, and if you multiply it by $r$ you get the same acceleration.

\end{solution}

\ifprintanswers

\section{Chapter 8}

\subsection{Questions}

\begin{description}

\item[Q11]
$\mathbf{F_1}$. $\mathbf{F_1}$ is entirely tangential while some of $\mathbf{F_2}$ is not pointing in the correct direction to make the wheel rotate.

\item[Q13]
As the hint indicates if two equal and opposite forces are not along the same line, the object will start rotating instead of moving linearly.

\item[Q14]
To maximize torque, you want the length of rod between the fulcrum and your hands to be as long as possible.  This
requires putting the fulcrum as close to the rock as possible.

\item[Q16]
The plank will tip once it gets past the half way point.  After this point, the torque from gravity on the part of the plank off the
platform will exceed the torque on the other part of the plank and the plank will start to rotate.

\item[Q20]
The object with the mass closer to the axis will have a smaller moment of inertia.  Because:
\[
  \tau = \alpha I
\]

decreasing $I$ also decreases $\tau$.

\item[Q22]
Just like ice skaters do, you can change your rotational inertia by moving your arms in or out.

\item[Q23]
The hollow sphere has the larger rotational inertia because all of its mass is at a distance $r$ from the axis.  In a
sold sphere, most of the mass is less than $r$ from the center.

\item[Q26]
Moving weight from the center to the edge increases the rotational inertia of the system without changing the angular
momentum.  

Since:
\[
  L = I \omega
\]

when the rotational inertia increases, the angular velocity needs to decrease to compensate.

\item[Q27]
This also increases the merry-go-round's rotational inertia and decreases its angular velocity, for the same reason
mentioned in {\em Q26}.

\item[Q29]
As the string winds around your finger, $r$ decreases.  Sine the rotational inertia for a point mass is:
\[
  I = mr^2
\]
when $r$ goes down, $I$ also goes down.  In order to conserve angular momentum in this situation, the angular velocity
needs to increase. 

\item[Q31]
According to the right-hand rule, her velocity vector will be straight up out of the ice.

\end{description}

\subsection{Exercises}
\begin{description}

\item[E8]
\begin{description}
\item[a]
\[
  \tau = Fr = (50\ N)(.24\ m) = 12\ N \cdot m
\]

\item[b]
\[
  \tau = Fr = (50\ N)(.12\ m) = 6\ N \cdot m
\]

\end{description}

\item[E9]
\begin{align*}
  (30\ N) (10\ cm) &= (20\ N) x \\
  x &= 15\ cm \\
\end{align*}

\item[E11]
\begin{description}
\item[a]
\[
  \tau = (80\ N)(1.2\ m) = 96\ N \cdot m
\]

\item[b]
\[
  \tau = (50\ N)(1.2\ m) = 60\ N \cdot m
\]

\item[c]
\[
  \tau = 96 - 60\ N \cdot m = 36 \ N \cdot m
\]

\end{description}

\item[E12]
\[
  \alpha = \frac{\tau}{I} = \frac{60 \ N \cdot m}{12 \ kg \cdot m/s^2} = 5 \ rad/s^2
\]

\item[E13]
\[
  \tau = I \alpha = (4.5 \ kg \cdot m^2)(3.0 \ rad/s^2) = 13.5 \ N \cdot m
\]

\item[E15]
For a point mass, $I = mr^2$.  We have two point masses in this system:
\[
  I = 2 mr^2 = 2 \cdot (0.2 \ kg)(0.5 \ m)^2 = 0.1 \ kg \cdot m^2
\]

\item[E16]
\begin{description}
\item[a]
\[
  I = mr^2 = (0.8 \ kg)(0.5 \ m)^2 = 0.2 \ kg \cdot m^2
\]

\item[b]
\[
  L = I \omega = (0.2 \ kg \cdot m^2)(3 \ rad/s) = 0.6 \ kg \cdot m^2/s
\]
\end{description}

\item[E18]

\begin{align*}
  (4.5 \ kg \cdot m^2) (20 \ rpm) &= (1.5 \ kg \cdot m^2) \omega \\
  \omega &= 60 \ rpm \\
\end{align*}


\end{description}

\subsection{Synthesis Problems}

\begin{description}
\item[SP1]
\begin{description}

\item[a]
\[
  \tau_{net} = (80 \ N)(1.8 \ m) - 12 \ N \cdot m = 132 \ N \cdot m
\]

\item[b]
\[
  \alpha = \frac{\tau}{I} = \frac{132 \ N \cdot m}{900 \ kg \cdot m^2} \approx 0.15 \ rad/s^2
\]

\item[c]
\[
  \omega = \omega_0 + \alpha t = (0.15 \ rad/s^2)(15 \ s) = 2.2 \ rad/s
\]

\item[d]
When the child stops pushing, the only torque comes from the friction.  The acceleration is now:

\[
  \alpha = \frac{\tau}{I} = - \frac{12 \ N \cdot m}{900 \ kg \cdot m^2} = -0.0133 \ rad/s^2
\]

To find when it stops spinning:
\begin{align*}
  \omega &= \omega_0 + \alpha t \\
  0 &= 2.2 \ rad/s - (0.013 \ rad/s^2) t \\
  t &\approx 165 \ s
\end{align*}

\end{description}

\item[SP3]

\begin{description}
\item[a]
Since the children are like a point mass, their rotational inertia is
\[
  I_{children} = mr^2 = (240 \ kg)(2 \ m)^2 = 960 \ kg \cdot m^2
\]

The total rotational inertia is:
\[
  I_{total} = 960 + 1,500 \ kg \cdot m^2 = 2,460 \ kg \cdot m^2
\]

\item[b]
The new rotational inertia for the children is:
\[
  I_{children} = mr^2 = (240 \ kg)(0.5 \ m)^2 = 60 \ kg \cdot m^2
\]

The new total rotational inertia is:
\[
  I_{total} = 60 + 1,500 \ kg \cdot m^2 = 1,560 \ kg \cdot m^2
\]

\item[c]
The angular momentum is:
\[
  L = I \omega = (2,460 \ kg \cdot m^2)(1.2 \ rad/s) = 2,952 \ kg \cdot m^2/s
\]

After the children move to the center:
\begin{align*}
  2,952 \ kg \cdot m^2/s &= (1,560 \ kg \cdot m^2) \omega \\
  \omega &\approx 1.9 \ rad/s
\end{align*}

\item[d]

The torque comes from the children expending energy to move towards the center while the merry-go-round is rotating.

\end{description}

\item[SP4]

\begin{description}
\item[a]
All the initial angular momentum is in the wheel:

Converting the wheel's velocity to rad/s
\[
  \omega = 5 \ rev/s \cdot 2 \pi = 31.4 \ rad/s
\]

\[
  L = I \omega = (2 \ kg \cdot m^2)(31.4 \ rad/s) = 62.8 \ kg \cdot m^2/s
\]

\item[b]
The angular momentum doesn't change and the wheel is still spinning.  So now the stool must be spinning in the opposite direction.

% FIX ME
% \begin{align*}
%   - 62.8 \ kg \cdot m^2/s + (6 \ kg \cdot m^2) \cdot \omega &= 62.8 \ kg \cdot m^2/s \\
%   \omega &= -20.93 \ rad/s \\
% \end{align*}

\item[c]
Flipping the wheel over requires the student to exert a force on the axle, and this force provides the torque.

\end{description}

\end{description}

\fi


\vspace{2.5 in}

\ifprintanswers
\else
{\em I have nothing to say and I am saying it and that is poetry.}
\vspace{.2 cm}

\hspace{1 cm} --John Cage
\fi

\end{document}

