\documentclass{exam}

\usepackage{units} 
\usepackage{graphicx}
\usepackage[fleqn]{amsmath}
\usepackage{cancel}
\usepackage{float}
\usepackage{mdwlist}
\usepackage{booktabs}
\usepackage{cancel}
\usepackage{polynom}
\usepackage{caption}
\usepackage{fullpage}
\usepackage{comment}
\usepackage{enumerate}
\usepackage{xfrac}
\usepackage{parskip}

\newcommand{\dg}{\ensuremath{^\circ}} 
\everymath{\displaystyle}

\printanswers
\excludecomment{comment}

\ifprintanswers 
  \usepackage{2in1, lscape} 
\fi

\author{}
\date{\today}
\title{Math 142 \\ Homework Fifteen}

\begin{document}

  \maketitle

  \section{Homework}
  Section 7.4:

  \section{Extra Credit}
  Section 7.4: 89

  \ifprintanswers
    Since the angles form a triangle, they add up to $180 \dg$:
    \begin{align*}
      A + B + C & = 180 \dg \\
      C         & = 180 \dg - (A + B) \\
      \\
      \cos C & = \cos (180 \dg - (A + B)) \\
             & = \cos 180 \dg \cos (A + B) + \sin 180 \dg \sin (A + B) \\
             & = - \cos (A + B) \\
             & = \sin A \sin B - \cos A \cos B \\
             \\
      \cos A &= \sin B \sin C - \cos B \cos C \\
      \cos B &= \sin A \sin C - \cos A \cos C \\
      \\
      \sin (A + B) & = \sin (180 \dg - C) \\
                   & = \sin C \\
      \\
      \cos (A + B) & = \cos (180 \dg - C) \\
                   & = - \cos C \\
    \end{align*}

    Some other useful facts are:
    \begin{align*}
      \sin (A + B) &= \sin A \cos B + \cos A \sin B \\
      \\
      \cos (A + B)  & = \cos A \cos B - \sin A \sin B \\
      \cos A \cos B & = \cos (A + B) + \sin A \sin B \\
    \end{align*}
    
    \pagebreak

    You can substitute the above expressions at the appropriate spots to get to the desired final expression:
    \begin{align*}
      \sin 2A & + \sin 2B + \sin 2C  = 2 \sin A \cos A + 2 \sin B \cos B + 2 \sin C \cos C \\
              & = 2 \sin A ( \sin B \sin C - \cos B \cos C ) \\
                & \qquad{} + 2 \sin B ( \sin A \sin C - \cos A \cos C ) \\
                & \qquad{} + 2 \sin C ( \sin A \sin B - \cos A \cos B ) \\
              & = 6 \sin A \sin B \sin C  - 2 (\sin A \cos B \cos C \\
                & \qquad{} + \cos A \sin B \cos C + \cos A \cos B \sin C) \\
              & = 6 \sin A \sin B \sin C  - 2 ( \cos C ( \sin A \cos B + \cos A \sin B ) \\
                & \qquad{} + \cos A \cos B \sin C) \\
              & = 6 \sin A \sin B \sin C  - 2 ( \cos C \sin (A + B) + \cos A \cos B \sin C) \\
              & = 6 \sin A \sin B \sin C  - 2 ( \cos C \sin (A + B) \\
                & \qquad{} + ( \cos (A + B) + \sin A \sin B ) \sin C) \\
              & = 6 \sin A \sin B \sin C \\
                & \qquad{} - 2 ( \cos C \sin C + ( - \cos C + \sin A \sin B ) \sin C) \\
              & = 6 \sin A \sin B \sin C \\
                & \qquad{} - 2 ( \cos C \sin C  - \cos C \sin C + \sin A \sin B \sin C) \\
              & = 6 \sin A \sin B \sin C  - 2 \sin A \sin B \sin C \\
              & = 4 \sin A \sin B \sin C \\
    \end{align*}
  \fi

  % \section{Review}

  \pagebreak

  \ifprintanswers
    \section{Section 7.4}
    \begin{description}

      \item[1] 
        \begin{parts}
          \part $\sin^{-1} \frac{1}{2} = \boxed{ \frac{\pi}{6} }$
          \part $\cos^{-1} \frac{1}{2} = \boxed{ \frac{\pi}{3} }$
          \part $\cos^{-1} 2$ is not defined
        \end{parts}

      \item[2] 
        \begin{parts}
          \part $\sin^{-1} \frac{\sqrt{3}}{2} = \boxed{ \frac{\pi}{3} }$
          \part $\cos^{-1} \frac{\sqrt{3}}{2} = \boxed{ \frac{\pi}{6} }$
          \part $\cos^{-1} \frac{- \sqrt{3}}{2} = \boxed{ \frac{5 \pi}{6} }$
        \end{parts}

      \item[3] 
        \begin{parts}
          \part $\sin^{-1} \frac{\sqrt{2}}{2} = \boxed{ \frac{\pi}{4} }$
          \part $\cos^{-1} \frac{\sqrt{2}}{2} = \boxed{ \frac{\pi}{4} }$
          \part $\sin^{-1} \frac{- \sqrt{2}}{2} = \boxed{ - \frac{\pi}{4} }$
        \end{parts}

      \item[4] 
        \begin{parts}
          \part $\tan^{-1} \sqrt{3} = \boxed{ \frac{\pi}{3} }$
          \part $\tan^{-1} (- \sqrt{3}) = \boxed{ -\frac{\pi}{3} }$
          \part $\sin^{-1} \sqrt{3}$ is not defined
        \end{parts}

      \item[5] 
        \begin{parts}
          \part $\sin^{-1} 1 = \boxed{ \frac{\pi}{2} }$
          \part $\cos^{-1} 1 = \boxed{ 0 }$
          \part $\cos^{-1} (-1) = \boxed{ \pi }$
        \end{parts}

      \item[6] 
        \begin{parts}
          \part $\tan^{-1} 1 = \boxed{ \frac{\pi}{4} }$
          \part $\tan^{-1} (-1) = \boxed{ \frac{- \pi}{4} }$
          \part $\tan^{-1} 0 = \boxed{ 0 }$
        \end{parts}

      \item[7] 
        \begin{parts}
          \part $\tan^{-1} \frac{\sqrt{3}}{3} = \boxed{ \frac{\pi}{6} }$
          \part $\tan^{-1} \left( - \frac{\sqrt{3}}{3} \right) = \boxed{ - \frac{\pi}{4} }$
          \part $\sin^{-1} (-2)$ is not defined
        \end{parts}

      \item[8] 
        \begin{parts}
          \part $\sin^{-1} 0 = \boxed{ 0 }$
          \part $\cos^{-1} 0 = \boxed{ \frac{\pi}{2} }$
          \part $\cos^{-1} \left( - \frac{1}{2} \right) = \boxed{ \frac{2 \pi}{3} }$
        \end{parts}

      \item[13] 
        \[
          \sin \left( \sin^{-1} \frac{1}{4} \right) = \boxed{ \frac{1}{4} }
        \]

      \item[14] 
        \[
          \cos \left( \cos^{-1} \frac{2}{3} \right) = \boxed{ \frac{2}{3} }
        \]

      \item[15] 
        \[
          \tan \left( \tan^{-1} 5 \right) = \boxed{ 5 }
        \]

      \item[16] $\sin \left( \sin^{-1} 5 \right)$ is not defined

      \item[17] 
        \[
          \cos{-1} \left( \cos \frac{\pi}{3} \right) = \boxed{ \frac{\pi}{3} }
        \]

      \item[18] 
        \[
          \tan{-1} \left( \tan \frac{\pi}{6} \right) = \boxed{ \frac{\pi}{6} }
        \]

      \item[19] 
        \[
          \sin{-1} \left( \sin \left( -\frac{\pi}{6} \right) \right) = \boxed{ - \frac{\pi}{6} }
        \]

      \item[20] 
        \[
          \sin{-1} \left( \sin \left( \frac{5 \pi}{6} \right) \right) = \boxed{ \frac{\pi}{6} }
        \]

      \item[21] 
        \[
          \tan{-1} \left( \tan \left( \frac{2 \pi}{3} \right) \right) = \boxed{ - \frac{\pi}{3} }
        \]

      \item[22] 
        \[
          \cos{-1} \left( \cos \left( - \frac{\pi}{4} \right) \right) = \boxed{ \frac{\pi}{4} }
        \]

      \item[23] 
        \[
          \tan \left( \sin^{-1} \frac{1}{2} \right) = \boxed{ \frac{\sqrt{3}}{3} }
        \]

      \item[24] 
        \[
          \sin \left( \sin^{-1} 0 \right) = \boxed{ 0 }
        \]

      \item[25] 
        \[
          \cos \left( \sin^{-1} \frac{\sqrt{3}}{2} \right) = \boxed{ \frac{1}{2} }
        \]

      \item[26] 
        \[
          \tan \left( \sin^{-1} \frac{\sqrt{2}}{2} \right) = \boxed{ 1 }
        \]

      \item[27] 
        \begin{align*}
          \tan^{-1} \left( 2 \sin \frac{\pi}{3} \right) & = \tan^{-1} \sqrt{3} \\
                                                        & = \boxed{ \frac{\pi}{3} }
        \end{align*}

      \item[28] 
        \begin{align*}
          \cos^{-1} \left( \sqrt{3} \sin \frac{\pi}{6} \right) & = \cos^{-1} \frac{\sqrt{3}}{2} \\
                                                               & = \boxed{ \frac{\pi}{6} } \\
        \end{align*}

      \item[29]
        \[
          \sin \left( \cos^{-1} \frac{3}{5} \right) = \boxed{ \frac{4}{5} } 
        \]

      \item[30]
        \[
          \tan \left( \sin^{-1} \frac{4}{5} \right) = \boxed{ \frac{4}{3} } 
        \]

      \item[31]
        \[
          \sin \left( \tan^{-1} \frac{12}{5} \right) = \boxed{ \frac{12}{13} } 
        \]

      \item[32]
        \[
          \cos \left( \tan^{-1} 5 \right) = \boxed{ \frac{\sqrt{26}}{26} } 
        \]

      \item[33]
        \[
          \sec \left( \sin^{-1} \frac{12}{13} \right) = \boxed{ \frac{13}{5} } 
        \]

      \item[37]
        \begin{align*}
          \sin \left( 2 \cos^{-1} \frac{3}{5} \right) & = 2 \sin \left( \cos^{-1} \frac{3}{5} \right) \cos \left( \cos^{-1} \frac{3}{5} \right) \\
                                                      & = 2 \cdot \frac{4}{5} \cdot \frac{3}{5} \\
                                                      & = \boxed{ \frac{24}{25} } \\
        \end{align*}

      \item[38]
        \begin{align*}
          \tan \left( 2 \tan^{-1} \frac{5}{13} \right) & = \frac{ 2 \tan \left( \tan^{-1} \sfrac{5}{13} \right) }
                                                                { 1 - \tan^2 \left( \tan^{-1} \sfrac{5}{13} \right) } \\
                                                       & = \frac{ 2 \cdot \sfrac{5}{13} }{1 - \left( \sfrac{5}{13} \right)^2 } \\
                                                       % & = \frac{10}{13 ( 1 - \sfrac{25}{13^2} )} \\
                                                       % & = \frac{10}{13 - \sfrac{25}{13} } \\
                                                       & = \boxed{ \frac{65}{72} } \\
        \end{align*}

    \end{description}

  \else
    \vspace{5 cm}

    \begin{quote}
      \begin{em}
      \end{em}
    \end{quote}
    \hspace{1 cm} --TO DO
  \fi

\end{document}

